\newcommand{\lecturetitle}[1]{
  \title{01204211 Discrete Mathematics \\ #1}
  \author{Jittat Fakcharoenphol}
  \frame{\titlepage}
}
\newcommand{\Mod}{\,\bmod\,}

\lecturetitle{Lecture 11a: Guassian Elimination and LU Decomposition} 

\begin{frame}
  \frametitle{Review: A Linear System}

  Consider the following system of linear equations:

  \[
  \begin{array}{ccccccr}
    x_1 &+& x_2 &+& x_3 &=& 5\\
    2x_1 &+& x_2 &+& 2x_3 &=& 10\\
    3x_1 &+& x_2 &+& 2x_3 &=& 4
  \end{array}
  \]
  \pause

  Again we can view it as a vector equation:
  \[
  \begin{bmatrix}
    1\\
    2\\
    3
  \end{bmatrix}
  x_1 +
  \begin{bmatrix}
    1\\
    1\\
    1
  \end{bmatrix}
  x_2 +
  \begin{bmatrix}
    1\\
    2\\
    2
  \end{bmatrix}
  x_3
  =
  \begin{bmatrix}
    5\\
    10\\
    4
  \end{bmatrix}
  \]
  \pause

  Solving this system is equivalent to testing if
  \[
    [5,10,4]\in\mathrm{Span}~\{[1,2,3],[1,1,1],[1,2,2]\}.
    \]
\end{frame}

\begin{frame}
  \frametitle{Review: Testing spans}

  \begin{block}{Problem}
    Given a set of $n$-vectors $S=\{\uv_1,\uv_2,\ldots,\uv_k\}$ over
    $\ff$ and an $n$-vector $\vv$, can we check if $\vv\in\vspan~S$?
  \end{block}

  {\small
  {\bf Example:}
  Consider $3$-vectors over $\rf$.  Let
  \[
  \uv_1=[1,2,3],
  \uv_2=[1,1,1], \uv_3=[1,2,2].
  \]
  We would like to check if
  \[
  \vv=[10,13,29]\in\vspan~\{\uv_1,\uv_2,\uv_3\}.
  \]

  Let us define variables $\alpha_1,\alpha_2,\alpha_3$ such that
  $\vv=\alpha_1\uv_1 + \alpha_2\uv_2 + \alpha_3\uv_3$; i.e., we want
  $\alpha_1,\alpha_2,\alpha_3$ to be such that
\[
\alpha_1
\begin{bmatrix}
  1 \\
  2 \\
  3
\end{bmatrix}
+
\alpha_2
\begin{bmatrix}
  1 \\
  1 \\
  1
\end{bmatrix}
+
\alpha_3
\begin{bmatrix}
  1 \\
  2 \\
  2
\end{bmatrix}
=
\begin{bmatrix}
  10 \\
  13 \\
  29
\end{bmatrix}
\]
}
\end{frame}

\begin{frame}
  \frametitle{Matrix form}
  
We can write these constraints down as the following linear equations.
\[
\begin{array}{ccccccl}
  1\alpha_1 &+& 1\alpha_2 &+& 1\alpha_3 &=& 10\\
  2\alpha_1 &+& 1\alpha_2 &+& 2\alpha_3 &=& 13\\
  3\alpha_1 &+& 1\alpha_2 &+& 2\alpha_3 &=& 29\\
\end{array}
\]

We also can, equivalently, write them in matrix form.
\[
\begin{bmatrix}
  1 & 1 & 1 \\
  2 & 1 & 2 \\
  3 & 1 & 2
\end{bmatrix}
\begin{bmatrix}
  \alpha_1 \\
  \alpha_2 \\
  \alpha_3
\end{bmatrix}
=
\begin{bmatrix}
  10 \\
  13 \\
  29
\end{bmatrix}
\]
\end{frame}

\begin{frame}
  \frametitle{Identity Matrix}

  \begin{block}{Definition}
    An $n$-by-$n$ matrix $I$ is an \textcolor{red}{\bf identity
      matrix} if
    \[
    I=
    \only<1>{
    \begin{bmatrix}
      1 & 0 & \cdots & 0 \\
      0 & 1 & \cdots & 0 \\
      \vdots & \vdots &  & \vdots \\
      0 & 0 & \cdots & 1 \\
    \end{bmatrix}
    }
    \onslide<2->{
    \begin{bmatrix}
      a_{11} & a_{12} & \cdots & a_{1n} \\
      a_{21} & a_{22} & \cdots & a_{2n} \\
      \vdots & \vdots &  & \vdots \\
      a_{n1} & a_{n2} & \cdots & a_{nn} \\
    \end{bmatrix},
    }
    \]
    \onslide<2->{
      such that $a_{ii}=1$ for $1\leq i\leq n$ and $a_{ij}=0$ for all $i\neq j$.
    }
  \end{block}
  
  \onslide<3->{
    For any $n$-by-$m$ matrix $A$,
    \[
    IA = A,
    \]
    and for any $m$-by-$n$ matrix $B$,
    \[
    BI = B.
    \]
  }
\end{frame}

\begin{frame}
  \frametitle{Inverses}

  \begin{block}{Definition}
    For an $n$-by-$n$ matrix $A$, the \textcolor{red}{\bf inverse of
      $A$}, denoted by $A^{-1}$, is an $n$-by-$n$ matrix such that
    \[
    AA^{-1}=A^{-1}A=I.
    \]
  \end{block}

  {\small

    {\bf Remark 1:} You can think of an inverse as a matrix that let
    you ``get back'' anything after multiplying with $A$, i.e.,
    \[
    A^{-1}AB = B.
    \]
    
    \onslide<2->{
      {\bf Remark 2:} $A^{-1}$ is {\em both} left and right inverses.  Note
      that if $L$ is a left inverse and $R$ is a right inverse, they must
      be the same, since
      \[
      \onslide<4->{R=IR=(LA)R=}LAR \onslide<3->{=L(AR)=LI=L}.
      \]
    }
  
    \onslide<5->{
      {\bf Remark 3:} $A$ might not have an inverse.
    }
  }
\end{frame}

\begin{frame}
  \frametitle{Solving a system of linear equations}
  For an $n\times n$ matrix $A$ and a system of linear equations
  \[
  A\vect{x} =\vect{b},
  \]
  with $n$ variables $\vect{x}=[x_1,x_2,\ldots,x_n]$,
  if $A^{-1}$ is an inverse of $A$, we can use it to solve for
  $\vect{x}$.

  \pause Note that we can multiply on the left of both sides with
  $A^{-1}$ to obtain
  \[
  \onslide<3->{\vect{x} =} A^{-1}A\vect{x} = A^{-1}\vect{b}.
  \]

  \pause
  \pause
  Thus
  \[
  \vect{x} = A^{-1}\vect{b},
  \]
  is the solution of the system.
\end{frame}

\begin{frame}
    
    \[
    \begin{bmatrix}
      1 & 1 & 1 \\
      2 & 1 & 2 \\
      3 & 1 & 2
    \end{bmatrix}
    \begin{bmatrix}
      \alpha_1 \\
      \alpha_2 \\
      \alpha_3
    \end{bmatrix}
    =
    \begin{bmatrix}
      10 \\
      13 \\
      29
    \end{bmatrix}
    \]

    Let's perform Gaussian elimination.
    
    \pause

    \[
    \begin{bmatrix}
      1 & 1 & 1 \\
      0 & -1 & 0 \\
      0 & -2 & -1
    \end{bmatrix}
    \begin{bmatrix}
      \alpha_1 \\
      \alpha_2 \\
      \alpha_3
    \end{bmatrix}
    =
    \begin{bmatrix}
      10 \\
      -7 \\
      -1
    \end{bmatrix}
    \]
    
    \pause

    \[
    \begin{bmatrix}
      1 & 1 & 1 \\
      0 & 1 & 0 \\
      0 & 0 & -1
    \end{bmatrix}
    \begin{bmatrix}
      \alpha_1 \\
      \alpha_2 \\
      \alpha_3
    \end{bmatrix}
    =
    \begin{bmatrix}
      10 \\
      7 \\
      13
    \end{bmatrix}
    \]
    
    \pause

    We can perform backward substitution to find that $\alpha_1=16,
    \alpha_2=7,$ and $\alpha_3=-13$.

\end{frame}

\begin{frame}
\frametitle{Gaussian elimination and matrix operations}

We will look closer to see how we could ``describe'' the steps from
Gaussian elimination using matrix multiplications.  This would be very
useful later.
We start with
\[
A =
\begin{bmatrix}
  1 & 1 & 1 \\
  2 & 1 & 2 \\
  3 & 1 & 2
\end{bmatrix}.
\]
The first row operation we did is: $R_2\leftarrow R_2-2R_1$

Can we explain this step with a matrix multiplication? \pause I.e., can we find $M$ such that
\[
MA = 
\begin{bmatrix}
  1 & 1 & 1 \\
  0 & -1 & 0 \\
  3 & 1 & 2
\end{bmatrix}.
\]

\end{frame}

\begin{frame}
We currently have
\[
MA =
\begin{bmatrix}
  1 & 1 & 1 \\
  0 & -1 & 0 \\
  3 & 1 & 2
\end{bmatrix}.
\]
The next row operation we did is: $R_3\leftarrow R_3-3R_1$

Can we explain this step with a matrix multiplication? \pause I.e., can we find $M'$ such that
\[
M'MA = 
\begin{bmatrix}
  1 & 1 & 1 \\
  0 & -1 & 0 \\
  0 & -2 & -1
\end{bmatrix}.
\]
\end{frame}

\begin{frame}
We currently have
\[
M'MA =
\begin{bmatrix}
  1 & 1 & 1 \\
  0 & -1 & 0 \\
  0 & -2 & -1
\end{bmatrix}.
\]
The next row operation we did is: $R_3\leftarrow R_3-2R_2$

Can we explain this step with a matrix multiplication? \pause I.e., can we find $M''$ such that
\[
M''M'MA = 
\begin{bmatrix}
  1 & 1 & 1 \\
  0 & -1 & 0 \\
  0 & 0 & -1
\end{bmatrix}.
\]
\end{frame}

\begin{frame}
  \frametitle{Elementary matrices}
  
  {\footnotesize
               
    \begin{tabular}{l|c|c|l}
      Operations & Result & Elementary matrix & Remarks \\
      \hline
      & & & \\
      $R_2\leftarrow R_2 - 2R_1$
      &
      $
      \begin{bmatrix}
        1 & 1 & 1 \\
        0 & -1 & 0 \\
        3 & 1 & 2
      \end{bmatrix}
      $
      &
      $
      E_{12}=
      \begin{bmatrix}
        1 & 0 & 0 \\
        -2 & 1 & 0 \\
        0 & 0 & 1
      \end{bmatrix}
      $
      &
      \\
      & & & \\
      \hline
      & & & \\
      $R_3\leftarrow R_3-3R_1$
      &
      $
      \begin{bmatrix}
        1 & 1 & 1 \\
        0 & -1 & 0 \\
        0 & -2 & -1
      \end{bmatrix}
      $
      &
      $
      E_{13}=
      \begin{bmatrix}
        1 & 0 & 0 \\
        0 & 1 & 0 \\
        -3 & 0 & 1
      \end{bmatrix}
      $
      &
      \\
      & & & \\
      \hline
      & & & \\
      $R_3\leftarrow R_3 - 2R_2$
      &
      $
      \begin{bmatrix}
        1 & 1 & 1 \\
        0 & -1 & 0 \\
        0 & 0 & -1
      \end{bmatrix}
      $
      &
      $
      E_{23}=
      \begin{bmatrix}
        1 & 0 & 0 \\
        0 & 1 & 0 \\
        0 & -2 & 1
      \end{bmatrix}
      $
      &
      \\
      & & & \\
    \end{tabular}
  }
  
\end{frame}

\begin{frame}
Let's denote the final upper triangular matrix by $B$.  Therefore, we
have
\[
E_{23}E_{13}E_{12}A = B,
\]
or equivalently,
\[
\begin{bmatrix}
  1 & 0 & 0 \\
  0 & 1 & 0 \\
  0 & -2 & 1
\end{bmatrix}
\begin{bmatrix}
  1 & 0 & 0 \\
  0 & 1 & 0 \\
  -3 & 0 & 1
\end{bmatrix}
\begin{bmatrix}
  1 & 0 & 0 \\
  -2 & 1 & 0 \\
  0 & 0 & 1
\end{bmatrix}
\begin{bmatrix}
  1 & 1 & 1 \\
  2 & 1 & 2 \\
  3 & 1 & 2
\end{bmatrix}
=
\begin{bmatrix}
  1 & 1 & 1 \\
  0 & -1 & 0 \\
  0 & 0 & -1
\end{bmatrix}.
\]

\end{frame}
\begin{frame}
  Recall that we have
  \[
  E_{12}=
  \begin{bmatrix}
  1 & 0 & 0 \\
  0 & 1 & 0 \\
  0 & -2 & 1
\end{bmatrix}, \ \ 
E_{13}=
\begin{bmatrix}
  1 & 0 & 0 \\
  0 & 1 & 0 \\
  -3 & 0 & 1
\end{bmatrix}, \ \
E_{23}=
\begin{bmatrix}
  1 & 0 & 0 \\
  -2 & 1 & 0 \\
  0 & 0 & 1
\end{bmatrix}.
  \]

  Let $E_{12}^{-1}, E_{13}^{-1}, E_{23}^{-1}$ be inverses of $E_{12},
  E_{13}, E_{23}$, respectively.

It is not hard to see that
\[
E_{12}^{-1} = \pause
\begin{bmatrix}
  1 & 0 & 0 \\
  2 & 1 & 0 \\
  0 & 0 & 1
\end{bmatrix}, \ \ 
E_{13}^{-1} = \pause
\begin{bmatrix}
  1 & 0 & 0 \\
  0 & 1 & 0 \\
  3 & 0 & 1
\end{bmatrix}, \ \ 
E_{23}^{-1} = \pause
\begin{bmatrix}
  1 & 0 & 0 \\
  0 & 1 & 0 \\
  0 & 2 & 1
\end{bmatrix}.
\]
Therefore, we can write
\[
E_{12}^{-1}E_{13}^{-1}E_{23}^{-1}
E_{23}E_{13}E_{12}A = A =
E_{12}^{-1}E_{13}^{-1}E_{23}^{-1}
B,
\]

\end{frame}
\begin{frame}
After working out the multiplication
\[
E_{12}^{-1}E_{13}^{-1}E_{23}^{-1} =
\begin{bmatrix}
  1 & 0 & 0 \\
  2 & 1 & 0 \\
  0 & 0 & 1
\end{bmatrix}
\begin{bmatrix}
  1 & 0 & 0 \\
  0 & 1 & 0 \\
  3 & 0 & 1
\end{bmatrix}
\begin{bmatrix}
  1 & 0 & 0 \\
  0 & 1 & 0 \\
  0 & 2 & 1
\end{bmatrix}
=
\begin{bmatrix}
  1 & 0 & 0 \\
  2 & 1 & 0 \\
  3 & 2 & 1
\end{bmatrix}
\]
we see that
\[
\begin{bmatrix}
  1 & 1 & 1 \\
  2 & 1 & 2 \\
  3 & 1 & 2
\end{bmatrix}
=
\begin{bmatrix}
  1 & 0 & 0 \\
  2 & 1 & 0 \\
  3 & 2 & 1
\end{bmatrix}
\begin{bmatrix}
  1 & 1 & 1 \\
  0 & -1 & 0 \\
  0 & 0 & -1
\end{bmatrix}.
\]

The matrix $A$ is ``factored'' into two matrices.  We denote the first
matrix $L$ (for lower triangular) and the second one $U$ (for upper
triangular).

This is called an {\bf LU decomposition} of $A$.

\end{frame}

\begin{frame}
  \frametitle{Why is an LU decomposition useful? (1)}
\end{frame}

\begin{frame}
  \frametitle{Why is an LU decomposition useful? (2)}
\end{frame}
