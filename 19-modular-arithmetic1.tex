\newcommand{\lecturetitle}[1]{
  \title{01204211 Discrete Mathematics \\ #1}
  \author{Jittat Fakcharoenphol}
  \frame{\titlepage}
}
\newcommand{\Mod}{\,\bmod\,}

\lecturetitle{Lecture 19: Modular arithmetic 1} 

\begin{frame}\frametitle{The jug puzzle}
\end{frame}

\begin{frame}\frametitle{Possibilities}
\end{frame}

\begin{frame}\frametitle{Integer linear combinations}
\end{frame}

\begin{frame}\frametitle{The minimum integer linear combinations}
\end{frame}

\begin{frame}\frametitle{Review}
  In previous lectures, we studied various properties of integers and
  primes, and discussed primality testing algorithms.  In this
  lecture, we will dive deeper into {\bf modular arithmetic}, where we
  work with integers that ``wrap around'' when reaching a particular
  value, called the ``modulus''.
\end{frame}

\begin{frame}\frametitle{An example}
  \begin{tcolorbox}
    Alice was born on July.  Betty was born in the next 7 months.
    Before Betty was born for 3 months, Cathy was born.  Dave was born
    10 months before Cathy.  What is Dave's birth month?
  \end{tcolorbox}

  \pause We shall encode 12 months as numbers from 0 (for January) to
  11 (for December).  Let $a$ be Alice's birth month.  If we denote by
  $b,c,$ and $d$ birth months of Betty, Cathy, and Dave, we can write
  down the conditions as follows:
  \pause
  \begin{eqnarray*}
    a \Mod 12 &=& 6\Mod 12\\
    (a + 7) \Mod 12 &=& b \Mod 12 \\
    (b - 3) \Mod 12 &=& c \Mod 12 \\
    (c - 10) \Mod 12 &=& d \Mod 12
  \end{eqnarray*}
  
  \pause This is a familiar system of linear equations, but with a
  little twist: a ``modulus'' at the end.
\end{frame}

\begin{frame}\frametitle{Congruence}
  To deal with these equations, Carl Friedrich Gauss introduced a
  notation for them, called congruence.  Instead of writing
  \[ (a+7)\Mod 12 = b \Mod 12,\]
  we write
  \[ a+7 \equiv b \pmod {12}.\]
  \pause

  Formally, if
  \[ x\Mod m = y\Mod m, \]
  we can write
  \[ x \equiv y \pmod m.\]
\end{frame}

\begin{frame}\frametitle{The system with the congruence notation}
  Let's rewrite our previous set of equations using this notation:
  \begin{eqnarray*}
    a &\equiv& 6 \pmod {12}\\
    a + 7 &\equiv& b \pmod {12} \\
    b - 3 &\equiv& c \pmod {12} \\
    c - 10 &\equiv& d \pmod {12}
  \end{eqnarray*}

  Now everything looks fairly much like normal equations.  But do they
  behave the same?
\end{frame}

\begin{frame}\frametitle{Addition, subtraction, and multiplication}
  Suppose that, for a positive integer $q$, we know that
  \[ a\equiv b \pmod q, \]
  and
  \[ c\equiv d \pmod q.\]

  It is not hard to show that
  \[ a+c\equiv b+d \pmod q, \]
  \[ a-c\equiv b-d \pmod q, \]
  and
  \[ ac\equiv bd \pmod q. \]

  Thus, we can treat a system of congruences in the same way we deal
  with a system of linear equations, except the division.
\end{frame}
