\newcommand{\lecturetitle}[1]{
  \title{01204211 Discrete Mathematics \\ #1}
  \author{Jittat Fakcharoenphol}
  \frame{\titlepage}
}
\newcommand{\Mod}{\,\bmod\,}

\lecturetitle{Lecture 13b: Eigenvalues and Eigenvectors}

\begin{frame}
  \frametitle{Review: Hamming codes (1)}
  \pause

  The code is defined by the generator matrix
  {\tiny
    \[
    G=
    \begin{bmatrix}
      1 & 0 & 0 & 0 \\
      0 & 1 & 0 & 0 \\
      0 & 0 & 1 & 0 \\
      0 & 0 & 0 & 1 \\
      0 & 1 & 1 & 1 \\
      1 & 0 & 1 & 1 \\
      1 & 1 & 0 & 1
    \end{bmatrix}
    \]
  }
  
  Consider the encoding function $e: GF(2)^4\rightarrow GF(2)^7$.  Let
  $e(\vect{x})=G\vect{x}$.  What is $\ker e$?

  \pause

  What is $\dim \img e$?

  \vspace{1.5in}
\end{frame}

\begin{frame}
  \frametitle{Review: Hamming codes (2)}
  \pause

  The code is defined by the generator matrix
  {\tiny
    $
    G=
    \begin{bmatrix}
      1 & 0 & 0 & 0 \\
      0 & 1 & 0 & 0 \\
      0 & 0 & 1 & 0 \\
      0 & 0 & 0 & 1 \\
      0 & 1 & 1 & 1 \\
      1 & 0 & 1 & 1 \\
      1 & 1 & 0 & 1
    \end{bmatrix}
    $
  }
  
  What can you say about the minimum ``distance''?

  \vspace{2.5in}
\end{frame}

\begin{frame}
  \frametitle{Examples: random walks}
\end{frame}

\begin{frame}
  \frametitle{Examples: differential equations (1)}
  Let's start with a simple system with one variable.
  \[
  \frac{du}{dt} = au,
  \]
  with $u=u(0)$ when $t=0$.
  \vspace{2in}
\end{frame}

\begin{frame}
  \frametitle{Examples: differential equations (2)}
  Now consider a system with two variables $v$ and $w$:
  \[
  \begin{array}{lcr}
    \frac{dv}{dt} &=& 4v - 5w \\
    \frac{dw}{dt} &=& 2v - 3w
  \end{array}
  \]
  with $v=5$ and $w=4$ when $t=0$, \pause or if we let
  $u(t)=\begin{bmatrix}v(t)\\ w(t)\end{bmatrix}$ and
  \[
  A = \begin{bmatrix}
    4 & -5\\
    2 & -3
  \end{bmatrix},
  \]
  we have \[
  \frac{du}{dt} = Au,
  \]
  with $u(0) = \begin{bmatrix} 5 \\ 4 \end{bmatrix}$
\end{frame}

\begin{frame}
  \frametitle{Examples: differential equations (3)}
\end{frame}

\begin{frame}
  \frametitle{Eigenvalues and eigenvectors}
  
  \begin{block}{Definition}
    For an $n$-by-$n$ matrix $A$, a vector $\vv$ is an
    \textcolor{red}{\bf eigenvector} of $A$ if
    \[
    A\vv = \lambda\vv,
    \]
    and $\vv\neq\vect{0}$.
    The scalar $\lambda$ is called an \textcolor{red}{\bf eigenvalue}
    associated with $\vv$.
  \end{block}
\end{frame}

\begin{frame}\
  \frametitle{Example}
  Consider matrix $A=\begin{bmatrix}5 & 7 \\ 5 & 3\end{bmatrix}$.

  If we let $\vv_1=\begin{bmatrix}-1 \\ 1\end{bmatrix}$, we have
  \[
  A\vv_1=
  \begin{bmatrix}
    5 & 7 \\ 5 & 3
  \end{bmatrix}
  \begin{bmatrix}
    -1 \\ 1
  \end{bmatrix}
  = \pause
  \begin{bmatrix}
    -5+7 \\ -5 + 3
  \end{bmatrix}
  = \pause
  \begin{bmatrix}
    2 \\ -2
  \end{bmatrix}
  =
  (-2)\cdot
  \begin{bmatrix}
    -1 \\ 1
  \end{bmatrix}.
  \]

  If we let $\vv_2=\begin{bmatrix}7 \\ 5\end{bmatrix}$, we have
  \[
  A\vv_2=
  \begin{bmatrix}
    5 & 7 \\ 5 & 3
  \end{bmatrix}
  \begin{bmatrix}
    7 \\ 5
  \end{bmatrix}
  = \pause
  \begin{bmatrix}
    35+35 \\ 35 + 15
  \end{bmatrix}
  = \pause
  \begin{bmatrix}
    70 \\ 50
  \end{bmatrix}
  =
  10\cdot
  \begin{bmatrix}
    7 \\ 5
  \end{bmatrix}.
  \]

  \pause
  See demo in colab.
\end{frame}

\begin{frame}
  \frametitle{Invariant subspace}

  \begin{block}{Definition (invariant subspace)}
    For an $n$-by-$n$ matrix $A$, subspace $\V\subseteq \rf^n$ is
    called an \textcolor{red}{\bf invariant subspace} under linear map
    $f(\vect{x})=A\vect{x}$ if for all $\uv\in \V$, $f(\uv)=A\uv\in\V$.
  \end{block}

  \pause
  \begin{block}{Eigenvector}
    If $\vv$ is an eigenvector of matrix $A$, then
    \[
    \vspan~\{\vv\}
    \]
    is a 1-dimensional invariant subspace under linear map defined by
    $A$.
  \end{block}

\end{frame}

\begin{frame}
  \frametitle{Finding eigenvalues and eigenvectors}

  Given $A$, we want to find an eigenvalue $\lambda$ and a vector
  $\uv\neq\vect{0}$ such that
  \[
  A\uv = \lambda\uv.
  \]
  
  \pause
  After some writing, we want to solve this equation
  \[
  (A - \lambda I)\uv = 0,
  \]
  where $\uv\neq 0$.
  
\end{frame}

\begin{frame}
  \frametitle{Review: ranks and invertible matrices}

  Consider an $n$-by-$n$ matrix $A$ and the following linear system of
  equations
  \[
  A\vect{x} = \vect{0}.
  \]

  \pause
  Suppose that there exists $\vect{x}\neq\vect{0}$ that satisfies the
  equation, what can you say about $A$?

  \pause Clearly, $A$ cannot have an inverse because no matrix $B$ can
  bring $\vect{x}$ back from $A\vect{x}=0$. \pause In this case, we
  say that $A$ is \textcolor{red}{\bf singular}.

  \pause
  \pause Equavilent conditions:
  \begin{itemize}
  \item The rank of $A$ is less than $n$.
  \item Rows of $A$ are not linearly independent.
  \item The linear function $f(\vect{x})=A\vect{x}$ is not injective.
  \item $\ker f\neq\{\vect{0}\}$.
  \item $\dim \ker f\neq 0$.
  \end{itemize}
\end{frame}

\begin{frame}
  \frametitle{Finding $\lambda$}
  From this equation
  \[
  (A-\lambda I)\vect{x} = \vect{0}.
  \]
  Since we want it to have nonzero solution $\vect{x}$. Our goal is to
  find $\lambda$ so that $A-\lambda I$ becomes singular.

  \pause Typically, the tool to use is the {\bf determinant}.
  However, we do not cover this topic in this class.  We will look at
  small examples and consider an iterative method instead.
\end{frame}

\begin{frame}
  \frametitle{Example: $2\times 2$ matrix}

  Consider matrix $A=\begin{bmatrix}5 & 7 \\ 5 & 3\end{bmatrix}$.  We
  want to find $\lambda$ such that
  \[
  \begin{bmatrix}
    5-\lambda & 7 \\
    5 & 3-\lambda
  \end{bmatrix}
  \]
  is singular.  \pause This amounts to solving
  \[
  \frac{5-\lambda}{5} = \frac{7}{3-\lambda},
  \]
  \pause
  i.e.,
  \[
  \lambda^2 - 8\lambda - 20 = 0.
  \]
  \pause
  The equation can be re-written as $(\lambda-10)(\lambda+2)=0$; thus,
  it has 2 roots: $10$ and $-2$.

  \pause You can find associated eigenvectors by solving corresponding
  $(A-\lambda I)\vect{x}=\vect{0}$ equations.
\end{frame}

\begin{frame}
  \frametitle{Matrix multiplication (again)}

  Consider matrix $A=\begin{bmatrix}5 & 7 \\ 5 & 3\end{bmatrix}$.
  We know that $A$ has two eigenvectors
  \[
  \vv_1=\begin{bmatrix} 7 \\ 5 \end{bmatrix}, \ \ \ \ \ \
  \vv_2=\begin{bmatrix} -1 \\ 1 \end{bmatrix}.
  \]
  with corresponding eigenvalues $\lambda_1=10$ and $\lambda_2=-2$.

  \pause What can we say about
  \[
  \begin{bmatrix}
    5 & 7 \\ 5 & 3
  \end{bmatrix}
  \begin{bmatrix}
    5 \\ 7
  \end{bmatrix}
  \pause
  =
  \begin{bmatrix}
    5 & 7 \\ 5 & 3
  \end{bmatrix}
  \begin{bmatrix}
    7 - 2 \\ 5 + 2
  \end{bmatrix}
  =
  \pause
  \begin{bmatrix}
    5 & 7 \\ 5 & 3
  \end{bmatrix}
  \left(
  \begin{bmatrix}
    7 \\ 5
  \end{bmatrix}
  +2\cdot
  \begin{bmatrix}
    -1 \\ 1
  \end{bmatrix}
  \right)
  \]
\end{frame}

\begin{frame}
  \frametitle{Matrix multiplication (again and again)}

  {\bf Fact:} An $n$-by-$n$ matrix $A$ has $n$ linearly independent
  eigenvectors $\vv_1,\ldots,\vv_n$ with corresponding eigenvalues
  $\lambda_1,\ldots,\lambda_n$.  (They might not be real vectors.)

  \pause Since $\vv_1,\ldots,\vv_n$ form a basis, for any vector
  $\vect{x}$ there exist $\alpha_1,\alpha_2,\cdots,\alpha_n$ such that
  \[
  \vect{x} = \alpha_1\vv_1 + \alpha_2\vv_2 + \cdots + \alpha_n\vv_n.
  \]

  \pause
  Let's multiply $\vect{x}$ with $A$:
  \[
  \begin{array}{rcl}
    A\vect{x} &=& A(\alpha_1\vv_1 + \alpha_2\vv_2 + \cdots + \alpha_n\vv_n) \\
    &=& A\alpha_1\vv_1 + A\alpha_2\vv_2 + \cdots + A\alpha_n\vv_n \\
    &=& \pause
    \lambda_1\alpha_1\vv_1 + \lambda_2\alpha_2\vv_2 + \cdots + \lambda_n\alpha_n\vv_n.
  \end{array}
  \]
  \pause We can keep multiplying with $A$ many times:
  \[
  A^k\vect{x}=
  \pause
  \lambda_1^k\alpha_1\vv_1 + \lambda_2^k\alpha_2\vv_2 + \cdots + \lambda_n^k\alpha_n\vv_n.
  \]
\end{frame}

\begin{frame}
  \frametitle{The power method}
  If $A$ has eigenvalues $\lambda_1,\lambda_2,\ldots,\lambda_n$ such that
  \[
  |\lambda_1| > |\lambda_i|,
  \]
  for $i\neq 1$.  We call $\lambda_1$ the {\bf dominant eigenvalue}.
  We also call the eigenvectors corresponding to $\lambda_1$ {\bf
    dominant eigenvectors}.

  \vspace{0.2in}
  \begin{block}{The power method (or power iteration)}
    \begin{itemize}
    \item Start with a random vector $\vect{x_0}$.
    \item For $i=0,1,\ldots,k$, \\
      \ \ \ \ Let $\vect{x}_{i+1}=A\vect{x}_i$, with
      probably some scaling.
    \end{itemize}
  \end{block}
\end{frame}
