\newcommand{\lecturetitle}[1]{
  \title{01204211 Discrete Mathematics \\ #1}
  \author{Jittat Fakcharoenphol}
  \frame{\titlepage}
}
\newcommand{\Mod}{\,\bmod\,}

\lecturetitle{Lecture 8b: Vectors} 

\begin{frame}\frametitle{What is a vector?}
  You can think of a {\bf vector} as an ``ordered'' list of elements (which are
  typically numbers).  For example:
  \begin{itemize}
  \item $[1,2,5,20]$
  \item $[0,0,1,1,0,0,0,1]$
  \end{itemize}

  \pause

  You can also view a vector as a {\bf function}, e.g., you can view
  $\uv = [1,2,5,20]$ as a function $\uv$ that maps
  \[
  0 \mapsto 1, \ \
  1 \mapsto 2, \ \
  2 \mapsto 5, \ \
  3 \mapsto 20.
  \]

  \pause
  
  Each element in the vector is typically a real number ($\rf$), but
  can be an element from other sets with appropriate property (more on
  this later).

  \pause

  { \tiny {\bf Remark:} Mathematically, a vector is an element of a
    vector space.  We will understand this more later.}
\end{frame}

\begin{frame}
  \frametitle{What can be represented as a vector?}
\end{frame}

\begin{frame}
  \frametitle{Viewing vectors: vectors in $\rf^2$}
\end{frame}

\begin{frame}
  \frametitle{Viewing vectors: vectors in $\rf^3$}
\end{frame}

\begin{frame}
  \frametitle{$n$-vectors over $\rf$}

  \begin{itemize}
  \item We mostly deal with vectors with finite number of elements.
  \item This is a \textcolor{red}{\bf $4$-vector}: $[10,20,500,4]$.
    \pause
  \item We sometimes also write it as a column vector:
    \[
    \begin{bmatrix}
      10 \\ 20 \\ 500 \\ 4
    \end{bmatrix}
    \]
    \pause
  \item When every element of a vector is from some set, we say that
    it is a vector {\bf over} that set.  For example, $[10,20,500,4]$
    is a $4$-vector over ${\mathbb R}$.
  \end{itemize}
\end{frame}

\begin{frame}
  \frametitle{Vector operations}

  \begin{itemize}
  \item As discussed in the previous slides, when working with a
    system of linear equations, we mostly deals with {\bf linear
      combinations} of vectors.
  \item We will look at the operations we do to vectors to obtain
    their linear combinations.
    \pause
  \item The operations are:
    \begin{itemize}
    \item Vector additions
    \item Scalar multiplications
    \end{itemize}
  \item These operations motivate the definition of vector spaces.
  \end{itemize}
\end{frame}

\begin{frame}
  \frametitle{Vector additions}

  Given two $n$-vectors
  \[
  \uv = [u_1,u_2,\ldots,u_n]
  \]
  and
  \[
  \vv = [v_1,v_2,\ldots,v_n],
  \]
  we have that
  \[
  \uv+\vv = [u_1+v_1,u_2+v_2,\ldots,u_n+v_n].
  \]
\end{frame}

\begin{frame}
  \frametitle{Vector additions, in picture}
\end{frame}

\begin{frame}
  \frametitle{Zero vectors}
  A zero $n$-vector ${\bm 0}=[0,0,\ldots,0]$ is an additive identity, i.e.,
  for any vector $\uv$,
  \[
  {\bm 0} + \uv = \uv + {\bm 0} = \uv. 
  \]
\end{frame}

\begin{frame}
  \frametitle{Scalar multiplications}

  For a vector over $\rf$, we refer to an element $\alpha$ in
  $\rf$ as a scalar.  For an $n$-vector
  \[
  \uv = [u_1,u_2,\ldots,u_n],
  \]
  we have that
  \[
  \alpha\cdot\uv = [\alpha\cdot u_1, \alpha\cdot u_2,\ldots, \alpha\cdot u_n],
  \]
\end{frame}

\begin{frame}
  \frametitle{Scalar multiplications, in pictures}
\end{frame}

\begin{frame}
  \frametitle{Linear combinations}
  For any scalar \
  \[
  \alpha_1,\alpha_2,\ldots,\alpha_m
  \]
  and vectors
  \[
  \uv_1,\uv_2,\ldots,\uv_m,
  \]
  we say that
  \[
  \alpha_1\uv_1 + \alpha_2\uv_2 + \cdots + \alpha_m \uv_m
  \]
  is a \textcolor{red}{\bf linear combination} of $\uv_1,\ldots,\uv_m$.
  \pause
  \vspace{0.2in}

  Examples:
  \vspace{1in}
\end{frame}


\begin{frame}
  \frametitle{A linear system with 3 variables}
  Give the following linear system.

  {\footnotesize
  \[
  \begin{array}{rcrcrcl}
    2x_1 & + & 4x_2 & + & 3x_3 & = & 7 \\
    x_1 & + &  &  & 5x_3 & = & 12 \\
    4x_1 & + & 2x_2 & + & 3x_3 & = & 10
  \end{array}
  \]
  }
  \pause
  If we rewrite the system as

  {\footnotesize
  \[
  \begin{bmatrix}
    2 \\ 1 \\ 4
  \end{bmatrix}
  \cdot x_1 +
  \begin{bmatrix}
    4 \\ 0 \\ 2
  \end{bmatrix}
  \cdot x_2 +
  \begin{bmatrix}
    3 \\ 5 \\ 3
  \end{bmatrix}
  \cdot x_3 +
  =
  \begin{bmatrix}
    7 \\ 12 \\ 10
  \end{bmatrix}.
  \]
  }

  \pause
  This becomes the problem of expressing a vector as linear
  combination of other vectors.  I.e., given vectors
  \[
  \uv_1 = [2,1,4], \ \ \uv_2 = [4,0,2], \ \ \uv_3 = [3,5,3]
  \]
  we would like to find coefficients $x_1,x_2,x_3$ such that
  \[
  x_1\cdot \uv_1 + x_2\cdot \uv_2 + x_3\cdot \uv_3 = [7,12,10].
  \]
\end{frame}


\begin{frame}
  \frametitle{Span}

  A set of all linear combination of vectors $\uv_1,\uv_2,\ldots,\uv_m$ is called the \textcolor{red}{\bf span} of that set of vectors.

  It is denote by $\mathrm{Span} \{\uv_1,\uv_2,\ldots,\uv_m\}$.

  \vspace{0.3in}

  Examples:
  \vspace{2in}
\end{frame}

\begin{frame}
  \frametitle{Convex combination}
  For any scalar \
  \[
  \alpha_1,\alpha_2,\ldots,\alpha_m,
  \]
  such that $\alpha_1+\alpha_2+\ldots+\alpha_m=1$ and $\alpha_i\geq 0$ for all $i$,
  and vectors
  \[
  \uv_1,\uv_2,\ldots,\uv_m,
  \]
  we say that
  \[
  \alpha_1\uv_1 + \alpha_2\uv_2 + \cdots + \alpha_m \uv_m
  \]
  is a \textcolor{red}{\bf convex combination} of $\uv_1,\ldots,\uv_m$.
  \pause
  \vspace{0.2in}

  Examples:
  \vspace{1in}
\end{frame}

\begin{frame}
  \frametitle{Elements in a vector}
  \begin{itemize}
  \item We see examples of vectors over $\rf$.
  \item However, elements in a vector can be from other sets with
    appropriate property.  (I.e., they should behave a real numbers.)
  \item What do we want from an element in a vector?
    \pause
    \begin{itemize}
    \item We should be able to perform addition, subtraction, multiplication, and division.
      \pause
    \item Operations should be commutative and associative.
    \item Additive and multiplicative identity should exist.
    \item Addition and multiplication should have inverses.
    \end{itemize}
    \pause

    \item We refer to a set with these properties as a {\bf field}.
  \end{itemize}
\end{frame}

\begin{frame}
  \frametitle{A field}

  A set $\ff$ with two operations $+$ and $\times$ (or $\cdot$) is a
  \textcolor{red}{\bf field} iff these operations satisfy the
  following properties:
  \begin{itemize}
    \pause
  \item (Associativity): $(a+b)+c = a+(b+c)$ and $(a\cdot b)\cdot c = a\cdot(b\cdot c)$
    \pause
  \item (Commutativity): $a+b=b+a$ and $a\cdot b=b\cdot a$
    \pause
  \item (Identities): There exist two elements $0\in\ff$ and $1\in\ff$ such that $a+0 = a$ and $a\cdot 1 = a$
    \pause
  \item (Additive inverse): For every element $a\in \ff$, there is an element $-a\in \ff$ such that $a+(-a) = 0$
    \pause
  \item (Multiplicative inverse): For every element $a\in \ff\setminus\{0\}$, there is an alement $a^{-1}$ such that $a\cdot a^{-1}=1$
    \pause
  \item (Distributive): $a\cdot(b+c)=a\cdot b + a\cdot c$
  \end{itemize}
\end{frame}

\begin{frame}
  \frametitle{Another useful field: $GF(2)$}
  $GF(2) = \{0,1\}$.  I.e., it is a ``bit'' field.

  What are $+$ and $\cdot$ in $GF(2)$?

  \pause

  \begin{itemize}
  \item We define $b_1+b_2$ to be XOR.
    \pause

    \[
    \begin{array}{c}
      0 + 0 = 0 \\ \pause
      0 + 1 = 1 + 0 = 1 \\ \pause
      1 + 1 = 0
    \end{array}
    \] \pause
  \item We define $b_1\cdot b_2$ to be standard multiplication.
    \pause
    \[
    \begin{array}{c}
      0 \cdot 0 = 0\cdot 1 = 1\cdot 0 = 0 \\ \pause
      1\cdot 1 = 1
    \end{array}
    \] 
    
  \end{itemize}

  \pause

  You can check that $GF(2)$ satisfies the axioms of fields.
\end{frame}

