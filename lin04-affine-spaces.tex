\newcommand{\lecturetitle}[1]{
  \title{01204211 Discrete Mathematics \\ #1}
  \author{Jittat Fakcharoenphol}
  \frame{\titlepage}
}
\newcommand{\Mod}{\,\bmod\,}

\lecturetitle{Lecture 9b: Affine Spaces} 

\begin{frame}
  \frametitle{Review: Linear combinations}

  \begin{block}{Definition}
    For any scalars
    \[
    \alpha_1,\alpha_2,\ldots,\alpha_m
    \]
    and vectors
    \[
    \uv_1,\uv_2,\ldots,\uv_m,
    \]
    we say that
    \[
    \alpha_1\uv_1 + \alpha_2\uv_2 + \cdots + \alpha_m \uv_m
    \]
    is a \textcolor{red}{\bf linear combination} of $\uv_1,\ldots,\uv_m$.
  \end{block}
\end{frame}

\begin{frame}
  \frametitle{Review: Span}

  \begin{block}{Definition}
    A set of all linear combination of vectors $\uv_1,\uv_2,\ldots,\uv_m$ is called the \textcolor{red}{\bf span} of that set of vectors.

    It is denoted by $\mathrm{Span} \{\uv_1,\uv_2,\ldots,\uv_m\}$.
  \end{block}
\end{frame}

\begin{frame}
  \frametitle{Review: Vector spaces}
  \begin{block}{Definition}
    A set $\V$ of vectors over $\ff$ is a \textcolor{red}{\bf vector space} iff
    \begin{itemize}
    \item \textcolor{blue}{(V1)} $\vect{0}\in\V$,
    \item \textcolor{blue}{(V2)} for any $\uv\in\V$,
      \[
      \alpha\cdot\uv\in\V
      \]
      for any
      $\alpha\in\ff$, and
    \item \textcolor{blue}{(V3)} for any $\uv,\vv\in\V$,
      \[
      \uv+\vv\in\V.
      \]
    \end{itemize}
  \end{block}

  Examples of vector spaces:
  \begin{itemize}
  \item A span of vectors is a vector space.
  \item A solution set to homogeneous linear equations is a vector space.
  \end{itemize}
\end{frame}

\begin{frame}
  \frametitle{Translation}

  If we have a line or a plane passing through a vector $\vect{a}$,
  but not through the origin, how can we represent it?

  \pause

  \begin{itemize}
  \item Translate the object so that it passes through the origin.
    \pause
  \item We obtain a vector space $\V$.
    \pause
  \item Then we translate it back so that it passes through $\vect{a}$.
    \pause
  \item We get the set
    \[
    {\mathcal A} = \{ \vect{a} + \vect{u} : \vect{u} \in \V \}
    \]
    \pause
  \item {\em Question:} Is $\mathcal A$ a vector space?
    \pause
  \item We also write it as $\vect{a} + \V$.
  \end{itemize}
\end{frame}

\begin{frame}
  \frametitle{Affine spaces}

  \begin{block}{Definition}
    If $\vect{a}$ is a vector and $\V$ is a vector space, then
    \[
    \vect{a} + \V
    \]
    is an \textcolor{red}{\bf affine space}.
  \end{block}
\end{frame}

\begin{frame}
  \frametitle{An affine space and convex combination: 2 dimensions}
\end{frame}

\begin{frame}
  \frametitle{An affine space and convex combination: 3 dimensions}
\end{frame}

\begin{frame}
  \frametitle{Affine combination}
  
  \begin{block}{Definition}
    For any scalars $\alpha_1,\alpha_2,\ldots,\alpha_m$
    such that 
    \[
    \alpha_1 + \alpha_2 + \ldots + \alpha_m = 1
    \]
    and vectors $\uv_1,\uv_2,\ldots,\uv_m$, we say that a linear combination
    \[
    \alpha_1\uv_1 + \alpha_2\uv_2 + \cdots + \alpha_m \uv_m
    \]
    is an \textcolor{red}{\bf affine combination} of $\uv_1,\ldots,\uv_m$.
  \end{block}

  \pause

  \begin{block}{Definition}
    The set of all affine combinations of 
    vectors $\uv_1,\uv_2,\ldots,\uv_m$
    is called the \textcolor{red}{\bf affine hull} of
    $\uv_1,\uv_2,\ldots,\uv_m$.
  \end{block}

\end{frame}

\begin{frame}
  \frametitle{Convex combination: review}
  
  \begin{block}{Definition}
    For any scalars $\alpha_1,\alpha_2,\ldots,\alpha_m \geq 0$
    such that 
    \[
    \alpha_1 + \alpha_2 + \ldots + \alpha_m = 1
    \]
    and vectors $\uv_1,\uv_2,\ldots,\uv_m$, we say that a linear combination
    \[
    \alpha_1\uv_1 + \alpha_2\uv_2 + \cdots + \alpha_m \uv_m
    \]
    is a \textcolor{red}{\bf convex combination} of $\uv_1,\ldots,\uv_m$.
  \end{block}

  \begin{block}{Definition}
    The set of all convex combinations of 
    vectors $\uv_1,\uv_2,\ldots,\uv_m$
    is called the \textcolor{red}{\bf convex hull} of
    $\uv_1,\uv_2,\ldots,\uv_m$.
  \end{block}

\end{frame}

\begin{frame}
  \frametitle{Writing an affine space using a span}
  \pause

  \begin{block}{An affine space}
    An affine space passing through $\uv_1,\uv_2,\ldots,\uv_n$ is
    \[
    \uv_1 + \mathrm{Span}\ \{\uv_2-\uv_1,\uv_3-\uv_1,\ldots,\uv_n-\uv_1\}.
    \]
  \end{block}
\end{frame}

\begin{frame}
  \frametitle{Non-homogeneous linear system}
  Two linear systems:
  \[
  \begin{array}{rcl}
    \vect{a_1}\cdot\vect{x} &=& b_1 \\
    \vect{a_2}\cdot\vect{x} &=& b_2 \\
    &\vdots&\\
    \vect{a_m}\cdot\vect{x} &=& b_m
  \end{array}
  \ \ \ \ \ \ \ \ \ \ \ \ \ \ \ \ \ \ 
  \begin{array}{rcl}
    \vect{a_1}\cdot\vect{x} &=& 0 \\
    \vect{a_2}\cdot\vect{x} &=& 0 \\
    &\vdots&\\
    \vect{a_m}\cdot\vect{x} &=& 0
  \end{array}
  \]

  What can you say about the solution sets of these two related linear
  systems?

  \pause

  $\vect{0}$ is always a solution to the linear system on the right.

  
  Note: A linear equation whose right-hand-side is zero is called a
  {\bf homogeneous linear equation}.  A system of linear homogeneous
  equations is called a {\bf homogeneous linear system}.
\end{frame}

\begin{frame}
  \frametitle{Solutions of the two systems}

  Recall that if $\uv_1$ and $\uv_2$ are both solutions to the
  non-homogeneous linear system, we have that for any $i$
  \[
  \vect{a}_i \uv_1 - \vect{a}_i \uv_2
  = b_i - b_i = 0 = \vect{a}_i (\uv_1 - \uv_2).
  \]
  \pause

  This implies that $\uv_1-\uv_2$ is a solution to the homogeneous
  linear system.
\end{frame}

\begin{frame}
  Suppose that $\W$ is the set of all solution to the non-homogeneous
  linear system, i.e.,
  \[
  \W = \{\vect{x} : \vect{a}_i\vect{x} = b_i, \ \mbox{for $1\leq i\leq m$}\},
  \]
  and let $\uv\in \W$ be one of the solutions, we have that
  \[
  \{\vv - \uv : \vv\in\W \} 
  \]
  \pause
  is a vector space, because
  \pause
  \[
  \{\vv - \uv : \vv\in\W \}
  =
  \{\vect{x} : \vect{a}_i\vect{x} = 0, \ \mbox{for $1\leq i\leq m$}\}
  \]

  \vspace{0.1in}
  \pause

  In other words,
  \[
  \begin{array}{rcl}
    \W &=& \uv + \{\vv - \uv : \vv\in\W \}  \\
    &=& \uv + \{\vect{x} : \vect{a}_i\vect{x} = 0, \ \mbox{for $1\leq i\leq m$}\},
  \end{array}
  \]
  \pause
  i.e., $\W$ is an affine space.
\end{frame}

\begin{frame}
  \frametitle{Solutions to a non-homogeneous linear system}

  \begin{lemma}
    If the solution set of a linear system is not empty, it is an
    affine space.
  \end{lemma}
\end{frame}
