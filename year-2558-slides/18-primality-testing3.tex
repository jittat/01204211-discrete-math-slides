\newcommand{\lecturetitle}[1]{
  \title{01204211 Discrete Mathematics \\ #1}
  \author{Jittat Fakcharoenphol}
  \frame{\titlepage}
}
\newcommand{\Mod}{\,\bmod\,}

\lecturetitle{Lecture 18: Primality testing (3)} 

\begin{frame}\frametitle{The first algorithm}
  This lecture presents the primality testing algorithm based on the
  Fermat test.

  To perform the test if $q$ is a prime, we pick an integer $a$ from
  the range $\{2,3,\ldots,q-1\}$ and check that the condition
  specified by the Fermat's Little Theorem is satisfied.
  \begin{tcolorbox}
    {\bf PROCEDURE} FermatTest($q$,$a$)\\
    1. {\bf if} $GCD(q,a)\neq 1$ {\bf then} {\bf return} ``COMPOSITE'' {\bf endif}\\
    2. Find $y = a^{q-1}\Mod q$\\
    3. {\bf if} $y\neq 1$ {\bf then}\\
    4. \ \ {\bf return} ``COMPOSITE''\\
    5. {\bf else}\\
    6. \ \ {\bf return} ``PRIME''\\
    7. {\bf endif}
  \end{tcolorbox}
\end{frame}

\begin{frame}\frametitle{Guarantees}
  Let's consider all input/output possibilities.  Let
  $z=$FermatTest($q$,$a$).

  \vspace{0.1in}
  \begin{tabular}{|l|c|c|}
    \hline
    & $z$=PRIME & $z$=COMPOSITE \\
    \hline
    $q$ is prime & \pause Correct & \pause Incorrect \\
    \hline
    $q$ is composite & \pause Incorrect & \pause Correct \\
    \hline
  \end{tabular}

  \vspace{0.2in} \pause However, from the Fermat's Little Theorem, we
  know that if $q$ is prime, the test always return ``PRIME''; thus
  the test is always correct.

  \textcolor{white}{We then need to consider the case when $q$ is composite.  In
  this case, the Fermat's Little Theorem does not provide any
  guarantee.  However, we hope that if $q$ is not a prime, we may be
  able to find $a$ such that FermatTest($q$,$a$) reveals the truth,
  i.e., it returns COMPOSITE.}
\end{frame}

\begin{frame}\frametitle{Guarantees}
  Let's consider all input/output possibilities.  Let
  $z=$FermatTest($q$,$a$).

  \vspace{0.1in}
  \begin{tabular}{|l|c|c|}
    \hline
    & $z$=PRIME & $z$=COMPOSITE \\
    \hline
    $q$ is prime & Correct & \textcolor{white}{Incorrect} \\
    \hline
    $q$ is composite & Incorrect & Correct \\
    \hline
  \end{tabular}

  \vspace{0.2in} However, from the Fermat's Little Theorem, we
  know that if $q$ is prime, the test always return ``PRIME''; thus
  the test is always correct.

  \pause We then need to consider the case when $q$ is
  composite. \pause In this case, the Fermat's Little Theorem does not
  provide any guarantee. \pause However, we hope that if $q$ is not a
  prime, we may be able to find $a$ such that FermatTest($q$,$a$)
  reveals the truth, i.e., it returns COMPOSITE.
\end{frame}

\begin{frame}\frametitle{Witness}
  \begin{itemize}
  \item 
    Let's try to be precise.  For a composite $q$, if $a$ is an
    integer such that FermatTest($q$,$a$) returns COMPOSITE, we say
    that $a$ is a {\bf witness} for $q$.
  \item
    From the code of FermatTest, an integer $a$ can be a witness for
    $q$ in two cases:
    \begin{itemize}
    \item (1) when $GCD(q,a)\neq 1$, or
    \item (2) when $a^{q-1}\Mod q\neq 1$.
    \end{itemize}
  \item Note that if we pick $a$ randomly from the set $\{2,3,\ldots,
    q-1\}$ and composite $q$ is a product of very large primes, it is
    very unlikely that we pick $a$ that satisfies (1) because $a$ must
    be a multiple of some of $q$'s prime factors.
  \end{itemize}
\end{frame}

\begin{frame}\frametitle{Carmichael numbers}
  \begin{itemize}
  \item
    It would be great if, for any composite $q$, we can quickly find
    its witness.  \pause But unfortunately, there exists a composite
    $q$ with very few witnesses. I.e., every witness $a$ for $q$
    shares its factor (i.e., it is such that $gcd(a,q)\neq 1$).
    \pause They are called {\bf Carmichael numbers}.
    \begin{tcolorbox}
      A {\bf Carmichael} number is a composite number $q$ such that
      \[ b^{q-1}\Mod q = 1, \]
      for all integers $1<b<n$ which are relatively prime to $q$.
    \end{tcolorbox}
    \pause
  \item The first Carmichael number is 561.  The next ones are 1105,
    1729, and 2465.
  \end{itemize}
\end{frame}

\begin{frame}\frametitle{Witnesses for non-Carmichael numbers (1)}
  Let's focus on the bright side.  Suppose that $q$ is not Carmichael,
  can we say anything about its witnesses?  Can we say that there are
  plenty of them?
\end{frame}

\begin{frame}\frametitle{Witnesses for non-Carmichael numbers (2)}
  Let $a$ be $q$'s witness such that $gcd(a,q)=1$, i.e., we also have that
  \[ a^{q-1}\Mod q\neq 1. \]

  Let's consider a non-witness $b$, i.e., $b$ is an integer such that
  $gcd(b,q)=1$ and \[ b^{q-1}\Mod q=1. \]

  \pause
  Now, consider $ab\Mod q$.  \pause We have that
  \begin{eqnarray*}
    (ab\Mod q)^{q-1}\Mod q &=& (ab)^{q-1}\Mod q \\
    &=&(a^{q-1}b^{q-1})\Mod q \\
    &=& (a^{q-1}\Mod q)(b^{q-1}\Mod q)\Mod q\\
    &=& a^{q-1}\Mod q\\
    &\neq& 1
  \end{eqnarray*}

  \pause Can we use this to say that there are a lot of witnesses?
\end{frame}

\begin{frame}\frametitle{A lot of witnesses}
  Since we can show that given a witness $a$ and a non-witness $b$,
  $ab$ is also witness, we might be able to construct a lot of
  witnesses from non-witnesses.  \pause
  
  Let $A=\{2,3,\ldots,q-1\}$ be the set of all candidates.  Let $B$ be
  the set of non-witnesses which are relatively prime to $q$, i.e., \[
  B=\{b\in A : (gcd(b,q)=1) \wedge (b^{q-1}\Mod q = 1)\}. \] Let
  $C=\{ab\Mod q : b\in A \}$.  We know that every $c\in C$ is a
  witness.

  \pause If we know that $|C| = |B|$, can you show that the
  probability of choosing a witness from the set $A$ is large (i.e, at
  least $1/2$).

  \pause Now, is it obvious that $|C| = |B|$?  What is missing?
\end{frame}

\begin{frame}\frametitle{The missing argument}
  To show that $|C|=|B|$, we need to argue that when we multiply every
  element of $B$, we do not get duplicate elements.  I.e., we need to
  prove that for $x\in B$ and $y\in B$ such that $x\neq y$,
  \[ ax\Mod q \neq ay\Mod q,\]
  when $gcd(a,q)=1$.

  {\bf Quick check:} prove this statement.

  (Note: From the definitions of $a$ and $B$, you may assume that
  $gcd(q,a)=gcd(q,x)=gcd(q,y)=1$.)
\end{frame}

\begin{frame}\frametitle{Conclusions}
  From the previous discussion, we know that for non-Carmichael
  numbers, the Fermat test succeeds with probability at least $1/2$.

  Further developments:
  \begin{itemize}
  \item In 1976, Miller and Rabin show that one can deal with
    Carmichael numbers, providing the first randomized algorithm for
    testing primes.
  \item In 2002, Agrawal, Kayal, and Saxena devise an $O(m^{12})$-time
    deterministic algorithm for primality testing.
  \end{itemize}
\end{frame}
