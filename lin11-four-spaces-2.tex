\newcommand{\lecturetitle}[1]{
  \title{01204211 Discrete Mathematics \\ #1}
  \author{Jittat Fakcharoenphol}
  \frame{\titlepage}
}
\newcommand{\Mod}{\,\bmod\,}

\lecturetitle{Lecture 11b: Four fundamental subspaces (II)}

\begin{frame}
  \frametitle{What is a matrix?}

  Matrices arise in many places.  We will see that there are
  essentially two ways to look at matrices.
  
  \[
  \left[
    \begin{array}{c|c|c}
      1 & 2 & 3 \\
      4 & 5 & 6 \\
      7 & 8 & 9 \\
      10 & 11 & 12 \\
    \end{array}
    \right]
  =
  \left[
    \begin{array}{ccc}
      1 & 2 & 3 \\
      4 & 5 & 6 \\
      7 & 8 & 9 \\
      10 & 11 & 12 \\
    \end{array}
    \right]
  =
  \left[
    \begin{array}{ccc}
      1 & 2 & 3 \\
      \hline
      4 & 5 & 6 \\
      \hline
      7 & 8 & 9 \\
      \hline
      10 & 11 & 12 \\
    \end{array}
    \right]
  \]
\end{frame}

\begin{frame}
  \frametitle{Four fundamental subspaces}

  \begin{block}{Four fundamental subspaces}
    Given an $m$-by-$n$ matrix $A$, we have the following subspaces
    \begin{itemize}
    \item The column space of $A$ (denoted by ${\mathcal R}(A)$
      {$\subseteq \rf^m$}
      )
    \item The row space of $A$ (denoted by ${\mathcal R}(A^T)$
      {$\subseteq \rf^n$}
      )
    \item The nullspace of $A$
      \[
        {\mathcal N}(A) = \{\vect{x} \;|\; A\vect{x}=\vect{0}\}
        {\subseteq \rf^n}
      \]
    \item The left nullspace of $A$
      \[
        {\mathcal N}(A^T) = \{\vect{y} \;|\; A^T\vect{y}=\vect{0}\}
        {\subseteq \rf^m}
      \]
    \end{itemize}
  \end{block}
\end{frame}

\begin{frame}
  \frametitle{Four fundamental subspaces}
\end{frame}

\begin{frame}
  \frametitle{Ranks}

  \begin{block}{Definition}
    Consider an $m$-by-$n$ matrix $A$.
    \begin{itemize}
    \item The \textcolor{red}{\bf row rank} of $A$ is the maximum
      number of linearly independent rows of $A$.
    \item The \textcolor{red}{\bf column rank} of $A$ is the maximum
      number of linearly independent columns of $A$.
    \end{itemize}
  \end{block}

  \vspace{0.2in}

  {\bf Remark:} The column rank of $A$ is $\dim {\mathcal R}(A)$.  The
  row rank of $A$ is $\dim {\mathcal R}(A^T)$.
\end{frame}

\begin{frame}
  \begin{theorem}
    For any matrix $A$, its row rank equals its column rank.
  \end{theorem}

  \begin{proof}
    Let $r$ be the column rank.  We will show that there are $r$
    $n$-vectors that span its row space.  This implies that the row
    rank is at most $r$.  We can use the same argument again on $A^T$
    to obtain that the column rank is at most the row rank; thus, they
    must be equal.

    \vspace{1.5in}
  \end{proof}
  
\end{frame}

\begin{frame}
  \begin{proof}[Proof (cont.)]
    \vspace{3in}
  \end{proof}
\end{frame}

\begin{frame}
  \frametitle{Rank and nullity}

  Given an $m$-by-$n$ matrix $A$, the rank of $A$ is $\dim~{\mathcal
    R}(A)$.  Let $r$ be the rank of $A$.

  What is $\dim~{\mathcal N}(A)$?

  \vspace{2.5in}
\end{frame}

\begin{frame}
  \frametitle{Dimensions}

  \begin{block}{Four fundamental subspaces}
    Given an $m$-by-$n$ matrix $A$ of rank $r$, we have the following
    subspaces
    \begin{itemize}
    \item The column space of $A$ (denoted by ${\mathcal R}(A)$
      {$\subseteq \rf^m$})

      $\dim~{\mathcal R}(A) = r$.
    \item The row space of $A$ (denoted by ${\mathcal R}(A^T)$
      {$\subseteq \rf^n$})

      $\dim~{\mathcal R}(A^T) = r$.
    \item The nullspace of $A$ (denoted by ${\mathcal N}(A) \subseteq \rf^n$)

      $\dim~{\mathcal N}(A) = n-r$.
    \item The left nullspace of $A$ (denoted by ${\mathcal N}(A^T) \subseteq \rf^m$)

      $\dim~{\mathcal N}(A) = m-r$.
    \end{itemize}
  \end{block}
\end{frame}

