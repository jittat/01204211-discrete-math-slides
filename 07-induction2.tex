\newcommand{\lecturetitle}[1]{
  \title{01204211 Discrete Mathematics \\ #1}
  \author{Jittat Fakcharoenphol}
  \frame{\titlepage}
}
\newcommand{\Mod}{\,\bmod\,}

\lecturetitle{Lecture 7: Mathematical Induction 2}

\begin{frame}\frametitle{Review: Mathematical Induction}
  \begin{tcolorbox}
    Suppose that you want to prove that property $P(n)$ is true for
    every natural number $n$.\\
    
    Suppose that we can prove the following two facts:
    
    {\bf Base case:} $P(1)$ \\
    {\bf Inductive step:} For any $k\geq 1$, $P(k)\Rightarrow P(k+1)$ \\
    
    The {\bf Principle of Mathematical Induction} states that $P(n)$
    is true for every natural number $n$.
  \end{tcolorbox}

  The assumption $P(k)$ in the inductive step is usually referred to
  as {\bf the Induction Hypothesis}.
\end{frame}

\begin{frame}\frametitle{Example 1}
  {\bf \textcolor{blue}{Theorem:}}
  For every natural number $n$, $\sum_{i=1}^n i^2 = \frac{n}{6}(n+1)(2n+1)$
  \vspace{0.15in}
  
  {\bf \textcolor{blue}{Proof:}}
  We prove by induction.  The property that we want to prove $P(n)$
  is ``$\sum_{i=1}^n i^2 = \frac{n}{6}(n+1)(2n+1)$.''
  \vspace{0.15in}
    
  {\bf Base case:} We can plug in $n=1$ to check that $P(1)$ is
  true: $1^2 = \frac{1}{6}(1+1)(2\cdot 1+1)$.
  \vspace{0.15in}

  {\bf Inductive step:} We assume that $P(k)$ is true for $k\geq 1$
  and show that $P(k+1)$ is true.
  \vspace{0.15in}

  We first assume the Induction Hypothesis $P(k)$: $ \sum_{i=1}^k
  i^2 = \frac{k}{6}(k+1)(2k+1)$
  \vspace{0.15in}

  (continue on the next page)
\end{frame}

\begin{frame}\frametitle{Example 1 (cont.)}
  Let's show $P(k+1)$.  We write
  $ \sum_{i=1}^{k+1} i^2 = \left(\sum_{i=1}^k i^2\right) + (k+1)^2 .$
  \vspace{0.15in}

  Using the Induction Hypothesis, we know that this is equal to
  {\small
  \begin{eqnarray*}
    (k/6)(k+1)(2k+1) + (k+1)^2 &=& \frac{(k+1)}{6}(k(2k+1)+6(k+1)) \\
    && \mbox{\tiny \ \ \ \ \ \ \textcolor{blue}{(In this step, we factor out $(k+1)/6$)}} \\
    &=& \frac{(k+1)}{6}(2k^2+7k+6) \\
    &=& \frac{(k+1)}{6}((k+1)+1)(2(k+1)+1).
  \end{eqnarray*}
  }
  
  This implies $P(k+1)$ as required.
  
  From the Principle of Mathematical Induction, this implies that
  $P(n)$ is true for every natural number $n$. $\blacksquare$
\end{frame}

\begin{frame}\frametitle{Not an example (1)}
  \begin{theorem}
    For any set of cows, all cows have the same color.
  \end{theorem}
  \textcolor{blue}{Proof.}

  We prove by induction on the size $n$ of the set of cows.
  
  {\bf Base case:} For $n=1$, clearly for any set of a single cow,
  every cow in the set has the same color.
  
  {\bf Inductive step:} Suppose that for every set of size $k$ of
  cows, all cows in the set have the same color.
  
  We will show that every set of size $k+1$ of cows, all cows in
  this set have the same color.
\end{frame}

\begin{frame}\frametitle{Not an example (2)}
  {\bf Inductive step (cont.):} Consider set $A$ of $k+1$ cows.

  \vspace{2.2in}
  \pause

  Because we have established that the base case and the inductive
  step is true, we can conclude that for any set of cows, all cows
  have the same color. $\blacksquare$
  
\end{frame}

\begin{frame}\frametitle{Not an example (3)}
  Clearly the following theorem cannot be true.

  \begin{tcolorbox}
    \begin{theorem}
      For any set of cows, all cows have the same color.
    \end{theorem}
  \end{tcolorbox}

  What is wrong with its proof based on mathematical induction?
  
\end{frame}

\begin{frame}\frametitle{Unused facts}
  \begin{itemize}
  \item Let's informally think about how proving $P(1)$ and
    $P(k)\Rightarrow P(k+1)$ for all $k\geq 1$ implies that $P(n)$ is
    true for all natural number $n$.
    \vspace{1.5in}
    \pause
  \item
    One may notice that when we prove a statement $P(n)$ for all
    natural number $n$ by induction, during the inductive step where
    we want to show $P(k+1)$ from $P(k)$, we usually have that
    $P(1),P(2),\ldots,P(k)$ is true at hands as well.
    \pause
  \item Then why don't we use them as well?
  \end{itemize}
\end{frame}

\begin{frame}\frametitle{Strong Mathematical Induction}
  \begin{tcolorbox}[title=Strong Induction]
    Suppose that you want to prove that property $P(n)$ is true for
    every natural number $n$.\\
    
    Suppose that we can prove the following two facts:
    
    {\bf Base case:} $P(1)$ \\
    {\bf Inductive step:} For any $k\geq 1$,
    \[P(1)\wedge P(2)\wedge\cdots\wedge P(k)\Rightarrow P(k+1).\]
    
    Then $P(n)$ is true for every natural number $n$.
  \end{tcolorbox}
\end{frame}

