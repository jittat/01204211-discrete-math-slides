\newcommand{\lecturetitle}[1]{
  \title{01204211 Discrete Mathematics \\ #1}
  \author{Jittat Fakcharoenphol}
  \frame{\titlepage}
}
\newcommand{\Mod}{\,\bmod\,}

\lecturetitle{Lecture 7d: Fibonacci sequence} 

\begin{frame}\frametitle{The Fibonacci sequence\footnote{This lecture mostly follows Chapter 4 of [LPV].}}

  \begin{columns}[c]
    \column{.3\textwidth}
    \includegraphics{images/fibonacci.jpg}

    {\tiny Source: https://en.wikipedia.org/wiki/ File:Fibonacci.jpg}
    
    \column{.7\textwidth}
    In 1202, Leonardo Bonacci (known as Fibonacci) asked the following
    question.

    \begin{tcolorbox}
      {\footnotesize ``[A]ssuming that: a newly born pair of rabbits,
        one male, one female, are put in a field; rabbits are able to
        mate at the age of one month so that at the end of its second
        month a female can produce another pair of rabbits; rabbits
        never die and a mating pair always produces one new pair (one
        male, one female) every month from the second month on.''

        ``The puzzle that Fibonacci posed was: how many pairs will
        there be in one year?''}
    \end{tcolorbox}
    
    {\tiny From https://en.wikipedia.org/wiki/Fibonacci\_number}
  \end{columns}
\end{frame}

\begin{frame}
  Let's try to solve Fibonacci's question. \pause

  Let $\spadesuit$ denote a newly born rabit pair, and $\heartsuit$
  denote a mature rabit pair.
  
  \begin{tabular}{c|l|r}
    Month & Rabits & \\ \hline
    1 & $\spadesuit$ & 1 \\ \pause
    2 & $\heartsuit$ & 1\\ \pause
    3 & $\heartsuit$ \pause $\spadesuit$ & 2 \\ \pause
    4 & $\heartsuit$ $\heartsuit$ \pause $\spadesuit$ & 3 \\ \pause
    5 & $\heartsuit$ $\heartsuit$ $\heartsuit$ \pause $\spadesuit$ $\spadesuit$ & 5 \\ \pause
    6 & $\heartsuit$ $\heartsuit$ $\heartsuit$ $\heartsuit$ $\heartsuit$ \pause $\spadesuit$ $\spadesuit$ $\spadesuit$ & 8 \\ \pause
    7 & $\heartsuit$ $\heartsuit$ $\heartsuit$ $\heartsuit$ $\heartsuit$ $\heartsuit$ $\heartsuit$ $\heartsuit$
    \pause $\spadesuit$ $\spadesuit$ $\spadesuit$  $\spadesuit$ $\spadesuit$ & 13
  \end{tabular}

  \vspace{0.1in}
  
  \pause How many rabit pairs do we have at the beginning of the 8th
  month? \pause

  {\small
    \begin{itemize}
    \item
      Surely all 13 rabit pairs we have in the 7th month remain there and
      are all mature.  So, the question is how many newly born rabbit
      pairs that we have. \pause
    \item
      The number of newly born rabbit pairs equals the number of mature
      rabbit pairs we have.  \pause This is also equal to the number of
      rabit pairs that we have in the 6th month: 8.
    \end{itemize}
  }
\end{frame}

\begin{frame}
  Thus, we will have 13+8 rabit pairs at the beginning of the 8th
  month.
  \pause
  
  If we write down the sequence, we get the Fibonacci sequence:
  \[
  1,1,2,3,5,8,13,21,\ldots
  \]
  \pause
  Again, what's the next number in this sequence?  How can you compute it? \pause

  21+13 = 34 is the answer. \pause You take the last two numbers and
  add them up to get the next number.  Why? 
\end{frame}

\begin{frame}
  To be precise, let $F_n$ be the $n$-th number in the Fibonacci
  sequence. (That is, $F_1=1, F_2=1, F_3=2, F_4=3$ and so on.)  We can
  define the $(n+1)$-th number as
  \[
  F_{n+1}=F_n+F_{n-1},
  \]
  for $n=2,3,\ldots$.
  \pause
  Is this enough to completely specify the sequence? \pause

  No, because we do not know how to start.  To get the Fibonacci
  sequence, we need to specify two starting values: $F_1=1$ and
  $F_2=1$ as well.

  Now, you can see that the equation and these special values uniquely
  determine the sequence.  It is also convenient to define $F_0=0$ so
  that the equation works for $n=1$.
\end{frame}

\begin{frame}\frametitle{A recurrence}
  The equation
  \[ F_{n+1}=F_n+F_{n-1} \]
  and the initial values $F_0=0$ and $F_1=1$ specify all values of the
  Fibonacci sequence.  With these two initial values, you can use the
  equation to find the value of any number in the sequence.

  This definition is called a {\bf recurrence}.  Instead of defining
  the value of each number in the sequence explicitly, we do so by
  using the values of other numbers in the sequence.
\end{frame}

\begin{frame}\frametitle{Tilings with 1x1 and 2x1 tiles}
  You have a walk way of length $n$ units.  The width of the walk way
  is 1 unit.  You have unlimited supplies of 1x1 tiles and 2x1 tiles.
  Every tile of the same size is indistinguishable.  In how many ways
  can you tile the walk way?

  Let's consider small cases.
  \begin{itemize}
  \item When $n=1$, there are 1 way.
  \item When $n=2$, there are 2 ways.
  \item When $n=3$, there are 3 ways.
  \item When $n=4$, there are 5 ways.
  \end{itemize}

  Let's define $J_n$ to be the number of ways you can tile a walk way
  of length $n$.  From the example above, we know that $J_1=1$ and $J_2=2$.

  Can you find a formula for general $J_n$?
\end{frame}

\begin{frame}\frametitle{Figuring out the recurrence for $J_n$}
  To figure out the general formula for $J_n$, we can think about the
  first choice we can make when tiling a walk way of length $n$.
  There are two choices:
  \begin{itemize}
  \item (1) We can start placing a 1x1 tile at the beginning, or
  \item (2) We can start placing a 2x1 tile at the beginning.
  \end{itemize}

  In each of the cases, let's think about how many ways we can tile
  the rest of the walk way, provided that the first step is made.

  \vspace{0.1in}
  Note that if we start by placing a 1x1 tile, we are left with a walk
  way of length $n-1$.  From the definition of $J_n$, we know that
  there are $J_{n-1}$ ways to tile the rest of the walk way of length
  $n-1$.  Using similar reasoning, we know that if we start with a 2x1
  tile, there are $J_{n-2}$ ways to tile the rest of the walk way.
\end{frame}

\begin{frame}\frametitle{The recurrence for $J_n$}
  \begin{tcolorbox}
    From the discussion, we have that
    \[ J_n = J_{n-1} + J_{n-2}, \]
    where $J_1=1$ and $J_2=2$.
  \end{tcolorbox}

  Note that this is exactly the same recurrence as the Fibonacci
  sequence, but with different initial values.  In fact, we have that
  \[ J_n = F_{n+1}. \]
\end{frame}

\begin{frame}\frametitle{Identities on Fibonacci numbers}
  There are a lot of identities related to Fibonacci numbers.  Let's
  see the first few values in the sequence:
  \[
  0,1,1,2,3,5,8,13,21,34,55,89,\ldots
  \]

  Now, let's add the first few numbers:
  \begin{eqnarray*}
    0+1 &=& 1\\
    0+1+1 &=& 2\\
    0+1+1+2 &=& 4\\
    0+1+1+2+3 &=& 7\\
    0+1+1+2+3+5 &=& 12\\
    0+1+1+2+3+5+8 &=& 20\\
    0+1+1+2+3+5+8+13 &=& 33
  \end{eqnarray*}
  From this we can formulate the following conjecture:
  \[ F_0+F_1+\cdots+F_n = F_{n+2} - 1.\]
\end{frame}

\begin{frame}
  \textcolor{blue}{Theorem:} For $n\geq 0$, we have that \[ F_0+F_1+\cdots+F_n = F_{n+2}-1.\]

  \textcolor{blue}{Proof:} We shall prove by induction on $n$.  The
  base case has already been demonstrated when we consider small
  values of $n$.
  
  \vspace{0.1in}
  {\bf Inductive Step:} Let's assume that the statement is true for $n=k$, for $k\geq 0$, i.e., assume that
  \[ F_0+F_1+\cdots+F_k = F_{k+2}-1.\]
  We shall prove that the statement is true when $n=k+1$.  This is not
  hard to show.  We write
  \begin{eqnarray*}
    (F_0+F_1+\cdots+F_k)+F_{k+1} &=& (F_{k+2}-1) + F_{k+1} \\
    &=& F_{k+3} - 1,
  \end{eqnarray*}
  as required.  Note that the first step follows from the induction
  hypothesis. $\blacksquare$
\end{frame}
