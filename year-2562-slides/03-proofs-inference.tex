\newcommand{\lecturetitle}[1]{
  \title{01204211 Discrete Mathematics \\ #1}
  \author{Jittat Fakcharoenphol}
  \frame{\titlepage}
}
\newcommand{\Mod}{\,\bmod\,}

\lecturetitle{Lecture 3: Inference rules}

\begin{frame}\frametitle{How to prove a mathematical statement?}
  This lecture covers two fundamental concepts in mathematical proofs:
  \begin{itemize}
  \item Proofs by exhaustion
  \item Inference rules\footnote{The materials on inference rules are from [Rosen].}
  \end{itemize}
\end{frame}

\begin{frame}\frametitle{De Morgan's Laws}
  Given propositions $P$ and $Q$, these are a very useful logical
  equivalences (referred to as the De Morgan's Laws).

  \begin{itemize}
  \item $\neg (P\vee Q)\equiv \neg P \wedge \neg Q$
  \item $\neg (P\wedge Q)\equiv \neg P \vee \neg Q$
  \end{itemize}

  (Note that $\neg$ takes precedence over $\vee$ or $\wedge$.)

  \vspace{0.2in}
  
  How can we prove that the first statement is true?
  \pause

  In this case, since there are not too many cases to consider, we can
  enumerate all the possibilities to show that the proposition is
  true.
\end{frame}

\begin{frame}\frametitle{Proof by exhaustion}
  \begin{tcolorbox}
    For any proposition $P$ and $Q$, $\neg (P\vee Q)\equiv \neg P
    \wedge \neg Q$.
  \end{tcolorbox}
  \begin{proof}
    We will prove by exhaustion.  \pause There are 4 cases as in the truth
    table below.

    \vspace{0.1in}
    
    \begin{tabular}{|c|c||c|c|c|}
      \hline
      $P$ & $Q$ & $P\vee Q$ & $\neg(P\vee Q)$ & $\neg Q \wedge \neg P$ \\
      \hline
      $T$ & $T$ & & & \\
      $T$ & $F$ & & & \\
      $F$ & $T$ & & & \\
      $F$ & $F$ & & & \\
      \hline
    \end{tabular}

    \vspace{0.1in}
    \pause

    Note that for all possible truth values of $P$ and $Q$, $\neg
    (P\vee Q)$ equals $\neg P \wedge \neg Q$.  Thus, the statement is
    true.
  \end{proof}
\end{frame}

\begin{frame}\frametitle{Quick check 1}
  Prove the following statement by exhaustion.
  \begin{tcolorbox}
    For any proposition $P$ and $Q$, $\neg (P\wedge Q)\equiv \neg P
    \vee \neg Q$.
  \end{tcolorbox}
  \vspace{2.5in}
\end{frame}

\begin{frame}\frametitle{Quick check 2}
  Prove the following statement by exhaustion.
  \begin{tcolorbox}
    I have 2 pairs of socks in 2 colors: black and white.  If I pick
    any 3 socks, I will have at least a pair of socks of the same
    color.
  \end{tcolorbox}
  \vspace{1.5in}
  \pause

  This is clearly a brute force method.  Sometimes, even in small
  cases, proofs by exhaustion can be very tedious and error-prone.
\end{frame}

\begin{frame}\frametitle{Logical deduction (1)}
  Consider the following statements:

  \begin{itemize}
  \item It rains.
  \item If it rains, then the road will get wet.
  \item If the road is wet, it will be dangerous to drive very fast.
  \end{itemize}

  \pause
  If we believe in these statements (i.e., if we believe that they are
  all true), is it OK to conclude that:

  \begin{itemize}
  \item It is dangerous to drive very fast.
  \end{itemize}

\end{frame}

\begin{frame}\frametitle{Quick check 3}
  Define propositional variables representing each proposition inside
  these statements and write proposition forms of them.

  \begin{itemize}
  \item It rains.
  \item If it rains, then the road will get wet.
  \item If the road is wet, it will be dangerous to drive very fast.
  \item It is dangerous to drive very fast.
  \end{itemize}

\end{frame}

\begin{frame}\frametitle{Logical deduction (2)}
  Using that proposition variables, our problem translate to the
  following.

  \vspace{2.5in}
  
\end{frame}

\begin{frame}\frametitle{Let's try to prove by exhaustion}
  \pause
  There are 3 variables.  These are all possible cases.

  \vspace{0.1in}

  \begin{tabular}{|c|c|c|}
    \hline
    $R$ & $W$ & $D$ \\
    \hline 
    $T$ & $T$ & $T$ \\
    $T$ & $T$ & $F$ \\
    $T$ & $F$ & $T$ \\
    $T$ & $F$ & $F$ \\
    $F$ & $T$ & $T$ \\
    $F$ & $T$ & $F$ \\
    $F$ & $F$ & $T$ \\
    $F$ & $F$ & $F$ \\
    \hline
  \end{tabular}

  \vspace{0.1in}

  We believe that $R$, $R\Rightarrow W$, and $W\Rightarrow D$ are
  true, and we want to conclude that $D$ must be true.

  \pause

  Proofs by exhaustion can be exhausted... 
\end{frame}

\begin{frame}\frametitle{Valid arguments (1)}
  Very often, the statement we want to prove is in the form:

  \begin{tcolorbox}
    Given:
    \begin{itemize}
    \item Hypothesis 1,
    \item Hypothesis 2,
    \item ...
    \item Hypothesis $n$
    \end{itemize}
    Then:
    \begin{itemize}
    \item Conclusion
    \end{itemize}
  \end{tcolorbox}

  \pause

  We say that the statement is {\bf valid} if when all hypotheses are
  true, the conclusion must be true as well.  In that case, we say
  that the conclusion {\bf logically follows} from the hypotheses.
\end{frame}

\begin{frame}\frametitle{Valid arguments (2)}
  More precisely, to show that conclusion $Q$ logically follows from
  hypotheses $P_1,P_2,\ldots,P_n$, we need to show that

  \[ (P_1\wedge P_2\wedge \cdots\wedge P_n)\Rightarrow Q, \]

  is always true, i.e., is a tautology.
\end{frame}

\begin{frame}\frametitle{An example}
  Consider the following argument:

  \begin{itemize}
  \item Hypotheses: $P$ and $P\Rightarrow Q$
  \item Conclusion: $Q$
  \end{itemize}

  Is this a valid argument?
  \pause

  \vspace{0.2in}
  
  It is.  See the following truth table.
  
  \begin{tabular}{|c|c||c|c|}
    \hline
    $P$ & $Q$ & $P\Rightarrow Q$ \\
    \hline 
    $T$ & $T$ & $T$ \\
    $T$ & $F$ & $F$ \\
    $F$ & $T$ & $T$ \\
    $F$ & $F$ & $T$ \\
    \hline
  \end{tabular}
\end{frame}

\begin{frame}\frametitle{$R/W/D$ again}
  Since we know that the previous argument is valid, maybe we can use
  that ``small'' step in our previous example.

  Recall our hypotheses:
  \begin{itemize}
  \item $R$
  \item $R\Rightarrow W$
  \item $W\Rightarrow D$
  \end{itemize}

  \pause
  
  Using the same reasoning, we can say that from $R$ and $R\Rightarrow
  W$, $W$ logically follows.
  \pause
  
  Then, since we know that $W$ is now true, and $W\Rightarrow D$, we
  can conclude that $D$ must follow.
\end{frame}

\begin{frame}\frametitle{A rule of inference}
  The previous ``small'' valid step that we can use in our argument is
  extremely useful when making arguments.  It is called {\em Modus
    ponens}, and is one of many useful rules of inference.

  \begin{tcolorbox}[title=Modus ponens]
    \begin{tabular}{l}
      $P$\\
      $P\Rightarrow Q$\\
      \hline
      $Q$
    \end{tabular}
  \end{tcolorbox}
  
\end{frame}

\begin{frame}\frametitle{Other rules of inference}
  \begin{columns}
    
    \begin{column}{0.4\textwidth}

      \begin{tcolorbox}[title=Addition]
        \begin{tabular}{l}
          $P$\\
          \hline
          $P\vee Q$
        \end{tabular}
      \end{tcolorbox}
      
      \begin{tcolorbox}[title=Modus tollens]
        \begin{tabular}{l}
          $\neg Q$\\
          $P\Rightarrow Q$\\
          \hline
          $\neg P$
        \end{tabular}
      \end{tcolorbox}
      
      \begin{tcolorbox}[title=Conjuction]
        \begin{tabular}{l}
          $P$\\
          $Q$\\
          \hline
          $P\wedge Q$
        \end{tabular}
      \end{tcolorbox}
      
    \end{column}
    
    \begin{column}{0.5\textwidth}

      \begin{tcolorbox}[title=Simplification]
        \begin{tabular}{l}
          $P\wedge Q$\\
          \hline
          $P$
        \end{tabular}
      \end{tcolorbox}
      
      \begin{tcolorbox}[title=Hypothetical syllogism]
        \begin{tabular}{l}
          $P\Rightarrow Q$\\
          $Q\Rightarrow R$\\
          \hline
          $P\Rightarrow R$
        \end{tabular}
      \end{tcolorbox}
      
      \begin{tcolorbox}[title=Disjunctive syllogism]
        \begin{tabular}{l}
          $P\vee Q$\\
          $\neg P$\\
          \hline
          $Q$
        \end{tabular}
      \end{tcolorbox}
      
    \end{column}
    
  \end{columns}
\end{frame}

\begin{frame}\frametitle{Using inference rules}
  \begin{tcolorbox}
    Argue that $P\Rightarrow Q$, $(P\vee R)$, and $\neg R$ logically
    leads to the conclusion $Q$.
  \end{tcolorbox}
  \pause
  \begin{tabular}{ll}
    {\bf Steps} & {\bf Reasons}\\
    \hline\pause
    1. $P\vee R$ & Hypothesis\pause \\
    2. $\neg R$ & Hypothesis\pause \\
    3. $P$ & Disjunctive syllogism using Step 1 and 2\pause \\
    4. $P\Rightarrow Q$ & Hypothesis\pause \\
    5. $Q$ & Modus ponens using Step 3 and 4.
  \end{tabular}
\end{frame}

\begin{frame}\frametitle{Other useful logical equivalences}
  We have discussed De Morgan's Laws, which are logical
  equivalences. The following logical equivalences are also useful
  when making valid arguments.  (Notes: do not get confused with
  operator $\Leftrightarrow$ and notation $P\equiv Q$.)

  \vspace{0.2in}
  
  \begin{tabular}{l|l}
    Equivalences & Names \\
    \hline
    $\neg (\neg P)\equiv P$ & Double negation law \\
    $(P\vee Q)\wedge R\equiv (P\wedge R)\vee(Q\wedge R)$ & Distributive law\\
    $(P\wedge Q)\vee R\equiv (P\vee R)\wedge(Q\vee R)$ & Distributive law\\
    $P\Rightarrow Q\equiv \neg P\vee Q$ & \\
  \end{tabular}
\end{frame}

\begin{frame}\frametitle{Another example}
  \begin{tcolorbox}
    Argue that $P\Rightarrow R$ and $Q\Rightarrow R$ logically leads
    to the conclusion $(P\vee Q)\Rightarrow R$.
  \end{tcolorbox}
  \pause
  \begin{tabular}{ll}
    {\bf Steps} & {\bf Reasons}\\
    \hline\pause
    1. $P\Rightarrow R$ & Hypothesis\pause \\
    2. $\neg P\vee R$ & Equivalence of Step 1\pause \\
    3. $Q\Rightarrow R$ & Hypothesis\pause \\
    4. $\neg Q\vee R$ & Equivalence of Step 3\pause \\
    5. $(\neg P\vee R)\wedge (\neg Q\vee R)$ & Conjuction of Steps 2 and 4.\pause \\
    6. ... (left as homework)
  \end{tabular}
\end{frame}
