\newcommand{\lecturetitle}[1]{
  \title{01204211 Discrete Mathematics \\ #1}
  \author{Jittat Fakcharoenphol}
  \frame{\titlepage}
}
\newcommand{\Mod}{\,\bmod\,}


\newcommand\sbullet[1][.5]{\mathbin{\vcenter{\hbox{\scalebox{#1}{$\bullet$}}}}}

\lecturetitle{Lecture 11b: Context-free languages and grammars (2)\footnote{Based on lecture notes of {\em Models of Computation} course by Jeff Erickson.}} 
\renewcommand{\epsilon}{\varepsilon}

\newcommand{\czero}{{\mathtt 0}}
\newcommand{\cone}{{\mathtt 1}}
\newcommand{\sigcupgam}{\Sigma\cup\Gamma}

\frame{
  \frametitle{Review: Definition}

  A {\color{red}\bf context-free grammer} consists of the following
  components:

  \begin{itemize}
  \item a finite set $\Sigma$, a set of {\em symbols} (or {\em
    terminals}),
  \item a finite set $\Gamma$ disjoint from $\Sigma$, a set of {\em
    non-terminals} (you can think of them as variables),
  \item a finite set $R$ of {\em production rules} of the form
    $A\rightarrow w$ where $A\in\Gamma$ and $w\in(\sigcupgam)^*$
    is a string of symbols and variable, and
  \item a {\em starting} non-terminal (usually the non-terminal of the first production rule).
  \end{itemize}
}

\frame{
  \frametitle{Review: Applying the rules}

  If you have strings $x,y,z\in(\sigcupgam)^*$ and the production rule
  \[
  A \rightarrow y,
  \]
  You can apply the rule to the string $xAz$.  This yields the string
  \[
  xyz.
  \]
  We use the notation
  \[
  xAz \rightsquigarrow xyz
  \]
  to describe this application.
}

\frame{
  \frametitle{Review: Derivation}

  We say that $z$ derives from $x$ if we can obtain $z$ from $x$ by
  production rule applications, denoted by $x\rightsquigarrow^* z$.

  Formally, for any string $x,z\in (\sigcupgam)^*$, we say that
  $x\rightsquigarrow^* z$ if either
  \begin{itemize}
  \item $x = z$, or
  \item $x \rightsquigarrow y$ and $y\rightsquigarrow^* z$ for some
    string $y\in(\sigcupgam)^*$.
  \end{itemize}

}

\frame{
  \frametitle{Review: $L(w)$}

  The {\em language} $L(w)$ of string $w\in(\sigcupgam)^*$ is the set
  of all strings in $\Sigma^*$ that derive from $w$, i.e.,
  \[
  L(w)=\{ x\in\Sigma^* \;|\; w\rightsquigarrow^* x\}.
  \]

  The language {\color{red} \bf generated by} a context-free grammar
  $G$, denoted by $L(G)$ is the language of its starting non-terminal.

  A language $L$ is {\color{red} \bf context-free} if there exists
  some context-free grammar $G$ such that $L(G)=L$.

}

\frame{
  \frametitle{Review: Parse tree}
  \begin{columns}
    \begin{column}{0.4\textwidth}
      {\small
        \begin{eqnarray*}
          S &\rightarrow& A \;|\; B \\
          A &\rightarrow& \czero A \;|\: \czero C \\
          B &\rightarrow& B \cone \;|\: C \cone \\
          C &\rightarrow& \epsilon \;|\; \czero C\cone
        \end{eqnarray*}
      }
    \end{column}

    \begin{column}{0.6\textwidth}
      \begin{itemize}
      \item $\czero\czero\czero\cone\cone$
      \end{itemize}
      \vspace{1.5in}
    \end{column}
  \end{columns}
  
}

\frame{
  \frametitle{Ambiguity}
  \begin{columns}
    \begin{column}{0.4\textwidth}
      {\small
        \begin{eqnarray*}
          S &\rightarrow& \cone \;|\; S+S \;|\; S*S \\
        \end{eqnarray*}
      }
    \end{column}

    \begin{column}{0.6\textwidth}
      \begin{itemize}
      \item $\cone+\cone+\cone+\cone+\cone$ 
      \end{itemize}
      \vspace{1.75in}
    \end{column}
  \end{columns}

  \begin{itemize}
  \item A string $w$ is {\color{red}\bf ambiguous} with respect to a
    grammar $G$ if more than one parse tree for $w$ exists.
  \item A grammar $G$ is {\color{red}\bf ambiguous} if some string is
    ambiguous with respect to $G$.
  \end{itemize}
}

\frame{
  \frametitle{More example}

  Palindrome in $\{\czero, \cone\}^*$ \pause
  \begin{eqnarray*}
    S &\rightarrow& \czero S \czero \;|\; \cone S \cone \;|\; \cone \;|\; \czero \;|\; \epsilon
  \end{eqnarray*}
}

\frame{

  Consider the following grammar
  \[
  S \longrightarrow \czero S \cone \;|\; \epsilon
  \]

  To show that
  \[
  L(S)=\{\czero^n \cone^n \;|\; n\geq 0\},
  \]
  we have to prove \pause
  \begin{itemize}
  \item $L(S) \supseteq \{\czero^n \cone^n \;|\; n\geq 0\},$ and
  \item $L(S) \subseteq \{\czero^n \cone^n \;|\; n\geq 0\}.$
  \end{itemize}
}

\frame{

  Consider the grammar $S \longrightarrow \czero S \cone
  \;|\; \epsilon$.

  \begin{lemma}
    $S\rightsquigarrow^* \czero^n\cone^n$ for every non-negative integer $n$.
  \end{lemma}
  \begin{proof}
    {\small

      Consider any non-negative integer $n$.

      {\bf Induction Hypothesis:} Assume that for every non-negative
      integer $k<n$, $S\rightsquigarrow^* \czero^k\cone^k$.

      There are two cases to consider. \pause
      \begin{itemize}
      \item Case 1: $n=0.$ \pause
      \item Case 2: $n>0.$ \pause From I.H., we know that
        \[
        S\rightsquigarrow^* \czero^{n-1}\cone^{n-1}.
        \]
        We can apply rule $S\longrightarrow \czero S\cone$ to obtain
        $\czero^n\cone^n$, i.e.,
        \[
        S\longrightarrow \czero S\cone \rightsquigarrow^* \czero\czero^{n-1}\cone^{n-1}\cone=\czero^n\cone^n.
        \]
      \end{itemize}

      In both cases, we conclude that $S\rightsquigarrow^*
      \czero^n\cone^n$, as required.

    }
  \end{proof}
}

\frame{

  Consider the following grammar
  \[
  S \longrightarrow \czero S \cone \;|\; \epsilon
  \]

  \begin{lemma}
    $L(S)=\{\czero^n \cone^n \;|\; n\geq 0\}$
  \end{lemma}
  \begin{proof}
    {\small

      \pause Consider any string $w\in L(S)$.  We show that
      $w=\czero^n\cone^n$ for some non-negative integer $n$.
      
      \pause {\bf I.H.:} Assume that for any string $x\in L(S)$ such
      that $|x|<|w|$, $x=\czero^k\cone^k$ for some non-negative
      integer $k$.

      There are \pause 2 cases:

      {\bf Case 1:} $w=\epsilon$. \pause

      {\bf Case 2:} $w=\czero x \cone$ for some $x\in L(S)$. \pause
      Since $|x|=|w|-2<|w|$, we can apply I.H., and get that
      $x=\czero^k \cone^k$;thus $w=\czero\czero^k\cone^k\cone$, i.e.,
      $w=\czero^n\cone^n$ where $n=k+1$, as required.
    }
  \end{proof}
}

\frame{
  \frametitle{Careful}

  \begin{itemize}
  \item When using inductive proof, you have to ensure that each part
    of the string $w$ is shorter than $w$.
  \item Consider this grammar
    \[
    S \longrightarrow \epsilon \;|\; SS \;|\; \czero S \cone \;|\; \cone S \czero.
    \]
  \item When $w$ is created by rule $S\rightarrow SS$, we know that
    $w=xy$ for $x,y\in L(S)$.
  \item Do we know that $|x|<|w|$ and $|y|<|w|$?
    \pause
  \item We can consider a {\color{red}\bf minimum-length derivation}
    in the proof to avoid this problem.
  \end{itemize}
}

\frame{

  Consider grammar $S \longrightarrow \epsilon \;|\; SS \;|\; \czero S
  \cone \;|\; \cone S \czero$. For every string $w\in L(S)$, we have
  $\#(0,w)=\#(1,w)$, where $\#(a,w)$ is the number of occurrences of
  $a$ in $w$.

  \begin{proof}
    {\small
      Consider $w\in L(S)$.  Fix a minimum-length derivation of $w$.
      
      {\color{red} Induction Hypothesis:} Assume that for any string
      $x\in L(S)$ such that $|x|<|w|$, we have $\#(0,x)=\#(1,x)$.

      There are four cases to consider, depending on the first
      production in this derivation.

      \begin{itemize}
      \item Case 1: The first production is $S\longrightarrow \epsilon$. \pause
      \item Case 2: The first production is $S\longrightarrow \czero S\cone$. \pause
        Case 3: The first production is $S\longrightarrow \cone S\czero$. \pause
        
      \item Case 4: The first production is $S\longrightarrow SS.$ \pause In
        this case $w=xy$ for some $x,y\in L(S)$. \pause Since we
        assume the minimum-length derivation, $x$ and $y$ cannot be
        $\epsilon$ because \pause in that case we can shorten the
        derivation of $w$. \pause

        From I.H., \pause we know that $\#(0,x)=\$(1,x)$ an
        $\#(0,y)=\#(1,y)$; thus,
        \begin{eqnarray*}
          \#(0,w) &=& \#(0,x)+\#(0,y) \\ \pause
          &=& \#(1,x)+\#(1,y) \pause
          = \#(1,w)
        \end{eqnarray*}
        
      \end{itemize}
      
      In all cases, we conclude that $\#(0,w)=\#(1,w)$.
      
    }
  \end{proof}
}

\frame{
  \frametitle{Examples: Not palindromes}

  Strings in $(\czero + \cone)^*$ that are not palindromes.
  
  \begin{eqnarray*}
    S &\longrightarrow& \czero S \czero \;|\; \cone S \cone \;|\; \czero Z\cone \;|\; \cone Z \czero \\
    Z &\longrightarrow& \epsilon \;|\; \czero Z \;|\; \cone Z
  \end{eqnarray*}

  Why does this work?
}

\frame{
  \frametitle{Strings with the same number of $\czero$s and $\cone$s}

  \[
  S \longrightarrow \epsilon \;|\; SS \;|\; \czero S \cone \;|\; \cone S \czero.
  \]

  We already show that every string in $L(S)$ contains the same number
  of $\czero$s and $\cone$s.  Why does it contain all possible
  required strings?

  \vspace{1.5in}
}

\frame{
  \frametitle{Strings in which the number of $\czero$s is greater than
    or equal to the number of $\cone$s}

  We can start with the previous grammar
  \[
  S \longrightarrow \epsilon \;|\; SS \;|\; \czero S \cone \;|\; \cone S \czero.
  \]

  And try to add more rules. \pause
  \[
  S \longrightarrow \epsilon \;|\; SS \;|\; \czero S \cone \;|\; \cone S \czero \;|\; \czero S \;|\; S\czero.
  \]
}

\frame{
  \frametitle{Strings with different numbers of $\czero$s and $\cone$s}

  We can start with the previous grammar $E$ of strings with equal number of $\czero$ and $\cone$.
  \[
  E \longrightarrow \epsilon \;|\; EE \;|\; \czero E \cone \;|\; \cone E \czero.
  \]

  There are two cases. \pause
  \[
  S \longrightarrow O \;|\; I
  \]
  \pause
  \[
  O \longrightarrow E\czero O \;|\; E\czero E
  \]
  How about $I$? \pause
  \[
  I \longrightarrow E\cone I \;|\; E\cone E
  \]
  
}

\frame{
  \frametitle{Balanced parentheses}

  \[
  S \longrightarrow (S) \;|\; SS \;|\; \epsilon
  \]

  \pause
  
  \[
  S \longrightarrow (S)S \;|\; \epsilon
  \]

}

\frame{
  \frametitle{Mutual induction}

  Consider grammar
  \[
  S \longrightarrow \czero A \cone \;|\; \epsilon
  \ \ \ \ \ \ \ \ \ \ \ \
  A \longrightarrow \cone S \czero \;|\; \epsilon
  \]

  What is $L(S)$? \pause

  From inspection, we may guess that $L(S)=(\czero\cone)^*$.  But how
  can we prove that?

  \pause

  To prove $L(S)=(\czero\cone)^*$, we must also prove
  $L(A)=(\cone\czero)^*$ {\em at the same time}.
  
}
