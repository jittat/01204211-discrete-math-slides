\newcommand{\lecturetitle}[1]{
  \title{01204211 Discrete Mathematics \\ #1}
  \author{Jittat Fakcharoenphol}
  \frame{\titlepage}
}
\newcommand{\Mod}{\,\bmod\,}

\lecturetitle{Lecture 11b: Four fundamental subspaces (I)}

\begin{frame}
  \frametitle{What is a matrix?}

  Matrices arise in many places.  We will see that there are
  essentially two ways to look at matrices.
  
  \[
  \left[
    \begin{array}{c|c|c}
      1 & 2 & 3 \\
      4 & 5 & 6 \\
      7 & 8 & 9 \\
      10 & 11 & 12 \\
    \end{array}
    \right]
  =
  \left[
    \begin{array}{ccc}
      1 & 2 & 3 \\
      4 & 5 & 6 \\
      7 & 8 & 9 \\
      10 & 11 & 12 \\
    \end{array}
    \right]
  =
  \left[
    \begin{array}{ccc}
      1 & 2 & 3 \\
      \hline
      4 & 5 & 6 \\
      \hline
      7 & 8 & 9 \\
      \hline
      10 & 11 & 12 \\
    \end{array}
    \right]
  \]
\end{frame}

\begin{frame}
  \frametitle{Vector spaces related to a matrix}

  Consider an $m$-by-$n$ matrix $A$ over $\rf$.

  \pause
  We can view $A$ as
  \begin{itemize}
  \item $n$ columns of $m$-vectors:
  $\vect{c}_1,\vect{c}_2,\ldots,\vect{c}_n$     
  \pause
  \item $m$ rows of $n$-vectors:
  $\vect{r}_1,\vect{r}_2,\ldots,\vect{r}_m$
  \end{itemize}

  \pause When we have a set of vectors, recall that its span forms a
  vector space.

  \vspace{0.2in}
  
  We have
  \pause
  \begin{itemize}
  \item Column space: $\vspan~\{\vect{c}_1,\vect{c}_2,\ldots,\vect{c}_n\} \subseteq \rf^{m}$     
    \pause
  \item Row space: $\vspan~\{\vect{r}_1,\vect{r}_2,\ldots,\vect{r}_m\} \subseteq \rf^{n}$
  \end{itemize}
\end{frame}

\begin{frame}
  \frametitle{Subspaces}

  \begin{block}{Definition}
    Let $\V$ and $\W$ be vector spaces such that $\V\subseteq\W$.  We
    say that $\V$ is a \textcolor{red}{\bf subspace} of $\W$.
  \end{block}

  \pause

  {\bf Examples:}
  \begin{itemize}
  \item $\vspan~\{[1,1]\}$ is a subspace of $\rf^2$.
  \item $\vspan~\{[1,0,0],[0,1,1]\}$ is a subspace of $\rf^3$.
  \item $\vspan~\{[1,0,0],[0,1,1],[1,1,2]\}$ is a subspace of $\rf^3$.
  \end{itemize}
\end{frame}

\begin{frame}
  \frametitle{Example 1}

  Let
  \[
  A =
  \begin{bmatrix}
    1 & 2 & 4 \\
    0 & 1 & 3
  \end{bmatrix}
  \]

  \vspace{0.2in}
  
  \pause
  \begin{itemize}
  \item Column space:
    \[
      {\mathcal R}(A) =
      \{
      \alpha_1[1,0] + \alpha_2[2,1] +\alpha_3[4,3] \;|\; \alpha_1,\alpha_2,\alpha_3\in\rf
      \}
      \pause
      = \rf^2.
      \]
      Note that: $\dim {\mathcal R}(A) = \pause 2$ \pause
    \item Row space:
    \[
      {\mathcal R}(A^T) =
      \{
      \alpha_1[1,2,4] + \alpha_2[0,1,3] \;|\; \alpha_1,\alpha_2\in\rf
      \}
      \pause
      \subseteq \rf^3.
      \]
      Note that: $\dim {\mathcal R}(A^T) = \pause 2$
  \end{itemize}
\end{frame}

\begin{frame}
  \frametitle{Example 1 (cont.)}

  Let
  \[
  A =
  \begin{bmatrix}
    1 & 2 & 4 \\
    0 & 1 & 3
  \end{bmatrix}
  \]

  Is there any other way to obtain vector spaces from $A$?  \pause

  \vspace{0.2in}

  We can think of $A$ as a coefficient matrix of a system of
  homogenous linear equations:
  \[
  A\vect{x} = 0.
  \]
  In this case, we have
  \[
  \begin{bmatrix}
    1 & 2 & 4 \\
    0 & 1 & 3
  \end{bmatrix}
  \begin{bmatrix}
    x_1\\ x_2 \\ x_3
  \end{bmatrix}
  =
  \begin{bmatrix}
    0\\ 0\\ 0
  \end{bmatrix}
  \]
  \pause
  The set of solutions $\{\vect{x} \;|\; A\vect{x}=\vect{0} \}$ form a vector space.
\end{frame}

\begin{frame}
  \frametitle{Example 1 (cont.)}

  Given a matrix $A$, we can look at the matrix-vector product $A\vect{x}$.

  Consider
  \[
  \begin{bmatrix}
    1 & 2 & 4 \\
    0 & 1 & 3
  \end{bmatrix}
  \begin{bmatrix}
    x_1\\ x_2 \\x_3
  \end{bmatrix}.
  \]

  \vspace{2in}
\end{frame}

\begin{frame}
  \frametitle{Four fundamental subspaces}

  \begin{block}{Four fundamental subspaces}
    Given an $m$-by-$n$ matrix $A$, we have the following subspaces
    \begin{itemize}
    \item The column space of $A$ (denoted by ${\mathcal R}(A)$
      \onslide<2->{$\subseteq \rf^m$}
      )
    \item The row space of $A$ (denoted by ${\mathcal R}(A^T)$
      \onslide<3->{$\subseteq \rf^n$}
      )
    \item The nullspace of $A$
      \[
        {\mathcal N}(A) = \{\vect{x} \;|\; A\vect{x}=\vect{0}\}
        \onslide<4->{\subseteq \rf^n}
      \]
    \item The left nullspace of $A$
      \[
        {\mathcal N}(A^T) = \{\vect{y} \;|\; A^T\vect{y}=\vect{0}\}
        \onslide<5->{\subseteq \rf^m}
      \]
    \end{itemize}
  \end{block}
\end{frame}

\begin{frame}
  \frametitle{Linearly independent rows}
  
\end{frame}

\begin{frame}
  \frametitle{Ranks}

  \begin{block}{Definition}
    Consider an $m$-by-$n$ matrix $A$.
    \begin{itemize}
    \item The \textcolor{red}{\bf row rank} of $A$ is the maximum
      number of linearly independent rows of $A$.
    \item The \textcolor{red}{\bf column rank} of $A$ is the maximum
      number of linearly independent columns of $A$.
    \end{itemize}
  \end{block}

  \vspace{0.2in}

  \pause

  {\bf Remark:} The column rank of $A$ is $\dim {\mathcal R}(A)$.  The
  row rank of $A$ is $\dim {\mathcal R}(A^T)$.
\end{frame}

\begin{frame}
  \frametitle{Row rank = Column rank}

  \begin{theorem}
    For any matrix $A$, its row rank equals its column rank.
  \end{theorem}

  \vspace{0.2in}
  {\small We will prove this theorem next time.}
  
\end{frame}
