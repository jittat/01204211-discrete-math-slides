\newcommand{\lecturetitle}[1]{
  \title{01204211 Discrete Mathematics \\ #1}
  \author{Jittat Fakcharoenphol}
  \frame{\titlepage}
}
\newcommand{\Mod}{\,\bmod\,}

\lecturetitle{Lecture 10c: Matrices} 

\begin{frame}
  \frametitle{What is a matrix?}

  Matrices arise in many places.  We will see that there are
  essentially two ways to look at matrices.
  
  \[
  \left[
    \begin{array}{ccc}
      1 & 2 & 3 \\
      4 & 5 & 6 \\
      7 & 8 & 9 \\
      10 & 11 & 12 \\
    \end{array}
    \right]
  \pause
  =
  \left[
    \begin{array}{c|c|c}
      1 & 2 & 3 \\
      4 & 5 & 6 \\
      7 & 8 & 9 \\
      10 & 11 & 12 \\
    \end{array}
    \right]
  \pause
  =
  \left[
    \begin{array}{ccc}
      1 & 2 & 3 \\
      \hline
      4 & 5 & 6 \\
      \hline
      7 & 8 & 9 \\
      \hline
      10 & 11 & 12 \\
    \end{array}
    \right]
  \]
\end{frame}

\begin{frame}
  \frametitle{A matrix from a system of linear equations}
  Consider the following system of linear equations:

  \[
  \begin{array}{ccccccr}
    x_1 &+& x_2 &+& x_3 &=& 5\\
    2x_1 &+& x_2 &+& 2x_3 &=& 10\\
    3x_1 &+& x_2 &+& 2x_3 &=& 4
  \end{array}
  \]
  \pause

  Again we can view it as a vector equation:
  \[
  \begin{bmatrix}
    1\\
    2\\
    3
  \end{bmatrix}
  x_1 +
  \begin{bmatrix}
    1\\
    1\\
    1
  \end{bmatrix}
  x_2 +
  \begin{bmatrix}
    1\\
    2\\
    2
  \end{bmatrix}
  x_3
  =
  \begin{bmatrix}
    5\\
    10\\
    4
  \end{bmatrix}
  \]
\end{frame}

\begin{frame}
  \frametitle{A matrix from a system of linear equations}
  {\small
  From the following system of linear equations
  \[
  \begin{array}{ccccccr}
    x_1 &+& x_2 &+& x_3 &=& 5\\
    2x_1 &+& x_2 &+& 2x_3 &=& 10\\
    3x_1 &+& x_2 &+& 2x_3 &=& 4
  \end{array}
  \]
  
  We can also view variables $x_1,x_2,x_3$ as a vector, i.e., let
  $  \vect{x} = 
  \begin{bmatrix}
    x_1\\
    x_2\\
    x_3
  \end{bmatrix}. $
  }
  
  \pause
  The coefficients form a nice rectangular ``matrix'' $A$:
  \[
  A =
  \begin{bmatrix}
    1 & 1 & 1 \\
    2 & 1 & 2 \\
    3 & 1 & 2
  \end{bmatrix},
  \]
  \pause
  and rewrite the system as
  \[
  \begin{bmatrix}
    1 & 1 & 1 \\
    2 & 1 & 2 \\
    3 & 1 & 2
  \end{bmatrix}
  \begin{bmatrix}
    x_1\\
    x_2\\
    x_3
  \end{bmatrix}
  =
  \begin{bmatrix}
    5\\
    10\\
    4
  \end{bmatrix}
  \]

  
\end{frame}

\begin{frame}
  \frametitle{Matrix-Vector Multiplication}
  How would we understand the multiplication
  \[
  \begin{bmatrix}
    1 & 1 & 1 \\
    2 & 1 & 2 \\
    3 & 1 & 2
  \end{bmatrix}
  \begin{bmatrix}
    x_1\\
    x_2\\
    x_3
  \end{bmatrix}
  \]
  \pause

  {\bf By rows.}  Consider the first row of $A$:
  \pause
  \[
  \begin{bmatrix}
    1 & 1 & 1
  \end{bmatrix}
  \begin{bmatrix}
    x_1\\
    x_2\\
    x_3
  \end{bmatrix}
  \pause
  =1\cdot x_1 + 1\cdot x_2 + 1\cdot x_3.
  \]

  Let's look at another two rows:
  {\footnotesize
  \[
  \begin{bmatrix}
    2 & 1 & 2
  \end{bmatrix}
  \begin{bmatrix}
    x_1\\
    x_2\\
    x_3
  \end{bmatrix}
  \pause
  =2\cdot x_1 + 1\cdot x_2 + 2\cdot x_3,
  \ \ \ \ \ \
  \pause
  \begin{bmatrix}
    3 & 1 & 2
  \end{bmatrix}
  \begin{bmatrix}
    x_1\\
    x_2\\
    x_3
  \end{bmatrix}
  \pause
  =3\cdot x_1 + 1\cdot x_2 + 2\cdot x_3,
  \]
  }
\end{frame}

\begin{frame}
  \frametitle{Matrix-Vector Multiplication {\bf by Rows}}
  We look at matrix-vector multiplication with ``row perspective''.
  This is a common way to view matrix-vector multiplication.
  \[
  \begin{bmatrix}
    \onslide<2->{1 & 1 & 1} \\
    \onslide<3->{2 & 1 & 2} \\
    \onslide<4->{3 & 1 & 2}
  \end{bmatrix}
  \begin{bmatrix}
    x_1\\
    x_2\\
    x_3
  \end{bmatrix}
  =
  \begin{bmatrix}
    \onslide<2->{1\cdot x_1 + 1\cdot x_2 + 1\cdot x_3} \\
    \onslide<3->{2\cdot x_1 + 1\cdot x_2 + 2\cdot x_3} \\
    \onslide<4->{3\cdot x_1 + 1\cdot x_2 + 2\cdot x_3}
  \end{bmatrix}
  \]

  Recall:
  \only<2>{
  \[
  \begin{bmatrix}
    1 & 1 & 1
  \end{bmatrix}
  \begin{bmatrix}
    x_1\\
    x_2\\
    x_3
  \end{bmatrix}
  \pause
  =1\cdot x_1 + 1\cdot x_2 + 1\cdot x_3.
  \]
  }
  \only<3>{
  \[
  \begin{bmatrix}
    2 & 1 & 2
  \end{bmatrix}
  \begin{bmatrix}
    x_1\\
    x_2\\
    x_3
  \end{bmatrix}
  \pause
  =2\cdot x_1 + 1\cdot x_2 + 2\cdot x_3,
  \]
  }
  \only<4>{
  \[
  \begin{bmatrix}
    3 & 1 & 2
  \end{bmatrix}
  \begin{bmatrix}
    x_1\\
    x_2\\
    x_3
  \end{bmatrix}
  \pause
  =3\cdot x_1 + 1\cdot x_2 + 2\cdot x_3,
  \]
  }
\end{frame}


\begin{frame}
  \frametitle{Review: Dot product}

  \begin{block}{Definition}
    For $n$-vectors $\uv=[u_1,u_2,\ldots,u_n]$ and $\vv=[v_1,v_2,\ldots,v_n]$, the {\bf dot product} of $\uv$ and $\vv$, denoted by $\uv\cdot\vv$, is
    \[
    u_1\cdot v_1 + 
    u_2\cdot v_2 +
    \cdots +
    u_n\cdot v_n 
    \]
  \end{block}
\end{frame}

\begin{frame}
  \frametitle{Matrix-Vector Multiplication {\bf by Rows}}

  We look at matrix-vector multiplication with ``row perspective'',
  which can be written nicely with \textcolor{blue}{\bf dot product}.

  I.e., from:
  \[
  \begin{bmatrix}
    1 & 1 & 1 \\
    2 & 1 & 2 \\
    3 & 1 & 2
  \end{bmatrix}
  \begin{bmatrix}
    x_1\\
    x_2\\
    x_3
  \end{bmatrix}
  =
  \begin{bmatrix}
    1\cdot x_1 + 1\cdot x_2 + 1\cdot x_3 \\
    2\cdot x_1 + 1\cdot x_2 + 2\cdot x_3 \\
    3\cdot x_1 + 1\cdot x_2 + 2\cdot x_3
  \end{bmatrix}
  \]

  we have
  \[
  \left[
    \begin{array}{c}
      \vect{r}_1 \\
      \hline
      \vect{r}_2 \\
      \hline
      \vect{r}_3 
    \end{array}
    \right]
  \vect{x}
  =
  \left[
    \begin{array}{c}
      \vect{r}_1\cdot\vect{x} \\
      \hline
      \vect{r}_2\cdot\vect{x} \\
      \hline
      \vect{r}_3\cdot\vect{x}
    \end{array}
    \right],
  \]
  where
  \[
  \vect{r}_1=
  \begin{bmatrix}
    1 & 1 & 1
  \end{bmatrix},
  \ \ \
  \vect{r}_2=
  \begin{bmatrix}
    2 & 1 & 2
  \end{bmatrix},
  \ \ \
  \vect{r_3}=
  \begin{bmatrix}
    3 & 1 & 2
  \end{bmatrix}.
  \]

  \pause
  \begin{block}{Dot-product perspective}
    The matrix-vector product is a vector of {\bf dot products}
    between each rows and the vector.
  \end{block}
\end{frame}
  
\begin{frame}
  \frametitle{Matrix-Vector Multiplication {\bf by Columns}}

  However, another nice way to look at matrix-vector multiplication is
  {\bf by columns}.  Notice that:
  \[
  \begin{bmatrix}
    1 & 1 & 1 \\
    2 & 1 & 2 \\
    3 & 1 & 2
  \end{bmatrix}
  \begin{bmatrix}
    x_1\\
    x_2\\
    x_3
  \end{bmatrix}
  =
  \begin{bmatrix}
    1\cdot x_1 + 1\cdot x_2 + 1\cdot x_3 \\
    2\cdot x_1 + 1\cdot x_2 + 2\cdot x_3 \\
    3\cdot x_1 + 1\cdot x_2 + 2\cdot x_3
  \end{bmatrix}
  \]
  \pause
  can be written as
  \[
  \begin{bmatrix}
    1\\
    2\\
    3
  \end{bmatrix}
  x_1 +
  \begin{bmatrix}
    1\\
    1\\
    1
  \end{bmatrix}
  x_2 +
  \begin{bmatrix}
    1\\
    2\\
    2
  \end{bmatrix}
  x_3
  =
  \begin{bmatrix}
    5\\
    10\\
    4
  \end{bmatrix}
  \]

  \pause
  \begin{block}{Linear combination perspective}
    The matrix-vector product is a {\bf linear combination} of column vectors.
  \end{block}
  
\end{frame} 

\begin{frame}
  \frametitle{Example: Matrix-Vector multiplication}
  \[
  \begin{bmatrix}
    a_{11} & a_{12} & a_{13} \\
    a_{21} & a_{22} & a_{23} \\
    a_{31} & a_{32} & a_{33} \\
    a_{41} & a_{42} & a_{43}
  \end{bmatrix}
  \begin{bmatrix}
    x_1 \\
    x_2 \\
    x_3 
  \end{bmatrix}
  =
  \onslide<2->{
    \left[
      \begin{array}{c}
        \onslide<2->{a_{11}\cdot x_1 + a_{12}\cdot x_2 + a_{13}\cdot x_3} \\
        \onslide<3->{a_{21}\cdot x_1 + a_{22}\cdot x_2 + a_{23}\cdot x_3} \\
        \onslide<4->{a_{31}\cdot x_1 + a_{32}\cdot x_2 + a_{33}\cdot x_3} \\
        \onslide<5->{a_{41}\cdot x_1 + a_{42}\cdot x_2 + a_{43}\cdot x_3} 
      \end{array}
      \right]
  }
  \]
  \[
  \begin{bmatrix}
    a_{11} & a_{12} & a_{13} \\
    a_{21} & a_{22} & a_{23} \\
    a_{31} & a_{32} & a_{33} \\
    a_{41} & a_{42} & a_{43}
  \end{bmatrix}
  \begin{bmatrix}
    x_1 \\
    x_2 \\
    x_3 
  \end{bmatrix}
  =
  \onslide<6->{
  \begin{bmatrix}
    a_{11} \\
    a_{21} \\
    a_{31} \\
    a_{41}
  \end{bmatrix}
  \cdot x_1
  +
  }
  \onslide<7->{
  \begin{bmatrix}
    a_{12} \\
    a_{22} \\
    a_{32} \\
    a_{42}
  \end{bmatrix}
  \cdot x_2
  +
  }
  \onslide<8->{
  \begin{bmatrix}
    a_{13} \\
    a_{23} \\
    a_{33} \\
    a_{43}
  \end{bmatrix}
  \cdot x_3
  }
  \]
\end{frame}

