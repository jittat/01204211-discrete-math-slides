\newcommand{\lecturetitle}[1]{
  \title{01204211 Discrete Mathematics \\ #1}
  \author{Jittat Fakcharoenphol}
  \frame{\titlepage}
}
\newcommand{\Mod}{\,\bmod\,}

\lecturetitle{Lecture 10c: Matrices} 

\begin{frame}
  \frametitle{What is linear algebra?}
\end{frame}

\begin{frame}
  \frametitle{What is a matrix?}

  Matrices arise in many places.  We will see that there are
  essentially two ways to look at matrices.
  
  \[
  \left[
    \begin{array}{ccc}
      1 & 2 & 3 \\
      4 & 5 & 6 \\
      7 & 8 & 9 \\
      10 & 11 & 12 \\
    \end{array}
    \right]
  \pause
  =
  \left[
    \begin{array}{c|c|c}
      1 & 2 & 3 \\
      4 & 5 & 6 \\
      7 & 8 & 9 \\
      10 & 11 & 12 \\
    \end{array}
    \right]
  \pause
  =
  \left[
    \begin{array}{ccc}
      1 & 2 & 3 \\
      \hline
      4 & 5 & 6 \\
      \hline
      7 & 8 & 9 \\
      \hline
      10 & 11 & 12 \\
    \end{array}
    \right]
  \]
\end{frame}

\begin{frame}
  \frametitle{A matrix from a system of linear equations}
  Consider the following system of linear equations:

  \[
  \begin{array}{ccccccr}
    x_1 &+& x_2 &+& x_3 &=& 5\\
    2x_1 &+& x_2 &+& 2x_3 &=& 10\\
    3x_1 &+& x_2 &+& 2x_3 &=& 4
  \end{array}
  \]
  \pause

  Again we can view it as a vector equation:
  \[
  \begin{bmatrix}
    1\\
    2\\
    3
  \end{bmatrix}
  x_1 +
  \begin{bmatrix}
    1\\
    1\\
    1
  \end{bmatrix}
  x_2 +
  \begin{bmatrix}
    1\\
    2\\
    2
  \end{bmatrix}
  x_3
  =
  \begin{bmatrix}
    5\\
    10\\
    4
  \end{bmatrix}
  \]
\end{frame}

\begin{frame}
  \frametitle{A matrix from a system of linear equations}
  {\small
  From the following system of linear equations
  \[
  \begin{array}{ccccccr}
    x_1 &+& x_2 &+& x_3 &=& 5\\
    2x_1 &+& x_2 &+& 2x_3 &=& 10\\
    3x_1 &+& x_2 &+& 2x_3 &=& 4
  \end{array}
  \]
  
  We can also view variables $x_1,x_2,x_3$ as a vector, i.e., let
  $  \vect{x} = 
  \begin{bmatrix}
    x_1\\
    x_2\\
    x_3
  \end{bmatrix}. $
  }
  
  \pause
  The coefficients form a nice rectangular ``matrix'' $A$:
  \[
  A =
  \begin{bmatrix}
    1 & 1 & 1 \\
    2 & 1 & 2 \\
    3 & 1 & 2
  \end{bmatrix},
  \]
  \pause
  and rewrite the system as
  \[
  \begin{bmatrix}
    1 & 1 & 1 \\
    2 & 1 & 2 \\
    3 & 1 & 2
  \end{bmatrix}
  \begin{bmatrix}
    x_1\\
    x_2\\
    x_3
  \end{bmatrix}
  =
  \begin{bmatrix}
    5\\
    10\\
    4
  \end{bmatrix}
  \]
\end{frame}

\begin{frame}
  \frametitle{Size}
  \[
  \begin{bmatrix}
    1 & 1 & 1 & 1\\
    2 & 1 & 2 & 5\\
    3 & 1 & 2 & 4
  \end{bmatrix}
  \]
  \pause

  The {\bf size} of a matrix is determined by the number of rows and
  columns.  A matrix with $m$ rows and $n$ columns is referred to as
  an $m$-by-$n$ matrix or an $m\times n$ matrix.  We refers to $m$ and
  $n$ as its {\bf dimensions}.
\end{frame}

\begin{frame}
  \frametitle{Matrix-Vector Multiplication}
  How would we understand the multiplication
  \[
  \begin{bmatrix}
    1 & 1 & 1 \\
    2 & 1 & 2 \\
    3 & 1 & 2
  \end{bmatrix}
  \begin{bmatrix}
    x_1\\
    x_2\\
    x_3
  \end{bmatrix}
  \]
  \pause

  {\bf By rows.}  Consider the first row of $A$:
  \pause
  \[
  \begin{bmatrix}
    1 & 1 & 1
  \end{bmatrix}
  \begin{bmatrix}
    x_1\\
    x_2\\
    x_3
  \end{bmatrix}
  \pause
  =1\cdot x_1 + 1\cdot x_2 + 1\cdot x_3.
  \]

  Let's look at another two rows:
  {\footnotesize
  \[
  \begin{bmatrix}
    2 & 1 & 2
  \end{bmatrix}
  \begin{bmatrix}
    x_1\\
    x_2\\
    x_3
  \end{bmatrix}
  \pause
  =2\cdot x_1 + 1\cdot x_2 + 2\cdot x_3,
  \ \ \ \ \ \
  \pause
  \begin{bmatrix}
    3 & 1 & 2
  \end{bmatrix}
  \begin{bmatrix}
    x_1\\
    x_2\\
    x_3
  \end{bmatrix}
  \pause
  =3\cdot x_1 + 1\cdot x_2 + 2\cdot x_3,
  \]
  }
\end{frame}

\begin{frame}
  \frametitle{Matrix-Vector Multiplication {\bf by Rows}}
  We look at matrix-vector multiplication with ``row perspective''.
  This is a common way to view matrix-vector multiplication.
  \[
  \begin{bmatrix}
    \onslide<2->{1 & 1 & 1} \\
    \onslide<3->{2 & 1 & 2} \\
    \onslide<4->{3 & 1 & 2}
  \end{bmatrix}
  \begin{bmatrix}
    x_1\\
    x_2\\
    x_3
  \end{bmatrix}
  =
  \begin{bmatrix}
    \onslide<2->{1\cdot x_1 + 1\cdot x_2 + 1\cdot x_3} \\
    \onslide<3->{2\cdot x_1 + 1\cdot x_2 + 2\cdot x_3} \\
    \onslide<4->{3\cdot x_1 + 1\cdot x_2 + 2\cdot x_3}
  \end{bmatrix}
  \]

  Recall:
  \only<2>{
  \[
  \begin{bmatrix}
    1 & 1 & 1
  \end{bmatrix}
  \begin{bmatrix}
    x_1\\
    x_2\\
    x_3
  \end{bmatrix}
  \pause
  =1\cdot x_1 + 1\cdot x_2 + 1\cdot x_3.
  \]
  }
  \only<3>{
  \[
  \begin{bmatrix}
    2 & 1 & 2
  \end{bmatrix}
  \begin{bmatrix}
    x_1\\
    x_2\\
    x_3
  \end{bmatrix}
  \pause
  =2\cdot x_1 + 1\cdot x_2 + 2\cdot x_3,
  \]
  }
  \only<4>{
  \[
  \begin{bmatrix}
    3 & 1 & 2
  \end{bmatrix}
  \begin{bmatrix}
    x_1\\
    x_2\\
    x_3
  \end{bmatrix}
  \pause
  =3\cdot x_1 + 1\cdot x_2 + 2\cdot x_3,
  \]
  }
\end{frame}


\begin{frame}
  \frametitle{Review: Dot product}

  \begin{block}{Definition}
    For $n$-vectors $\uv=[u_1,u_2,\ldots,u_n]$ and $\vv=[v_1,v_2,\ldots,v_n]$, the {\bf dot product} of $\uv$ and $\vv$, denoted by $\uv\cdot\vv$, is
    \[
    u_1\cdot v_1 + 
    u_2\cdot v_2 +
    \cdots +
    u_n\cdot v_n 
    \]
  \end{block}
\end{frame}

\begin{frame}
  \frametitle{Matrix-Vector Multiplication {\bf by Rows}}

  We look at matrix-vector multiplication with ``row perspective'',
  which can be written nicely with \textcolor{blue}{\bf dot product}.

  I.e., from:
  \[
  \begin{bmatrix}
    1 & 1 & 1 \\
    2 & 1 & 2 \\
    3 & 1 & 2
  \end{bmatrix}
  \begin{bmatrix}
    x_1\\
    x_2\\
    x_3
  \end{bmatrix}
  =
  \begin{bmatrix}
    1\cdot x_1 + 1\cdot x_2 + 1\cdot x_3 \\
    2\cdot x_1 + 1\cdot x_2 + 2\cdot x_3 \\
    3\cdot x_1 + 1\cdot x_2 + 2\cdot x_3
  \end{bmatrix}
  \]

  we have
  \[
  \left[
    \begin{array}{c}
      \vect{r}_1 \\
      \hline
      \vect{r}_2 \\
      \hline
      \vect{r}_3 
    \end{array}
    \right]
  \vect{x}
  =
  \left[
    \begin{array}{c}
      \vect{r}_1\cdot\vect{x} \\
      \hline
      \vect{r}_2\cdot\vect{x} \\
      \hline
      \vect{r}_3\cdot\vect{x}
    \end{array}
    \right],
  \]
  where
  \[
  \vect{r}_1=
  \begin{bmatrix}
    1 & 1 & 1
  \end{bmatrix},
  \ \ \
  \vect{r}_2=
  \begin{bmatrix}
    2 & 1 & 2
  \end{bmatrix},
  \ \ \
  \vect{r_3}=
  \begin{bmatrix}
    3 & 1 & 2
  \end{bmatrix}.
  \]

  \pause
  \begin{block}{Dot-product perspective}
    The matrix-vector product is a vector of {\bf dot products}
    between each rows and the vector.
  \end{block}
\end{frame}
  
\begin{frame}
  \frametitle{Matrix-Vector Multiplication {\bf by Columns}}

  However, another nice way to look at matrix-vector multiplication is
  {\bf by columns}.  Notice that:
  \[
  \begin{bmatrix}
    1 & 1 & 1 \\
    2 & 1 & 2 \\
    3 & 1 & 2
  \end{bmatrix}
  \begin{bmatrix}
    x_1\\
    x_2\\
    x_3
  \end{bmatrix}
  =
  \begin{bmatrix}
    1\cdot x_1 + 1\cdot x_2 + 1\cdot x_3 \\
    2\cdot x_1 + 1\cdot x_2 + 2\cdot x_3 \\
    3\cdot x_1 + 1\cdot x_2 + 2\cdot x_3
  \end{bmatrix}
  \]
  \pause
  can be written as
  \[
  \begin{bmatrix}
    1\\
    2\\
    3
  \end{bmatrix}
  x_1 +
  \begin{bmatrix}
    1\\
    1\\
    1
  \end{bmatrix}
  x_2 +
  \begin{bmatrix}
    1\\
    2\\
    2
  \end{bmatrix}
  x_3
  =
  \begin{bmatrix}
    5\\
    10\\
    4
  \end{bmatrix}
  \]

  \pause
  \begin{block}{Linear combination perspective}
    The matrix-vector product is a {\bf linear combination} of column vectors.
  \end{block}
  
\end{frame} 

\begin{frame}
  \frametitle{Two perspectives: Matrix-Vector multiplication}

  {\bf Dot products between rows and the vector}
  \[
  \begin{bmatrix}
    a_{11} & a_{12} & a_{13} \\
    a_{21} & a_{22} & a_{23} \\
    a_{31} & a_{32} & a_{33} \\
    a_{41} & a_{42} & a_{43}
  \end{bmatrix}
  \begin{bmatrix}
    x_1 \\
    x_2 \\
    x_3 
  \end{bmatrix}
  =
  \onslide<2->{
    \left[
      \begin{array}{c}
        \onslide<2->{a_{11}\cdot x_1 + a_{12}\cdot x_2 + a_{13}\cdot x_3} \\
        \onslide<3->{a_{21}\cdot x_1 + a_{22}\cdot x_2 + a_{23}\cdot x_3} \\
        \onslide<4->{a_{31}\cdot x_1 + a_{32}\cdot x_2 + a_{33}\cdot x_3} \\
        \onslide<5->{a_{41}\cdot x_1 + a_{42}\cdot x_2 + a_{43}\cdot x_3} 
      \end{array}
      \right]
  }
  \]

  {\bf Linear combination of column vectors}
  \[
  \begin{bmatrix}
    a_{11} & a_{12} & a_{13} \\
    a_{21} & a_{22} & a_{23} \\
    a_{31} & a_{32} & a_{33} \\
    a_{41} & a_{42} & a_{43}
  \end{bmatrix}
  \begin{bmatrix}
    x_1 \\
    x_2 \\
    x_3 
  \end{bmatrix}
  =
  \onslide<6->{
  \begin{bmatrix}
    a_{11} \\
    a_{21} \\
    a_{31} \\
    a_{41}
  \end{bmatrix}
  \cdot x_1
  +
  }
  \onslide<7->{
  \begin{bmatrix}
    a_{12} \\
    a_{22} \\
    a_{32} \\
    a_{42}
  \end{bmatrix}
  \cdot x_2
  +
  }
  \onslide<8->{
  \begin{bmatrix}
    a_{13} \\
    a_{23} \\
    a_{33} \\
    a_{43}
  \end{bmatrix}
  \cdot x_3
  }
  \]

  \onslide<9->{
    \begin{block}{Dimensions}
      If the matrix has $n$ columns, the vector should be an
      $n$-vector.
    \end{block}
  }
\end{frame}

\begin{frame}
  \frametitle{Document search}
  \begin{itemize}
  \item You have 1,000,000 documents in a library.  Given another
    document, you would like to find similar documents from the
    library.  How can you do that?
    \pause
  \item You need some way to measure document {\bf similarity}.
    \pause
  \item Suppose that you nave $N$ documents in the library:
    $d_1,d_2,\ldots, d_N$.  Given a query document $q$, you want to
    find document $d_i$ that maximize
    \[
    sim(d_i,q),
    \]
    where $sim(d,d')$ is the similarity score between documents $d$
    and $d'$.
  \end{itemize}
\end{frame}

\begin{frame}
  \frametitle{Document vector models}

  What is a document? \pause It's just a list of words.  \pause If you
  throw all the ordering away, a document is simply a set of words.

  \pause Let's start with an example.  Suppose that we only care about
  5 words: \textcolor{blue}{{\tt dog}, {\tt cat}, {\tt food}, {\tt
      restaurant},} and \textcolor{blue}{\tt coffee}.

  \pause Consider the following 4 (very short) documents:

  \begin{itemize}
  \item $d_1$: {\tiny People love pets.  Most famous pets are cats and
    dogs.} \\
    \onslide<6->{$d_1=\{\mbox{\tt dog}, \mbox{\tt cat}\}$}
  \item $d_2$: {\tiny Bar Mai has many restaurants with cheap foods.} \\
    \onslide<7->{$d_2=\{\mbox{\tt restaurant}, \mbox{\tt food}\}$}
  \item $d_3$: {\tiny Cat cafe used to be popular in Thailand.  People
    buy coffee and play with cats there.} \\
    \onslide<8->{$d_3=\{\mbox{\tt coffee}, \mbox{\tt cat}\}$}
  \item $d_4$: {\tiny Dogs are human's best friends.  They were around
    in civilization for a long long time.} \\
    \onslide<9->{$d_4=\{\mbox{\tt dog}\}$}
  \end{itemize}

  \onslide<10->{
  How can we translate these sets into vectors?
  }
\end{frame}

\begin{frame}
  \frametitle{Document vector models}

  We assign a fixed co-ordinate for each word, and if a set contain a particular word, we put $1$ in that co-ordinate.

  \pause Here are our 5 words: \textcolor{blue}{{\tt dog}, {\tt cat},
    {\tt food}, {\tt restaurant},} and \textcolor{blue}{\tt coffee}.

  Each document becomes:

  \begin{itemize}
  \item $d_1$: {\tiny People love pets.  Most famous pets are cats and
    dogs.} \\
    $d_1=\{\mbox{\tt dog}, \mbox{\tt cat}\}$ \\
    \pause $\vect{d}_1=[1,1,0,0,0]$
    \pause
  \item $d_2$: {\tiny Bar Mai has many restaurants with cheap foods.} \\
    $d_2=\{\mbox{\tt restaurant}, \mbox{\tt food}\}$\\
    \pause $\vect{d}_2=[0,0,1,1,0]$
    \pause
  \item $d_3$: {\tiny Cat cafe used to be popular in Thailand.  People
    buy coffee and play with cats there.} \\
    $d_3=\{\mbox{\tt coffee}, \mbox{\tt cat}\}$ \\
    \pause $\vect{d}_3=[0,1,0,0,1]$
    \pause
  \item $d_4$: {\tiny Dogs are human's best friends.  They were around
    in civilization for a long long time.} \\
    $d_4=\{\mbox{\tt dog}\}$\\
    \pause $\vect{d}_4=[1,0,0,0,0]$
  \end{itemize}

\end{frame}

\begin{frame}
  \frametitle{Document vector models}
  
  Words: \textcolor{blue}{{\tt dog}, {\tt cat}, {\tt food}, {\tt
      restaurant},} and \textcolor{blue}{\tt coffee}.

  Suppose that we have query document:

  $q$: {\tiny I love cats and coffee.  What restaurant should I
    visit?} \\
  \pause as a set:
  $q=\{\mbox{\tt cat}, \mbox{\tt coffee}, \mbox{\tt restaurant}\}$ \\
  \pause as a vector:
  $\vect{q}=[0,1,0,1,1]$

  \pause
  \vspace{0.2in}
  {\small
  Our documents are:
  \begin{itemize}
  \item $d_1$: {\tiny People love pets.  Most famous pets are cats and
    dogs.} \\
    $d_1=\{\mbox{\tt dog}, \mbox{\tt cat}\}$ \ \ \ \ $\vect{d}_1=[1,1,0,0,0]$
  \item $d_2$: {\tiny Bar Mai has many restaurants with cheap foods.} \\
    $d_2=\{\mbox{\tt restaurant}, \mbox{\tt food}\}$ \ \ \ \ $\vect{d}_2=[0,0,1,1,0]$
  \item $d_3$: {\tiny Cat cafe used to be popular in Thailand.  People
    buy coffee and play with cats there.} \\
    $d_3=\{\mbox{\tt coffee}, \mbox{\tt cat}\}$ \ \ \ \ $\vect{d}_3=[0,1,0,0,1]$
  \item $d_4$: {\tiny Dogs are human's best friends.  They were around
    in civilization for a long long time.} \\
    $d_4=\{\mbox{\tt dog}\}$ \ \ \ \ $\vect{d}_4=[1,0,0,0,0]$
  \end{itemize}
  }
  
  \pause
  How can we define ``similarity'' measure?
\end{frame}

\begin{frame}
  \frametitle{Dot products as a similarity measure}

  From the previous example, we see that the dot products between
  $\vect{d}_i$'s and $\vect{q}$ count the number of common words.

  \pause
  \vspace{0.1in}
  This simple idea can be extended in many ways.

  \pause
  \begin{itemize}
  \item We can increase our ``dictionary'''s size to include more
    words.
    \pause
  \item We can group similar words into the same ``co-ordinates''.
    \pause
  \item In fact, the dot product measures the ``angle'' between
    vectors.  For vectors over $\rf$, we have that
    \[
    \uv\cdot\vv = |\uv||\vv|\cos\theta,
    \]
    where $\theta$ is the angle between vectors $\uv$ and $\vv$.
  \end{itemize}
\end{frame}

\begin{frame}
  \frametitle{Computing all similarity scores}

  If we have documents $\vect{d}_1,\vect{d}_1,\ldots,\vect{d}_N$, as
  vectors, and a query $\vect{q}$, how can we compute all similarity
  scores?

  \pause
  By performing matrix-vector multiplication:
  
  \[
  \left[
    \begin{array}{c}
      \ \ \ \ \ \vect{d}_1 \ \ \ \ \ \\
      \hline
      \vect{d}_2 \\
      \hline
      \vdots \\
      \hline
      \vect{d}_N
    \end{array}
    \right]
  \begin{bmatrix}
    \\
    \vect{q}\\
    \\
  \end{bmatrix}
  =
  \begin{bmatrix}
    sim(\vect{d}_1,\vect{q}) \\
    sim(\vect{d}_2,\vect{q}) \\
    \vdots \\
    sim(\vect{d}_N,\vect{q})
  \end{bmatrix}
  \]
\end{frame}

\begin{frame}
  \frametitle{Vector-matrix multiplication}

  Let's consider another direction.
  
  What is
  {\small
  \[
  \begin{bmatrix}
    x_1 & x_2 & x_3
  \end{bmatrix}
  \begin{bmatrix}
    a_{11} & a_{12} & a_{13} & a_{14} \\
    a_{21} & a_{22} & a_{23} & a_{24} \\
    a_{31} & a_{32} & a_{33} & a_{34}
  \end{bmatrix}?
  \]
  }
  \pause
  As a linear combination 
  \vspace{1in}

  \pause
  As dot products
  \vspace{1in}
\end{frame}

\begin{frame}
  \frametitle{Matrix-matrix multiplication}

  Consider
  \[
  \begin{bmatrix}
    x_{11} & x_{12} & x_{13} \\
    x_{21} & x_{22} & x_{23}
  \end{bmatrix}
  \begin{bmatrix}
    a_{11} & a_{12} & a_{13} & a_{14} \\
    a_{21} & a_{22} & a_{23} & a_{24} \\
    a_{31} & a_{32} & a_{33} & a_{34}
  \end{bmatrix}.
  \]
\end{frame}

\begin{frame}
  \frametitle{Matrix-matrix multiplication (based on matrix-vector
    multiplication)}

  \[
  \begin{bmatrix}
    x_{11} & x_{12} & x_{13} \\
    x_{21} & x_{22} & x_{23}
  \end{bmatrix}
  \left[
  \begin{array}{c|c|c|c}
    a_{11} & a_{12} & a_{13} & a_{14} \\
    a_{21} & a_{22} & a_{23} & a_{24} \\
    a_{31} & a_{32} & a_{33} & a_{34}
  \end{array}
  \right].
  \]

  \vspace{2in}
\end{frame}

\begin{frame}
  \frametitle{Matrix-matrix multiplication (based on vector-matrix
    multiplication)}

  \[
  \left[
  \begin{array}{ccc}
    x_{11} & x_{12} & x_{13} \\
    \hline
    x_{21} & x_{22} & x_{23}
  \end{array}
  \right]
  \left[
  \begin{array}{cccc}
    a_{11} & a_{12} & a_{13} & a_{14} \\
    a_{21} & a_{22} & a_{23} & a_{24} \\
    a_{31} & a_{32} & a_{33} & a_{34}
  \end{array}
  \right].
  \]

  \vspace{2in}
\end{frame}

\begin{frame}
  \frametitle{Matrix transpose}

  If $A$ is an $m\times n$ matrix
  {\footnotesize
  \[
  \left[
  \begin{array}{ccccc}
    a_{11} & a_{12} & a_{13} & \cdots & a_{1n} \\
    a_{21} & a_{22} & a_{23} & \cdots & a_{2n} \\
    \vdots & \vdots & \vdots & & \vdots \\
    a_{m1} & a_{m2} & a_{m3} & \cdots & a_{mn}
  \end{array}
  \right],
  \]
  }
  the \textcolor{red}{\bf transpose} of $A$, denoted by $A^{T}$ is
  an $n\times m$ matrix {\footnotesize
  \[
  \left[
  \begin{array}{cccc}
    a_{11} & a_{21} & \cdots & a_{m1} \\
    a_{12} & a_{22} & \cdots & a_{m2} \\
    a_{13} & a_{23} & \cdots & a_{m3} \\
    \vdots & \vdots & & \vdots \\
    a_{1n} & a_{m2} & \cdots & a_{mn}
  \end{array}
  \right].
  \]
  }
  \vspace{0.1in}

  \pause

  {\small
    Remark: We usually view a vector as a column vector.
    Therefore, a dot product between $m$-vectors can be viewed also as
    a matrix multiplication:
  \[
  \uv\cdot\vv = \uv^T\vv
  \]
  }
\end{frame}
