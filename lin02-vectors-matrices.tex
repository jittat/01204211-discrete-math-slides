\newcommand{\lecturetitle}[1]{
  \title{01204211 Discrete Mathematics \\ #1}
  \author{Jittat Fakcharoenphol}
  \frame{\titlepage}
}
\newcommand{\Mod}{\,\bmod\,}

\lecturetitle{Lecture 8b: Vectors and Matrices} 

\begin{frame}\frametitle{What is a vector?}
  You can think of a {\bf vector} as an ``ordered'' list of elements (which are
  typically numbers).  For example:
  \begin{itemize}
  \item $[1,2,5,20]$
  \item $[0,0,1,1,0,0,0,1]$
  \end{itemize}

  \pause

  You can also view a vector as a {\bf function}, e.g., you can view
  $\uv = [1,2,5,20]$ as a function $\uv$ that maps
  \[
  0 \mapsto 1, \ \
  1 \mapsto 2, \ \
  2 \mapsto 5, \ \
  3 \mapsto 20.
  \]

  \pause
  
  Each element in the vector is typically a real number ($\rf$), but
  can be an element from other sets with appropriate property (more on
  this later).

  \pause

  { \tiny {\bf Remark:} Mathematically, a vector is an element of a
    vector space.  We will understand this more later.}
\end{frame}

\begin{frame}
  \frametitle{What can be represented as a vector?}
\end{frame}

\begin{frame}
  \frametitle{Applications in machine learning}
\end{frame}

\begin{frame}
  \frametitle{Viewing vectors: vectors in $\rf^2$}
\end{frame}

\begin{frame}
  \frametitle{Viewing vectors: vectors in $\rf^3$}
\end{frame}

\begin{frame}
  \frametitle{$n$-vectors over $\rf$}

  \begin{itemize}
  \item We mostly deal with vectors with finite number of elements.
  \item This is a \textcolor{red}{\bf $4$-vector}: $[10,20,500,4]$.
    \pause
  \item We sometimes also write it as a column vector:
    \[
    \begin{bmatrix}
      10 \\ 20 \\ 500 \\ 4
    \end{bmatrix}
    \]
    \pause
  \item When every element of a vector is from some set, we say that
    it is a vector {\bf over} that set.  For example, $[10,20,500,4]$
    is a $4$-vector over ${\mathbb R}$.
  \end{itemize}
\end{frame}

\begin{frame}
  \frametitle{Vector operations}

  \begin{itemize}
  \item As discussed in the previous slides, when working with a
    system of linear equations, we mostly deals with {\bf linear
      combinations} of vectors.
  \item We will look at the operations we do to vectors to obtain
    their linear combinations.
    \pause
  \item The operations are:
    \begin{itemize}
    \item Vector additions
    \item Scalar multiplications
    \end{itemize}
  \item These operations motivate the definition of vector spaces.
  \end{itemize}
\end{frame}

\begin{frame}
  \frametitle{Vector additions}

  Given two $n$-vectors
  \[
  \uv = [u_1,u_2,\ldots,u_n]
  \]
  and
  \[
  \vv = [v_1,v_2,\ldots,v_n],
  \]
  we have that
  \[
  \uv+\vv = [u_1+v_1,u_2+v_2,\ldots,u_n+v_n].
  \]
\end{frame}

\begin{frame}
  \frametitle{Vector additions, in picture}
\end{frame}

\begin{frame}
  \frametitle{Zero vectors}
  A zero $n$-vector ${\bm 0}=[0,0,\ldots,0]$ is an additive identity, i.e.,
  for any vector $\uv$,
  \[
  {\bm 0} + \uv = \uv + {\bm 0} = \uv. 
  \]
\end{frame}

\begin{frame}
  \frametitle{Scalar multiplications}

  For a vector over $\rf$, we refer to an element $\alpha$ in
  $\rf$ as a scalar.  For an $n$-vector
  \[
  \uv = [u_1,u_2,\ldots,u_n],
  \]
  we have that
  \[
  \alpha\cdot\uv = [\alpha\cdot u_1, \alpha\cdot u_2,\ldots, \alpha\cdot u_n],
  \]
\end{frame}

\begin{frame}
  \frametitle{Scalar multiplications, in pictures}
\end{frame}

\begin{frame}
  \frametitle{Linear combinations}
  For any scalar \
  \[
  \alpha_1,\alpha_2,\ldots,\alpha_m
  \]
  and vectors
  \[
  \uv_1,\uv_2,\ldots,\uv_m,
  \]
  we say that
  \[
  \alpha_1\uv_1 + \alpha_2\uv_2 + \cdots + \alpha_m \uv_m
  \]
  is a \textcolor{red}{\bf linear combination} of $\uv_1,\ldots,\uv_m$.
  \pause
  \vspace{0.2in}

  Examples:
  \vspace{1in}
\end{frame}


\begin{frame}
  \frametitle{A linear system with 3 variables}
  Give the following linear system.

  {\footnotesize
  \[
  \begin{array}{rcrcrcl}
    2x_1 & + & 4x_2 & + & 3x_3 & = & 7 \\
    x_1 & + &  &  & 5x_3 & = & 12 \\
    4x_1 & + & 2x_2 & + & 3x_3 & = & 10
  \end{array}
  \]
  }
  \pause
  If we rewrite the system as

  {\footnotesize
  \[
  \begin{bmatrix}
    2 \\ 1 \\ 4
  \end{bmatrix}
  \cdot x_1 +
  \begin{bmatrix}
    4 \\ 0 \\ 2
  \end{bmatrix}
  \cdot x_2 +
  \begin{bmatrix}
    3 \\ 5 \\ 3
  \end{bmatrix}
  \cdot x_3 +
  =
  \begin{bmatrix}
    7 \\ 12 \\ 10
  \end{bmatrix}.
  \]
  }

  \pause
  This becomes the problem of expressing a vector as linear
  combination of other vectors.  I.e., given vectors
  \[
  \uv_1 = [2,1,4], \ \ \uv_2 = [4,0,2], \ \ \uv_3 = [3,5,3]
  \]
  we would like to find coefficients $x_1,x_2,x_3$ such that
  \[
  x_1\cdot \uv_1 + x_2\cdot \uv_2 + x_3\cdot \uv_3 = [7,12,10].
  \]
\end{frame}


\begin{frame}
  \frametitle{Span}

  A set of all linear combination of vectors $\uv_1,\uv_2,\ldots,\uv_m$ is called the \textcolor{red}{\bf span} of that set of vectors.

  It is denote by $\mathrm{Span} \{\uv_1,\uv_2,\ldots,\uv_m\}$.

  \vspace{0.3in}

  Examples:
  \vspace{2in}
\end{frame}

\begin{frame}
  \frametitle{Convex combination}
  For any scalar \
  \[
  \alpha_1,\alpha_2,\ldots,\alpha_m,
  \]
  such that $\alpha_1+\alpha_2+\ldots+\alpha_m=1$ and $\alpha_i\geq 0$ for all $i$,
  and vectors
  \[
  \uv_1,\uv_2,\ldots,\uv_m,
  \]
  we say that
  \[
  \alpha_1\uv_1 + \alpha_2\uv_2 + \cdots + \alpha_m \uv_m
  \]
  is a \textcolor{red}{\bf convex combination} of $\uv_1,\ldots,\uv_m$.
  \pause
  \vspace{0.2in}

  Examples:
  \vspace{1in}
\end{frame}


\begin{frame}
  \frametitle{What is a matrix?}

  Matrices arise in many places.  We will see that there are
  essentially two ways to look at matrices.
  
  \[
  \left[
    \begin{array}{ccc}
      1 & 2 & 3 \\
      4 & 5 & 6 \\
      7 & 8 & 9 \\
      10 & 11 & 12 \\
    \end{array}
    \right]
  \pause
  =
  \left[
    \begin{array}{c|c|c}
      1 & 2 & 3 \\
      4 & 5 & 6 \\
      7 & 8 & 9 \\
      10 & 11 & 12 \\
    \end{array}
    \right]
  \pause
  =
  \left[
    \begin{array}{ccc}
      1 & 2 & 3 \\
      \hline
      4 & 5 & 6 \\
      \hline
      7 & 8 & 9 \\
      \hline
      10 & 11 & 12 \\
    \end{array}
    \right]
  \]
\end{frame}

\begin{frame}
  \frametitle{A matrix from a system of linear equations}
  Consider the following system of linear equations:

  \[
  \begin{array}{ccccccr}
    x_1 &+& x_2 &+& x_3 &=& 5\\
    2x_1 &+& x_2 &+& 2x_3 &=& 10\\
    3x_1 &+& x_2 &+& 2x_3 &=& 4
  \end{array}
  \]
  \pause

  Again we can view it as a vector equation:
  \[
  \begin{bmatrix}
    1\\
    2\\
    3
  \end{bmatrix}
  x_1 +
  \begin{bmatrix}
    1\\
    1\\
    1
  \end{bmatrix}
  x_2 +
  \begin{bmatrix}
    1\\
    2\\
    2
  \end{bmatrix}
  x_3
  =
  \begin{bmatrix}
    5\\
    10\\
    4
  \end{bmatrix}
  \]
\end{frame}

\begin{frame}
  \frametitle{A matrix from a system of linear equations}
  {\small
  From the following system of linear equations
  \[
  \begin{array}{ccccccr}
    x_1 &+& x_2 &+& x_3 &=& 5\\
    2x_1 &+& x_2 &+& 2x_3 &=& 10\\
    3x_1 &+& x_2 &+& 2x_3 &=& 4
  \end{array}
  \]
  
  We can also view variables $x_1,x_2,x_3$ as a vector, i.e., let
  $  \vect{x} = 
  \begin{bmatrix}
    x_1\\
    x_2\\
    x_3
  \end{bmatrix}. $
  }
  
  \pause
  The coefficients form a nice rectangular ``matrix'' $A$:
  \[
  A =
  \begin{bmatrix}
    1 & 1 & 1 \\
    2 & 1 & 2 \\
    3 & 1 & 2
  \end{bmatrix},
  \]
  \pause
  and rewrite the system as
  \[
  \begin{bmatrix}
    1 & 1 & 1 \\
    2 & 1 & 2 \\
    3 & 1 & 2
  \end{bmatrix}
  \begin{bmatrix}
    x_1\\
    x_2\\
    x_3
  \end{bmatrix}
  =
  \begin{bmatrix}
    5\\
    10\\
    4
  \end{bmatrix}
  \]
\end{frame}

\begin{frame}
  \frametitle{Size}
  \[
  \begin{bmatrix}
    1 & 1 & 1 & 1\\
    2 & 1 & 2 & 5\\
    3 & 1 & 2 & 4
  \end{bmatrix}
  \]
  \pause

  The {\bf size} of a matrix is determined by the number of rows and
  columns.  A matrix with $m$ rows and $n$ columns is referred to as
  an $m$-by-$n$ matrix or an $m\times n$ matrix.  We refers to $m$ and
  $n$ as its {\bf dimensions}.
\end{frame}

\begin{frame}
  \frametitle{Matrix-Vector Multiplication}
  How would we understand the multiplication
  \[
  \begin{bmatrix}
    1 & 1 & 1 \\
    2 & 1 & 2 \\
    3 & 1 & 2
  \end{bmatrix}
  \begin{bmatrix}
    x_1\\
    x_2\\
    x_3
  \end{bmatrix}
  \]
  \pause

  {\bf By rows.}  Consider the first row of $A$:
  \pause
  \[
  \begin{bmatrix}
    1 & 1 & 1
  \end{bmatrix}
  \begin{bmatrix}
    x_1\\
    x_2\\
    x_3
  \end{bmatrix}
  \pause
  =1\cdot x_1 + 1\cdot x_2 + 1\cdot x_3.
  \]

  Let's look at another two rows:
  {\footnotesize
  \[
  \begin{bmatrix}
    2 & 1 & 2
  \end{bmatrix}
  \begin{bmatrix}
    x_1\\
    x_2\\
    x_3
  \end{bmatrix}
  \pause
  =2\cdot x_1 + 1\cdot x_2 + 2\cdot x_3,
  \ \ \ \ \ \
  \pause
  \begin{bmatrix}
    3 & 1 & 2
  \end{bmatrix}
  \begin{bmatrix}
    x_1\\
    x_2\\
    x_3
  \end{bmatrix}
  \pause
  =3\cdot x_1 + 1\cdot x_2 + 2\cdot x_3,
  \]
  }
\end{frame}

\begin{frame}
  \frametitle{Matrix-Vector Multiplication {\bf by Rows}}
  We look at matrix-vector multiplication with ``row perspective''.
  This is a common way to view matrix-vector multiplication.
  \[
  \begin{bmatrix}
    \onslide<2->{1 & 1 & 1} \\
    \onslide<3->{2 & 1 & 2} \\
    \onslide<4->{3 & 1 & 2}
  \end{bmatrix}
  \begin{bmatrix}
    x_1\\
    x_2\\
    x_3
  \end{bmatrix}
  =
  \begin{bmatrix}
    \onslide<2->{1\cdot x_1 + 1\cdot x_2 + 1\cdot x_3} \\
    \onslide<3->{2\cdot x_1 + 1\cdot x_2 + 2\cdot x_3} \\
    \onslide<4->{3\cdot x_1 + 1\cdot x_2 + 2\cdot x_3}
  \end{bmatrix}
  \]

  Recall:
  {\small
  \only<2>{
  \[
  \begin{bmatrix}
    1 & 1 & 1
  \end{bmatrix}
  \begin{bmatrix}
    x_1\\
    x_2\\
    x_3
  \end{bmatrix}
  \pause
  =1\cdot x_1 + 1\cdot x_2 + 1\cdot x_3.
  \]
  }
  \only<3>{
  \[
  \begin{bmatrix}
    2 & 1 & 2
  \end{bmatrix}
  \begin{bmatrix}
    x_1\\
    x_2\\
    x_3
  \end{bmatrix}
  \pause
  =2\cdot x_1 + 1\cdot x_2 + 2\cdot x_3,
  \]
  }
  \only<4>{
  \[
  \begin{bmatrix}
    3 & 1 & 2
  \end{bmatrix}
  \begin{bmatrix}
    x_1\\
    x_2\\
    x_3
  \end{bmatrix}
  \pause
  =3\cdot x_1 + 1\cdot x_2 + 2\cdot x_3,
  \]
  }
  }
\end{frame}


\begin{frame}
  \frametitle{Review: Dot product}

  \begin{block}{Definition}
    For $n$-vectors $\uv=[u_1,u_2,\ldots,u_n]$ and $\vv=[v_1,v_2,\ldots,v_n]$, the {\bf dot product} of $\uv$ and $\vv$, denoted by $\uv\cdot\vv$, is
    \[
    u_1\cdot v_1 + 
    u_2\cdot v_2 +
    \cdots +
    u_n\cdot v_n 
    \]
  \end{block}
\end{frame}

\begin{frame}
  \frametitle{Matrix-Vector Multiplication {\bf by Rows}}

  We look at matrix-vector multiplication with ``row perspective'',
  which can be written nicely with \textcolor{blue}{\bf dot product}.

  {\small
  I.e., from:
  \[
  \begin{bmatrix}
    1 & 1 & 1 \\
    2 & 1 & 2 \\
    3 & 1 & 2
  \end{bmatrix}
  \begin{bmatrix}
    x_1\\
    x_2\\
    x_3
  \end{bmatrix}
  =
  \begin{bmatrix}
    1\cdot x_1 + 1\cdot x_2 + 1\cdot x_3 \\
    2\cdot x_1 + 1\cdot x_2 + 2\cdot x_3 \\
    3\cdot x_1 + 1\cdot x_2 + 2\cdot x_3
  \end{bmatrix}
  \]

  we have
  \[
  \left[
    \begin{array}{c}
      \vect{r}_1 \\
      \hline
      \vect{r}_2 \\
      \hline
      \vect{r}_3 
    \end{array}
    \right]
  \vect{x}
  =
  \left[
    \begin{array}{c}
      \vect{r}_1\cdot\vect{x} \\
      \hline
      \vect{r}_2\cdot\vect{x} \\
      \hline
      \vect{r}_3\cdot\vect{x}
    \end{array}
    \right],
  \]
  where
  \[
  \vect{r}_1=
  \begin{bmatrix}
    1 & 1 & 1
  \end{bmatrix},
  \ \ \
  \vect{r}_2=
  \begin{bmatrix}
    2 & 1 & 2
  \end{bmatrix},
  \ \ \
  \vect{r_3}=
  \begin{bmatrix}
    3 & 1 & 2
  \end{bmatrix}.
  \]
  }
  \pause
  \begin{block}{Dot-product perspective}
    {\small
    The matrix-vector product is a vector of {\bf dot products}
    between each rows and the vector.}
  \end{block}
\end{frame}
  
\begin{frame}
  \frametitle{Matrix-Vector Multiplication {\bf by Columns}}

  However, another nice way to look at matrix-vector multiplication is
  {\bf by columns}.  Notice that:
  \[
  \begin{bmatrix}
    1 & 1 & 1 \\
    2 & 1 & 2 \\
    3 & 1 & 2
  \end{bmatrix}
  \begin{bmatrix}
    x_1\\
    x_2\\
    x_3
  \end{bmatrix}
  =
  \begin{bmatrix}
    1\cdot x_1 + 1\cdot x_2 + 1\cdot x_3 \\
    2\cdot x_1 + 1\cdot x_2 + 2\cdot x_3 \\
    3\cdot x_1 + 1\cdot x_2 + 2\cdot x_3
  \end{bmatrix}
  \]
  \pause
  can be written as
  \[
  \begin{bmatrix}
    1\\
    2\\
    3
  \end{bmatrix}
  x_1 +
  \begin{bmatrix}
    1\\
    1\\
    1
  \end{bmatrix}
  x_2 +
  \begin{bmatrix}
    1\\
    2\\
    2
  \end{bmatrix}
  x_3
  =
  \begin{bmatrix}
    5\\
    10\\
    4
  \end{bmatrix}
  \]

  \pause
  \begin{block}{Linear combination perspective}
    The matrix-vector product is a {\bf linear combination} of column vectors.
  \end{block}
  
\end{frame} 

\begin{frame}
  \frametitle{Two perspectives: Matrix-Vector multiplication}

  {\bf Dot products between rows and the vector}
  \[
  \begin{bmatrix}
    a_{11} & a_{12} & a_{13} \\
    a_{21} & a_{22} & a_{23} \\
    a_{31} & a_{32} & a_{33} \\
    a_{41} & a_{42} & a_{43}
  \end{bmatrix}
  \begin{bmatrix}
    x_1 \\
    x_2 \\
    x_3 
  \end{bmatrix}
  =
  \onslide<2->{
    \left[
      \begin{array}{c}
        \onslide<2->{a_{11}\cdot x_1 + a_{12}\cdot x_2 + a_{13}\cdot x_3} \\
        \onslide<3->{a_{21}\cdot x_1 + a_{22}\cdot x_2 + a_{23}\cdot x_3} \\
        \onslide<4->{a_{31}\cdot x_1 + a_{32}\cdot x_2 + a_{33}\cdot x_3} \\
        \onslide<5->{a_{41}\cdot x_1 + a_{42}\cdot x_2 + a_{43}\cdot x_3} 
      \end{array}
      \right]
  }
  \]

  {\bf Linear combination of column vectors}
  \[
  \begin{bmatrix}
    a_{11} & a_{12} & a_{13} \\
    a_{21} & a_{22} & a_{23} \\
    a_{31} & a_{32} & a_{33} \\
    a_{41} & a_{42} & a_{43}
  \end{bmatrix}
  \begin{bmatrix}
    x_1 \\
    x_2 \\
    x_3 
  \end{bmatrix}
  =
  \onslide<6->{
  \begin{bmatrix}
    a_{11} \\
    a_{21} \\
    a_{31} \\
    a_{41}
  \end{bmatrix}
  \cdot x_1
  +
  }
  \onslide<7->{
  \begin{bmatrix}
    a_{12} \\
    a_{22} \\
    a_{32} \\
    a_{42}
  \end{bmatrix}
  \cdot x_2
  +
  }
  \onslide<8->{
  \begin{bmatrix}
    a_{13} \\
    a_{23} \\
    a_{33} \\
    a_{43}
  \end{bmatrix}
  \cdot x_3
  }
  \]

  \onslide<9->{
    \begin{block}{Dimensions}
      If the matrix has $n$ columns, the vector should be an
      $n$-vector.
    \end{block}
  }
\end{frame}

\begin{frame}
  \frametitle{Document search}
  \begin{itemize}
  \item You have 1,000,000 documents in a library.  Given another
    document, you would like to find similar documents from the
    library.  How can you do that?
    \pause
  \item You need some way to measure document {\bf similarity}.
    \pause
  \item Suppose that you nave $N$ documents in the library:
    $d_1,d_2,\ldots, d_N$.  Given a query document $q$, you want to
    find document $d_i$ that maximize
    \[
    sim(d_i,q),
    \]
    where $sim(d,d')$ is the similarity score between documents $d$
    and $d'$.
  \end{itemize}
\end{frame}

\begin{frame}
  \frametitle{Document vector models}

  What is a document? \pause It's just a list of words.  \pause If you
  throw all the ordering away, a document is simply a set of words.

  \pause Let's start with an example.  Suppose that we only care about
  5 words: \textcolor{blue}{{\tt dog}, {\tt cat}, {\tt food}, {\tt
      restaurant},} and \textcolor{blue}{\tt coffee}.

  \pause Consider the following 4 (very short) documents:

  \begin{itemize}
  \item $d_1$: {\tiny People love pets.  Most famous pets are cats and
    dogs.} \\
    \onslide<6->{$d_1=\{\mbox{\tt dog}, \mbox{\tt cat}\}$}
  \item $d_2$: {\tiny Bar Mai has many restaurants with cheap foods.} \\
    \onslide<7->{$d_2=\{\mbox{\tt restaurant}, \mbox{\tt food}\}$}
  \item $d_3$: {\tiny Cat cafe used to be popular in Thailand.  People
    buy coffee and play with cats there.} \\
    \onslide<8->{$d_3=\{\mbox{\tt coffee}, \mbox{\tt cat}\}$}
  \item $d_4$: {\tiny Dogs are human's best friends.  They were around
    in civilization for a long long time.} \\
    \onslide<9->{$d_4=\{\mbox{\tt dog}\}$}
  \end{itemize}

  \onslide<10->{
  How can we translate these sets into vectors?
  }
\end{frame}

\begin{frame}
  \frametitle{Document vector models}

  We assign a fixed co-ordinate for each word, and if a set contain a particular word, we put $1$ in that co-ordinate.

  \pause Here are our 5 words: \textcolor{blue}{{\tt dog}, {\tt cat},
    {\tt food}, {\tt restaurant},} and \textcolor{blue}{\tt coffee}.

  Each document becomes:

  \begin{itemize}
  \item $d_1$: {\tiny People love pets.  Most famous pets are cats and
    dogs.} \\
    $d_1=\{\mbox{\tt dog}, \mbox{\tt cat}\}$,
    \pause $\vect{d}_1=[1,1,0,0,0]$
    \pause
  \item $d_2$: {\tiny Bar Mai has many restaurants with cheap foods.} \\
    $d_2=\{\mbox{\tt restaurant}, \mbox{\tt food}\}$,
    \pause $\vect{d}_2=[0,0,1,1,0]$
    \pause
  \item $d_3$: {\tiny Cat cafe used to be popular in Thailand.  People
    buy coffee and play with cats there.} \\
    $d_3=\{\mbox{\tt coffee}, \mbox{\tt cat}\}$,
    \pause $\vect{d}_3=[0,1,0,0,1]$
    \pause
  \item $d_4$: {\tiny Dogs are human's best friends.  They were around
    in civilization for a long long time.} \\
    $d_4=\{\mbox{\tt dog}\}$,
    \pause $\vect{d}_4=[1,0,0,0,0]$
  \end{itemize}

\end{frame}

\begin{frame}
  \frametitle{Document vector models}
  
  Words: \textcolor{blue}{{\tt dog}, {\tt cat}, {\tt food}, {\tt
      restaurant},} and \textcolor{blue}{\tt coffee}.

  Suppose that we have query document:

  $q$: {\tiny I love cats and coffee.  What restaurant should I
    visit?} \\
  \pause as a set:
  $q=\{\mbox{\tt cat}, \mbox{\tt coffee}, \mbox{\tt restaurant}\}$ \\
  \pause as a vector:
  $\vect{q}=[0,1,0,1,1]$

  \pause
  \vspace{0.2in}
  {\small
  Our documents are:
  \begin{itemize}
  \item $d_1$: {\tiny People love pets.  Most famous pets are cats and
    dogs.} \\
    $d_1=\{\mbox{\tt dog}, \mbox{\tt cat}\}$ \ \ \ \ $\vect{d}_1=[1,1,0,0,0]$
  \item $d_2$: {\tiny Bar Mai has many restaurants with cheap foods.} \\
    $d_2=\{\mbox{\tt restaurant}, \mbox{\tt food}\}$ \ \ \ \ $\vect{d}_2=[0,0,1,1,0]$
  \item $d_3$: {\tiny Cat cafe used to be popular in Thailand.  People
    buy coffee and play with cats there.} \\
    $d_3=\{\mbox{\tt coffee}, \mbox{\tt cat}\}$ \ \ \ \ $\vect{d}_3=[0,1,0,0,1]$
  \item $d_4$: {\tiny Dogs are human's best friends.  They were around
    in civilization for a long long time.} \\
    $d_4=\{\mbox{\tt dog}\}$ \ \ \ \ $\vect{d}_4=[1,0,0,0,0]$
  \end{itemize}
  }
  
  \pause
  How can we define ``similarity'' measure?
\end{frame}

\begin{frame}
  \frametitle{Dot products as a similarity measure}

  From the previous example, we see that the dot products between
  $\vect{d}_i$'s and $\vect{q}$ count the number of common words.

  \pause
  \vspace{0.1in}
  This simple idea can be extended in many ways.

  \pause
  \begin{itemize}
  \item We can increase our ``dictionary'''s size to include more
    words.
    \pause
  \item We can group similar words into the same ``co-ordinates''.
    \pause
  \item In fact, the dot product measures the ``angle'' between
    vectors.  For vectors over $\rf$, we have that
    \[
    \uv\cdot\vv = |\uv||\vv|\cos\theta,
    \]
    where $\theta$ is the angle between vectors $\uv$ and $\vv$.
  \end{itemize}
\end{frame}

\begin{frame}
  \frametitle{Computing all similarity scores}

  If we have documents $\vect{d}_1,\vect{d}_1,\ldots,\vect{d}_N$, as
  vectors, and a query $\vect{q}$, how can we compute all similarity
  scores?

  \pause
  By performing matrix-vector multiplication:
  
  \[
  \left[
    \begin{array}{c}
      \ \ \ \ \ \vect{d}_1 \ \ \ \ \ \\
      \hline
      \vect{d}_2 \\
      \hline
      \vdots \\
      \hline
      \vect{d}_N
    \end{array}
    \right]
  \begin{bmatrix}
    \\
    \vect{q}\\
    \\
  \end{bmatrix}
  =
  \begin{bmatrix}
    sim(\vect{d}_1,\vect{q}) \\
    sim(\vect{d}_2,\vect{q}) \\
    \vdots \\
    sim(\vect{d}_N,\vect{q})
  \end{bmatrix}
  \]
\end{frame}

\begin{frame}
  \frametitle{Vector-matrix multiplication}

  Let's consider another direction.
  
  What is
  {\small
  \[
  \begin{bmatrix}
    x_1 & x_2 & x_3
  \end{bmatrix}
  \begin{bmatrix}
    a_{11} & a_{12} & a_{13} & a_{14} \\
    a_{21} & a_{22} & a_{23} & a_{24} \\
    a_{31} & a_{32} & a_{33} & a_{34}
  \end{bmatrix}?
  \]
  }
  \pause
  As a linear combination 
  \vspace{1in}

  \pause
  As dot products
  \vspace{1in}
\end{frame}

\begin{frame}
  \frametitle{Matrix-matrix multiplication}

  Consider
  \[
  \begin{bmatrix}
    x_{11} & x_{12} & x_{13} \\
    x_{21} & x_{22} & x_{23}
  \end{bmatrix}
  \begin{bmatrix}
    a_{11} & a_{12} & a_{13} & a_{14} \\
    a_{21} & a_{22} & a_{23} & a_{24} \\
    a_{31} & a_{32} & a_{33} & a_{34}
  \end{bmatrix}.
  \]
\end{frame}

\begin{frame}
  \frametitle{Matrix-matrix multiplication (based on matrix-vector
    multiplication)}

  \[
  \begin{bmatrix}
    x_{11} & x_{12} & x_{13} \\
    x_{21} & x_{22} & x_{23}
  \end{bmatrix}
  \left[
  \begin{array}{c|c|c|c}
    a_{11} & a_{12} & a_{13} & a_{14} \\
    a_{21} & a_{22} & a_{23} & a_{24} \\
    a_{31} & a_{32} & a_{33} & a_{34}
  \end{array}
  \right].
  \]

  \vspace{2in}
\end{frame}

\begin{frame}
  \frametitle{Matrix-matrix multiplication (based on vector-matrix
    multiplication)}

  \[
  \left[
  \begin{array}{ccc}
    x_{11} & x_{12} & x_{13} \\
    \hline
    x_{21} & x_{22} & x_{23}
  \end{array}
  \right]
  \left[
  \begin{array}{cccc}
    a_{11} & a_{12} & a_{13} & a_{14} \\
    a_{21} & a_{22} & a_{23} & a_{24} \\
    a_{31} & a_{32} & a_{33} & a_{34}
  \end{array}
  \right].
  \]

  \vspace{2in}
\end{frame}

\begin{frame}
  \frametitle{Matrix transpose}

  If $A$ is an $m\times n$ matrix
  {\footnotesize
  \[
  \left[
  \begin{array}{ccccc}
    a_{11} & a_{12} & a_{13} & \cdots & a_{1n} \\
    a_{21} & a_{22} & a_{23} & \cdots & a_{2n} \\
    \vdots & \vdots & \vdots & & \vdots \\
    a_{m1} & a_{m2} & a_{m3} & \cdots & a_{mn}
  \end{array}
  \right],
  \]
  }
  the \textcolor{red}{\bf transpose} of $A$, denoted by $A^{T}$ is
  an $n\times m$ matrix {\footnotesize
  \[
  \left[
  \begin{array}{cccc}
    a_{11} & a_{21} & \cdots & a_{m1} \\
    a_{12} & a_{22} & \cdots & a_{m2} \\
    a_{13} & a_{23} & \cdots & a_{m3} \\
    \vdots & \vdots & & \vdots \\
    a_{1n} & a_{m2} & \cdots & a_{mn}
  \end{array}
  \right].
  \]
  }
  \vspace{0.1in}

  \pause

  {\small
    Remark: We usually view a vector as a column vector.
    Therefore, a dot product between $m$-vectors can be viewed also as
    a matrix multiplication:
  \[
  \uv\cdot\vv = \uv^T\vv
  \]
  }
\end{frame}

\begin{frame}
  \frametitle{Matrix multiplication and transpose}
  What is $(AB)^T$?
  \vspace{2.5in}
\end{frame}

\begin{frame}
  \frametitle{Key-Value database}

  Suppose you have a database of key-value pairs:
  \[
  \{(somchai,10), (somying,14), (somnuk,23), (somjai,50), (somsom,-40)\}
  \]
  
  Given a query $q$, you want to find a value $v$ such that $(q,v)$ is
  in the database.  E.g., \pause

  Let's see how we could do that (very {\bf inefficiently}) with
  matrix multiplication.

  \vspace{2in}
  
\end{frame}

\begin{frame}
  \frametitle{Vector encodings of keys and queries}

  \begin{itemize}
  \item You want to have {\bf distinct} keys: $\vect{k}_1,\vect{k}_2,\ldots,\vect{k}_n$
    \vspace{1in}
  \item You want a query $\vect{q}$ to {\bf match} with an appropriate key.
    (Maybe the key which is exactly the same.)
    \vspace{1in}
  \end{itemize}
\end{frame}

\begin{frame}
  \frametitle{Example}
  \begin{itemize}
  \item Key encoding:
    \[
    somchai = [0,1,0,0,0,0], \ \
    somying = [0,0,0,0,1,0], \ \
    somnuk = [1,0,0,0,0,0],
    \]
    \[
    somjai = [0,0,1,0,0,0], \ \ \ \ \
    somsom = [0,0,0,0,0,1]
    \]
  \item A value table (or vector):
    {\small
    $\vv = 
    \begin{bmatrix}
      10 \\ 14 \\ 23 \\ 50 \\ -40
    \end{bmatrix}
    $
    }
  \item A query $\vect{q}$ is a $5$-vector.  A query matches key
    $\vect{k}_i$ if
    \[
    \vect{k}_i^T\vect{q} = 1
    \]
    
  \end{itemize}
\end{frame}

\begin{frame}
  \frametitle{Example (cont)}
  \begin{itemize}
  \item Key matrix
    {\small
      $
      K=\begin{bmatrix}
      0 & 1 & 0 & 0 & 0 & 0\\
      0 & 0 & 0 & 0 & 1 & 0\\
      1 & 0 & 0 & 0 & 0 & 0\\
      0 & 0 & 1 & 0 & 0 & 0\\
      0 & 0 & 0 & 0 & 0 & 1
      \end{bmatrix}
      $
    }
  \item A value table (or vector):
    {\small
      $\vv = 
      \begin{bmatrix}
        10 \\ 14 \\ 23 \\ 50 \\ -40
      \end{bmatrix}
      $
    }
  \item Let's try querying with various $\vect{q}$
    \pause
    \pause
  \item The final formula is
    \[
    (K\vect{q})^T\vv \pause
    = (\vect{q}^T K^T)\vv
    \]
  \end{itemize}
\end{frame}

\begin{frame}
  \frametitle{Key-Value database (with vector values)}

  Suppose you have a database of key-value pairs, where a value is a $2$-vector:
  \[
  \{(somchai,[10,20]), (somying,[14,-2]), (somnuk,[23,3]), (somjai,[50,-10])\}
  \]
  
  Given a query $q$, can you find a $2$-vector $\vv$ such that $(q,\vv)$ is in the database?

  \vspace{2in}
  
\end{frame}

\begin{frame}
  \frametitle{Understanding self-attention formula}

  Self-attention mechanisms are key steps in transformers, work horses
  for all chatbots you have been using recently.  The formula looks
  like (from wikipedia)
  \[
  Attention(Q,K,V) = softmax\left(\frac{QK^T}{\sqrt{d_k}}\right)V
  \]
  \vspace{1.5in}
  
\end{frame}

\end{document}

%%%%%%%%%%%%%%%%%%%%%%%%%%%%%%%%%%%%%%%%%%%%%%%%%%%%%%%%%%%%%%%%%
%%%%%%%%%%%%%%%%%%%%%%%%%%%%%%%%%%%%%%%%%%%%%%%%%%%%%%%%%%%%%%%%%
%%%%%%%%%%%%%%%%%%%%%%%%%%%%%%%%%%%%%%%%%%%%%%%%%%%%%%%%%%%%%%%%%
%%%%%%%%%%%%%%%%%%%%%%%%%%%%%%%%%%%%%%%%%%%%%%%%%%%%%%%%%%%%%%%%%
%%%%%%%%%%%%%%%%%%%%%%%%%%%%%%%%%%%%%%%%%%%%%%%%%%%%%%%%%%%%%%%%%

\begin{frame}
  \frametitle{Elements in a vector}
  \begin{itemize}
  \item We see examples of vectors over $\rf$.
  \item However, elements in a vector can be from other sets with
    appropriate property.  (I.e., they should behave a real numbers.)
  \item What do we want from an element in a vector?
    \pause
    \begin{itemize}
    \item We should be able to perform addition, subtraction, multiplication, and division.
      \pause
    \item Operations should be commutative and associative.
    \item Additive and multiplicative identity should exist.
    \item Addition and multiplication should have inverses.
    \end{itemize}
    \pause

    \item We refer to a set with these properties as a {\bf field}.
  \end{itemize}
\end{frame}

\begin{frame}
  \frametitle{A field}

  A set $\ff$ with two operations $+$ and $\times$ (or $\cdot$) is a
  \textcolor{red}{\bf field} iff these operations satisfy the
  following properties:
  \begin{itemize}
    \pause
  \item (Associativity): $(a+b)+c = a+(b+c)$ and $(a\cdot b)\cdot c = a\cdot(b\cdot c)$
    \pause
  \item (Commutativity): $a+b=b+a$ and $a\cdot b=b\cdot a$
    \pause
  \item (Identities): There exist two elements $0\in\ff$ and $1\in\ff$ such that $a+0 = a$ and $a\cdot 1 = a$
    \pause
  \item (Additive inverse): For every element $a\in \ff$, there is an element $-a\in \ff$ such that $a+(-a) = 0$
    \pause
  \item (Multiplicative inverse): For every element $a\in \ff\setminus\{0\}$, there is an alement $a^{-1}$ such that $a\cdot a^{-1}=1$
    \pause
  \item (Distributive): $a\cdot(b+c)=a\cdot b + a\cdot c$
  \end{itemize}
\end{frame}

\begin{frame}
  \frametitle{Another useful field: $GF(2)$}
  $GF(2) = \{0,1\}$.  I.e., it is a ``bit'' field.

  What are $+$ and $\cdot$ in $GF(2)$?

  \pause

  \begin{itemize}
  \item We define $b_1+b_2$ to be XOR.
    \pause

    \[
    \begin{array}{c}
      0 + 0 = 0 \\ \pause
      0 + 1 = 1 + 0 = 1 \\ \pause
      1 + 1 = 0
    \end{array}
    \] \pause
  \item We define $b_1\cdot b_2$ to be standard multiplication.
    \pause
    \[
    \begin{array}{c}
      0 \cdot 0 = 0\cdot 1 = 1\cdot 0 = 0 \\ \pause
      1\cdot 1 = 1
    \end{array}
    \] 
    
  \end{itemize}

  \pause

  You can check that $GF(2)$ satisfies the axioms of fields.
\end{frame}

