\newcommand{\lecturetitle}[1]{
  \title{01204211 Discrete Mathematics \\ #1}
  \author{Jittat Fakcharoenphol}
  \frame{\titlepage}
}
\newcommand{\Mod}{\,\bmod\,}

\lecturetitle{Lecture 9b: Polynomials (1)\footnote{This section is from Berkeley CS70 lecture notes.}} 

\begin{frame}
  \frametitle{Quick exercise} 

  For any integer $a$, $a-1|a^2-1$.

  \pause

  For any integer $a$ and $n\geq 1$, $a-1|a^n-1$.

  \vspace{3in}
\end{frame}

\begin{frame}
  \frametitle{Polynomials}

  A \textcolor{red}{\bf single-variable polynomial} is a function
  $p(x)$ of the form
  \[
  p(x) = a_dx^d+a_{d-1}x^{d-1}+\cdots+a_1x+a_0.
  \]

  We call $a_i$'s {\em coefficients}.  Usually, variable $x$ and
  coefficients $a_i$'s are real numbers.  The \textcolor{red}{\bf
    degree} of a polynomial is the largest exponent of the terms with
  non-zero coefficients.
  
  \vspace{0.2in}

  {\bf Examples}
  \begin{itemize}
  \item $x^3-3x+1$
  \item $x+10$
  \item $10$
  \item $0$
  \end{itemize}
\end{frame}

\begin{frame}
  \frametitle{Folklore}
\end{frame}

\begin{frame}
  \frametitle{Applications}
  \begin{itemize}
  \item Secret sharing
    \pause
  \item Error-correcting codes
  \end{itemize}
\end{frame}

\begin{frame}
  \frametitle{Basic facts}

  \begin{block}{Definition}
    $a$ is a \textcolor{red}{\bf root} of polynomial $f(x)$ if
    $f(a)=0$.
  \end{block}
  
  \begin{block}{Properties}
    {\bf Property 1:} A non-zero polynomial of degree $d$ has at most
    $d$ roots.

    {\bf Property 2:} Given $d+1$ pairs
    $(x_1,y_1),\ldots,(x_{d+1},y_{d+1})$ with distinct $x_i$'s, there
    is a {\em unique} polynomial $p(x)$ of degree at most $d$ such
    that $p(x_i)=y_i$ for $1\leq i\leq d+1$.
  \end{block}
\end{frame}

\begin{frame}
  \begin{lemma}
    If two polynomials $f(x)$ and $g(x)$ of degree at most $d$ that
    share $d+1$ points $(x_1,y_1),\ldots,(x_{d+1},y_{d+1})$, where all
    $x_i$'s are distinct, i.e., $f(x_i)=g(x_i)=y_i$, then $f(x)=g(x)$.
  \end{lemma}
  \begin{proof}
    {\small

      Suppose that $f(x)=a_dx^d+a_{d-1}x^{d-1}+\cdots+a_0$ and
      $g(x)=b_dx^d+b_{d-1}x^{d-1}+\cdots+b_0$.
      
      Let $h(x)=f(x)-g(x)$, i.e., let
      $h(x)=c_dx^d+c_{d-1}x^{d-1}+\cdots+c_0$, where $c_i=a_i-b_i$.
      Note that $h(x)$ is also a polynomial of degree (at most) $d$.

      We claim that $h(x)$ has $d+1$ roots.  Note that since
      $f(x_i)=g(x_i)=y_i$, we have that
      \[
      h(x_i)=f(x_i)-g(x_i)=y_i-y_i=0,
      \]
      i.e., every $x_i$ is a root of $h(x)$.

      From {\bf Property 1}, if $h(x)$ is non-zero it has at most $d$
      roots; therefore, $h(x)$ must be zero, i.e., $f(x)-g(x)=0$ or
      $f(x)=g(x)$ as required.
    }
  \end{proof}
  
\end{frame}

\begin{frame}
  \frametitle{Polynomial interpolation}
\end{frame}
