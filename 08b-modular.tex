\newcommand{\lecturetitle}[1]{
  \title{01204211 Discrete Mathematics \\ #1}
  \author{Jittat Fakcharoenphol}
  \frame{\titlepage}
}
\newcommand{\Mod}{\,\bmod\,}

\lecturetitle{Lecture 8b: Modular arithmetic} 

\begin{frame}
  \frametitle{Quick check 1} 

  If $a|m$ and $b|m$, can we say that $ab|m$?  Prove this fact or
  provide a counter example.
\end{frame}

\begin{frame}
  \frametitle{Quick check 2}

  If $a|m$, $b|m$, and $a\neq b$ are both prime, can we say that
  $ab|m$?  Prove this fact or provide a counter example.
\end{frame}

\begin{frame}
  \frametitle{Prime factorization}

  One useful fact that we use over and over again is the following.

  \begin{block}{Unique Factorization (or Fundamental Theorem of Arithmetic)}
    Every integer greater than 1 can be written {\em uniquely} as a
    product of prime numbers (up to the order of factors).
  \end{block}

  \vspace{0.2in}
  Examples:
  \begin{itemize}
  \item $10 = 2\cdot 5$
  \item $13 = 13$
  \item $112 = 2\cdot 2\cdot 2\cdot 2\cdot 7=2^4\cdot 7$
  \end{itemize}
\end{frame}

\begin{frame}
  There are 3 clocks.  At this moment, all three clocks ring at the
  same time.  The first clock rings every 3 hours, the second clock
  rings every 4 hours, and the third clock rings every 10 hours.  How
  long do you have to wait until you would hear all clocks ring a the
  same time again?
  
\end{frame}

\begin{frame}
  You have a large water container and two smaller buckets. The first
  bucket carries 3 litres of water and the second bucket carries 5
  litres of water.

  Can you put exactly 1 litre of water in the water container?
\end{frame}

\begin{frame}
  You have a large water container and two smaller buckets. The first
  bucket carries 6 litres of water and the second bucket carries 15
  litres of water.

  What is the minimum volume of water you can exactly put in the water
  container?

  \pause

  \vspace{0.2in}
  In general if you have two buckets of volumes $x$ and $y$, the amount
  that you can exactly make must be in the form of
  \[
  ax + by,
  \]
  for some integers $x$ and $y$.  (Note that $x$ and $y$ may be
  negative.)

  \pause Do you see why the sum must be divisible by any common
  divisor of $x$ and $y$?
\end{frame}

\begin{frame}
  \frametitle{Useful fact}

  For any integer $x$ and $y$, consider the term
  \[
  a\cdot x + b\cdot y,
  \]
  for some integer $a$ and $b$.

  \pause
  When the term is non-zero, it must be divisible by $gcd(x,y)$, so it
  has to be at least $gcd(x,y)$.

  \vspace{0.2in}

  It turns out that you can actually attain that value, i.e.,
  there exist a pair of integer $a$ and
  $b$ such that
  \[
  a\cdot x + b\cdot y = gcd(x,y).
  \]
\end{frame}

\begin{frame}[fragile]
  \frametitle{Finding $a$ and $b$: Extended Euclid Algorithm}

  We will modify the Euclid algorithm so that it also returns $a$ and
  $b$ together with $gcd(x,y)$.
  
  \begin{tcolorbox}
  {\small
\begin{verbatim}
  Algorithm Euclid(x,y):
    if x mod y == 0:

      return y,          ,
    else:
      g, a', b' = Euclid(y, x mod y)

      a =

      b =

      return g, a, b
\end{verbatim}
  }
  \end{tcolorbox}
  
\end{frame}

\begin{frame}
  \frametitle{Notes:}

  We have $a'$ and $b'$ such that
  \[
  a'\cdot y + b'\cdot(x \bmod y) = g.
  \]

  \vspace{3in}
\end{frame}


\begin{frame}
  \frametitle{Days}

  What day is it today? \pause Thursday.

  \pause
  What day is 3 days after today? \pause Sunday.
  
  \pause
  What day is 20 days after today? \pause Wednesday.

  \pause
  What day is 10 days before today? \pause Monday.
\end{frame}

\begin{frame}
  \frametitle{Clocks}

  Suppose that it is 1 o'clock.

  \pause
  What time is the next 5 hours?  \pause 6 o'clock.
  
  \pause
  What time is the next 10 hours?  \pause 11 o'clock.
  
  \pause
  What time is the next 20 hours?  \pause 9 o'clock.
  
\end{frame}

\begin{frame}
  \frametitle{Modular arithmetic}

  As in the days of weeks and clocks examples (and also as the modulo
  in RSA algorithm in our experiment), when working under modular
  arithmetic, we start with a \textcolor{red}{\bf modulus} $m$.

  \vspace{0.1in}
  \pause
  We can then define all arithmetic operations {\bf modulo} $m$.

  \vspace{0.1in}
  \pause
  Suppose that $m=7$.  We would like to say that
  {\small
  \[
  4 + 5 = 9 \bmod m= 2.
  \]
  Or
  \[
  3\cdot 4 = \pause 12 \bmod m = \pause 5.
  \]
  Or
  \[
  2 - 6 = \pause -4 \bmod 7 = \pause 3 \bmod 7 = 3.
  \]
  }

  \pause
  Note that when you view integers under the lense of modulus $7$,
  these numbers
  \[
  \ldots,-19,-12,-5,2,9,16,23,\ldots
  \]
  are essentially {\bf the same}.
\end{frame}

\begin{frame}
  \frametitle{Properties (1)}

  $a \bmod m = b \bmod m$, if and only if $m|a - b$.

  \pause

  \begin{proof}
    {\small
      $(\Rightarrow)$
      Let $r=a \bmod m$.  We can write
      \[
      a = qm + r,
      \]
      and
      \[
      b = pm + r,
      \]
      for some integers $q$ and $p$.  Thus, we have
      \[
      a - b = qm + r - pm - r = (q-p)m.
      \]
      Therefore $m|a-b$.

      $(\Leftarrow)$ Exercise.
    }
  \end{proof}
  
\end{frame}

\begin{frame}
  \frametitle{Properties (2)}

  \begin{itemize}
  \item $(a + b) \bmod m = ((a \bmod m) + (b \bmod m)) \bmod m$
  \item $(a - b) \bmod m = ((a \bmod m) - (b \bmod m)) \bmod m$
  \item $(a \cdot b) \bmod m = ((a \bmod m) \cdot (b \bmod m)) \bmod m$
  \end{itemize}
  
\end{frame}

\begin{frame}
  \frametitle{Congruences}

  \begin{block}{Definition (congruences)}
    For an integer $m>0$,
    if integers $a$ and $b$ are such that
    \[
    a \bmod m = b \bmod m,
    \]
    we write
    \[
    a \equiv b \pmod m.
    \]
  \end{block}

  \pause

  We also have that
  \[
  a \equiv b \pmod m \ \ \ \ \Leftrightarrow \ \ \ \
  m|(a-b)
  \]
\end{frame}

\begin{frame}
  \frametitle{Congruences: properties (1)}

  \begin{itemize}
  \item (reflexivity) \\
    $a\equiv a \pmod m$.
  \item (symmetry) \\
    $a\equiv b \pmod m$ implies $b\equiv a \pmod m$.
  \item (transitiviey) \\
    $a\equiv b \pmod m$ and $b \equiv c \pmod m$ implies $a \equiv c\pmod m$.
  \end{itemize}
\end{frame}
\begin{frame}
  \frametitle{Congruences: properties (2) -- operations}

  If we have that
  \[
  a \equiv b \pmod m,
  \]
  and
  \[
  c \equiv d \pmod m,
  \]
  then
  
  \begin{itemize}
  \item $a+c \equiv b+d \pmod m$
  \item $a-c \equiv b-d \pmod m$
  \item $ac \equiv bd \pmod m$
  \end{itemize}

  \pause
  \vspace{0.2in}

  We can pretty much think of this ``congruence'' as a normal
  equation.

  \vspace{0.1in}
  \pause
  {\em What is missing here?}
  \pause

  \textcolor{red}{Division!}

\end{frame}

\begin{frame}
  Also, we wish we can do ``cancellation'', i.e., if
  \[
  xa \equiv xb \pmod m,
  \]
  then $a \equiv b \pmod m$.  {\bf BUT THIS IS NOT ALWAYS TRUE.}
  
  \pause

  Let's see the following example:

  \[
  2\cdot 1 \equiv 2\cdot 3 \pmod 4,
  \]
  but
  \[
  1 \not\equiv 3\pmod 4.
  \]
\end{frame}

\begin{frame}
  \frametitle{Multiplications as functions}

  Let's view multiplication by 2 as a function, i.e., let $f(x)=2\cdot
  x\bmod 4$.

  \vspace{1in}

  \pause
  Let's also see $g(x) = 3\cdot x \bmod 4$.

  \vspace{1in}

  \pause
  Which functions have inverses?
\end{frame}

\begin{frame}
  \frametitle{Multiplicative inverses (standard arithmetic)}

  In standard arithmetic, what is 2/5?

  \pause

  We are looking to a number $x$ such that $2 = 5x$.  How can we do that?

  \pause

  By dividing on both sides with $5$:
  \[
  2/5 = 5x/5 = x,
  \]
  \pause
  or equivalently, by multiplying with $(1/5)=5^{-1}$:
  \[
  2\cdot 5^{-1} = 5x\cdot5^{-1} = x\cdot 5\cdot 5^{-1} = x\cdot 1 = x.
  \]
  Here $5^{-1}$ is a multiplicative inverse of $5$.
\end{frame}

\begin{frame}
  \frametitle{Multiplicative inverses (modular arithmetic)}
  
  You can do the same thing in modular arithmetic.  Let the
  modulus be $m=7$.  Note that
  \[
  5\cdot 3 \equiv 15 \equiv 1 \pmod 7.
  \]
  Therefore, $5^{-1}\equiv 3 \pmod 7$.

  \pause
  \vspace{0.1in}
  To find $2/5$, we can view our goal as to find the value of $x$ such that
  \[
  2 \equiv 5x \pmod 7.
  \]
  We can multiply both sides with $5^{-1}\equiv 3$ to get
  \[
  2\cdot 5^{-1} \equiv 2\cdot 3 \equiv 6 \equiv 5^{-1}\cdot 5x \equiv x \pmod 7.
  \]
  \pause

  Let's check:
  \[
  5\cdot 6 \equiv 30 \equiv 2 \pmod 7,
  \]
  as requied.
  
\end{frame}

\begin{frame}
  \frametitle{Multiplicative inverse modulo $m$}

  \begin{block}{Definition}
    The multiplicative inverse modulo $m$ of $a$, denoted by $a^{-1}$,
    is an integer such that
    \[
    a\cdot a^{-1}\equiv 1 \pmod m.
    \]
  \end{block}
\end{frame}

\begin{frame}
  \begin{theorem}
    An integer $a$ has a multiplicative inverse modulo $m$ iff
    $gcd(a,m) = 1$.
  \end{theorem}
  \begin{proof}
    \pause
    {\footnotesize
    $(\Leftarrow)$ Recall that there exist integers $x$ and $y$ such
    that
    \[
    x\cdot a + y\cdot m = gcd(a,m) = 1.
    \]
    Thus,
    $(x\cdot a + y\cdot m)\bmod m = x\cdot a \bmod m = 1 \bmod m$,
    i.e., $x\cdot a\equiv 1 \pmod m$.  Therefore $x$ is the inverse.

    \pause
    $(\Rightarrow)$ Let $r=gcd(a,m)$.
    Suppose that $b$ is the multiplicative inverse of
    $a$ modulo $m$, i.e., we have that
    \[
    b\cdot a \equiv 1 \pmod m,
    \]
    Thus, $ba \bmod m = 1 \bmod m = 1$, i.e., there exists an integer
    $q$ such that
    \[
    ba = qm + 1,
    \]
    or $ba - qm = 1$.  However, $r$ since $r|a$ and $r|m$, $r$ also
    divides $bd-qm$ and $1$.  But it $r\not|1$ because $r>1$ and we
    have the contradiction.
    }
  \end{proof}
\end{frame}

\begin{frame}
  \frametitle{Examples: division in modular arithmetic}

  Since the requirement for an existance of $a^{-1}$ modulo $m$ is
  that $gcd(a,m)=1$, if we let $m$ be a prime number, every $a$ which
  is not a multiple of $m$ has an inverse.

  Can you solve this equation?
  \[
  4x + 9 \equiv 0 \pmod{11}.
  \]

  \pause

  We can even perform gaussian elimination ({\em which is very useful
    later}):
  \[
  \begin{array}{rcl}
    2x + y & \equiv & 3 \pmod 7 \\
    x + 3y & \equiv & 5 \pmod 7
  \end{array}
  \]
  
\end{frame}

\begin{frame}
  \frametitle{Quick recap: RSA}

  {\small
  \begin{itemize}
  \item Private key: $(e,n)$, \ \ \  Public key: $(d,n)$
  \item Encryption $E(m) = m^{e} \bmod n$,\ \ \  Decryption: $D(w) = w^{d} \bmod n$.
  \item Goal: Select $e,d,n$ such that $D(E(m)) = m^{ed}\bmod n = m$.
  \end{itemize}
  }
  
  \vspace{0.1in}
  \pause
  {\footnotesize
  \begin{itemize}
  \item Pick two primes $p$ and $q$.  Let $n=pq$.
  \item Pick $e$ (usually a small number)
  \item Pick $d$ such that $d = e^{-1} \pmod{(p-1)(q-1)}$, i.e., $ed\equiv 1 \pmod{(p-1)(q-1)}$, or
    \[
    ed = k\cdot(p-1)(q-1) + 1, 
    \]
    for some integer $k$.
  \item What is $m^{ed}\bmod n$?
  \end{itemize}
  }
  \vspace{0.5in}
\end{frame}

\begin{frame}
  \frametitle{What's next?}

  \begin{itemize}
  \item We will prove Fermat's Little Theorem and show how to
    efficiently test if a number is prime.
  \item We will also use Fermat's Little Theorem to prove the
    correctness of RSA.
  \item Modular arithmetic is also key to our usage of polynomials to
    perform secret sharing and error correcting codes, because now we
    can do Gaussian elimination using only integers.
  \end{itemize}
\end{frame}
