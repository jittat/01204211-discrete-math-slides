\newcommand{\lecturetitle}[1]{
  \title{01204211 Discrete Mathematics \\ #1}
  \author{Jittat Fakcharoenphol}
  \frame{\titlepage}
}
\newcommand{\Mod}{\,\bmod\,}

\lecturetitle{Lecture 12b: Linear functions (I)}


\begin{frame}
  \frametitle{Linear functions}

  \begin{block}{Linear functions}
    Consider vector spaces $\V$ and $\W$ over $\rf$.  A function
    $f:\V\rightarrow\W$ is \textcolor{red}{\bf linear} if
    \begin{enumerate}
    \item for all $\vect{x},\vect{y}\in\V$,
      $f(\vect{x}+\vect{y})=f(\vect{x}) + f(\vect{y})$ and
    \item for all $\alpha\in\rf$ and $\vect{x}\in\V$,
      $f(\alpha\vect{x})=\alpha f(\vect{x})$.
    \end{enumerate}
  \end{block}
\end{frame}

\begin{frame}
  \frametitle{Example 1 - MLP}
\end{frame}

\begin{frame}
  \frametitle{Example 2 - Page rank (1)}
\end{frame}

\begin{frame}
  \frametitle{Example 2 - Page rank (2)}
\end{frame}

\begin{frame}
  \frametitle{Matrix-vector multiplication}

  Given an $m\times n$ matrix $M$ over $\rf$, consider a product
  \[
  M\vect{x}.
  \]
  Note that for the multiplication to work, $\vect{x}$ must be in
  $\rf^n$ and the result vector is in $\rf^m$.  Therefore, we can define
  a function $f: \rf^n \rightarrow \rf^m$ as
  \[
  f(\vect{x}) = M\vect{x}.
  \]
  Note that $f$ is linear because:
  \[
  f(\vect{x}+\vect{y})=M(\vect{x}+\vect{y})=M\vect{x} + M\vect{y}=
  f(\vect{x})+f(\vect{y}),
  \]
  and
  \[
  f(\alpha\vect{x})=M(\alpha\vect{x})=\alpha M\vect{x} = \alpha f(\vect{x}).
  \]
\end{frame}

\begin{frame}
  \frametitle{The converse}
  \begin{lemma}
    For any linear function $f:\rf^n \rightarrow \rf^m$, there exists an
    $m\times n$ matrix $M$ such that
    \[
    f(\vect{x}) = M\vect{x}.
    \]
  \end{lemma}
\end{frame}

\begin{frame}

    \begin{proof}
      Consider any $x\in\rf^n$.
      Let $\vect{x}=[x_1,x_2,\ldots,x_n]$.  Note that
      \[
      \vect{x}=[x_1,0,\ldots,0] + [0,x_2,0,\ldots,0] + \cdots + [0,\ldots,0,x_n].
      \]
      Let $\vect{e}_1,\vect{e}_2,\ldots,\vect{e}_n\in\rf^n$ be standard
      generators, i.e., $\vect{e}_i$ be a vector with 1 at the $i$-th row
      and 0 at every other positions.  (For example
      $\vect{e}_1=[1,0,\ldots,0]$ and $\vect{e}_3=[0,0,1,0,\ldots,0]$.)

      We thus have
      \[
      \vect{x}=x_1\vect{e}_1 + x_2\vect{e}_2 + \cdots + x_n\vect{e}_n.
      \]
      Since $f$ is linear, this implies that
      \[
      f(\vect{x}) = x_1f(\vect{e}_1) + x_2f(\vect{e}_2) + \cdots + x_nf(\vect{e}_n).
      \]
    \end{proof}
\end{frame}

\begin{frame}
  \begin{proof}[Proof (cont.)]
    {\small
      Define $M$ as follows
      \[
      M=\left[
        \begin{array}{c|c|c|c}
          & & &  \\
          %& & &  \\
          f(\vect{e}_1) &
          f(\vect{e}_2) &
          \cdots &
          f(\vect{e}_n) \\
          %& & &  \\
          & & & 
        \end{array}
        \right].
      \]
      Hence,
      \begin{eqnarray*}
        M\vect{x} &=& \left[
          \begin{array}{c|c|c|c}
            & & &  \\
            %& & &  \\
            f(\vect{e}_1) &
            f(\vect{e}_2) &
            \cdots &
            f(\vect{e}_n) \\
            %& & &  \\
            & & & 
          \end{array}
          \right]
        \begin{bmatrix}
          x_1 \\
          x_2 \\
          \vdots \\
          x_n
        \end{bmatrix} \\
        &=&
        x_1f(\vect{e}_1) + x_2f(\vect{e}_2) + \cdots + x_nf(\vect{e}_n)
        = f(\vect{x}),
      \end{eqnarray*}
      as required.
    }
  \end{proof}
\end{frame}

\begin{frame}
  \frametitle{Structures of linear functions (overview)}
\end{frame}

