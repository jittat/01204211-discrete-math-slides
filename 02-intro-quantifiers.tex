\newcommand{\lecturetitle}[1]{
  \title{01204211 Discrete Mathematics \\ #1}
  \author{Jittat Fakcharoenphol}
  \frame{\titlepage}
}
\newcommand{\Mod}{\,\bmod\,}

\lecturetitle{Lecture 2: Quantifiers and proofs}

\begin{frame}\frametitle{This lecture covers:}
  \begin{itemize}
  \item More on quantifiers
  \item How to prove a proposition
  \item Basic proof techniques
  \end{itemize}
\end{frame}

\begin{frame}\frametitle{Review: Quantifiers}
  \begin{itemize}
  \item A {\em predicate} is a statement with variables, which can be
    either true or false, after all its variables are specified.
  \item If we quantify a predicate completely, the quantified
    expression now has a truth value, and it is called a quantified
    proposition.
  \item Two ways to quantify:
    \begin{itemize}
    \item {\bf Using universal quantifiers ($\forall$).} This
      quantifier states that the quantified proposition is true when
      the predicate is true for every value of the variable in the
      specified set.
    \item {\bf Using existential quantifiers ($\exists$).} This
      quantifier states that the quantified proposition is true when
      the predicate is true for at least one value of the variable in
      the specified set.
    \end{itemize}

  \item Quantifiers can be nested.  E.g., 
    \begin{itemize}
    \item $\forall x\forall y P(x,y)\equiv \forall x(\forall y (P(x,y)))$
    \item $\forall x\exists y P(x,y)\equiv \exists x(\forall y (P(x,y)))$
    \end{itemize}
  \end{itemize}
\end{frame}

\begin{frame}\frametitle{Quick check 1}
\end{frame}

\begin{frame}\frametitle{Negations (1)}
  Let consider a set of positive integers $\mathbb Z^+$ as our
  universe.  Let predicate $P(x)\equiv$ ``$x$ is a prime number.''

  Consider this proposition

  \[(\forall x\in {\mathbb Z^+}) P(x).\]

  How can we show that this is false? \pause

  When showing that a universally quantified proposition is false, we
  need to show ``one'' counter example.  In this case, since $P(4)$ is
  false, $\forall x P(x)$ is false.  \pause

  This way of disproving a statement is equivalent to showing that

  \[(\exists x)(\neg P(x)).\]
\end{frame}

\begin{frame}\frametitle{Negations of quantified propositions}
  Let consider a set of positive integers $\mathbb Z^+$ as our
  universe.  Let predicate $Q(x)\equiv$ ``if $x > 2$, then  $x^2\leq 2x$.''

  Consider this proposition

  \[(\exists x\in {\mathbb Z^+}) Q(x).\]

  How can we show that this is false? \pause

  When showing that an existential quantified proposition is false, we
  need to show that $Q(x)$ is false for every possible values of $x$.
  In this case, since $x^2 = x\cdot x > 2\cdot x$ for every $x>2$, we
  have that $(\exists x) Q(x)$ is false. \pause

  This way of disproving a statement is equivalent to showing that

  \[(\forall x)(\neg Q(x)).\]
\end{frame}

\begin{frame}\frametitle{Negations (3)}
  Thus, the following equivalences:

  \begin{itemize}
  \item $\neg(\forall x P(x)) \equiv \exists x (\neg P(x))$
  \item $\neg(\exists x P(x)) \equiv \forall x (\neg P(x))$
  \end{itemize}
\end{frame}

\begin{frame}\frametitle{Quick check 2}
\end{frame}

\begin{frame}\frametitle{How to prove a mathematical statement}
  Given propositions $P$ and $Q$, these are a very useful logical
  equivalences (referred to as the De Morgan's Laws).

  \begin{itemize}
  \item $\neg (P\vee Q)\equiv \neg P \wedge \neg Q$
  \item $\neg (P\wedge Q)\equiv \neg P \vee \neg Q$
  \end{itemize}

  (Note that $\neg$ takes precedence over $\vee$ or $\wedge$.)

  \vspace{0.2in}
  
  How can we prove that the first statement is true?
\end{frame}

\begin{frame}\frametitle{Proof by exhaustion}
  \begin{tcolorbox}
    For any proposition $P$ and $Q$, $\neg (P\vee Q)\equiv \neg P
    \wedge \neg Q$.
  \end{tcolorbox}
  \begin{proof}
    We will prove by exhaustion.  There are 4 cases as in the truth
    table below.

    \vspace{0.1in}
    
    \begin{tabular}{|c|c||c|c|c|}
      \hline
      $P$ & $Q$ & $P\vee Q$ & $\neg(P\vee Q)$ & $\neg Q \wedge \neg P$ \\
      \hline
      $T$ & $T$ & $T$ & $F$ & $F$ \\
      $T$ & $F$ & $T$ & $F$ & $F$ \\
      $F$ & $T$ & $T$ & $F$ & $F$ \\
      $F$ & $F$ & $F$ & $T$ & $T$ \\
      \hline
    \end{tabular}

    \vspace{0.1in}

    Note that for all possible truth values of $P$ and $Q$, $\neg
    (P\vee Q)$ equals $\neg P \wedge \neg Q$.  Thus, the statement is
    true.
  \end{proof}
\end{frame}
