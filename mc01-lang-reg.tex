\newcommand{\lecturetitle}[1]{
  \title{01204211 Discrete Mathematics \\ #1}
  \author{Jittat Fakcharoenphol}
  \frame{\titlepage}
}
\newcommand{\Mod}{\,\bmod\,}


\newcommand\sbullet[1][.5]{\mathbin{\vcenter{\hbox{\scalebox{#1}{$\bullet$}}}}}

\lecturetitle{Lecture 7a: Languages and regular expressions\footnote{Based on lecture notes of {\em Models of Computation} course by Jeff Erickson.}} 
\renewcommand{\epsilon}{\varepsilon}

\newcommand{\czero}{{\mathtt 0}}
\newcommand{\cone}{{\mathtt 1}}

\begin{frame}
  \frametitle{What is computation?}
\end{frame}

\begin{frame}
  \frametitle{Models of computations}
\end{frame}

\begin{frame}
  \frametitle{Languages = specifications}
\end{frame}

\begin{frame}
  \frametitle{Formal definition: strings}

  Intuitively, a string is a {\em finite} sequence of symbols.
  However, to be able to formally prove properties of strings we need
  a precise definition.

  \vspace{0.1in}
  
  Let a finite set $\Sigma$ be the {\color{red}\bf alphabet}. (E.g.,
  for bit strings, $\Sigma=\{\czero,\cone\}$; for digits,
  $\Sigma=\{0,1,\ldots,9\}$; for English string
  $\Sigma=\{a,b,\ldots,z\}$.)
  
  The following is a recursive definition of strings.

  \begin{block}{Recursive definition of strings}
    A {\color{red} \bf string} $w$ over alphabet $\Sigma$ is either
    \begin{itemize}
    \item the empty string $\epsilon$, or
    \item $a\cdot x$ where $a\in\Sigma$ and $x$ is a string.
    \end{itemize}
  \end{block}

  \vspace{0.1in}

  The set of all strings over alphabet $\Sigma$ is denoted by $\Sigma^*$.
\end{frame}

\begin{frame}
  \frametitle{Review: more recursive definitions}

  \begin{block}{Lengths}
    For a string $w$, let $|w|$ be the length of $w$ defined as
    \[
    |w| = \left\{
    \begin{array}{ll}
      0 & \mbox{when $w=\epsilon$} \\
      1 + |x| & \mbox{when $w=a\cdot x$}
    \end{array}
    \right.
    \]
  \end{block}

  \begin{block}{Concatenation}
    For strings $w$ and $z$, the concatenation $w\sbullet z$ is defiend
    recursively as
    \[
    w\sbullet z = \left\{
    \begin{array}{ll}
      z                   & \mbox{when $w=\epsilon$} \\
      a\cdot(x\sbullet z)  & \mbox{when $w=a\cdot x$}
    \end{array}
    \right.
    \]
  \end{block} 
\end{frame}

\begin{frame}
  \frametitle{Review: proving facts about strings}

  \begin{lemma}
    For strings $w$ and $x$, $|w\sbullet x|=|w|+|x|$.
  \end{lemma}

  \begin{proof}
    \vspace{2in}
  \end{proof}
\end{frame}

\begin{frame}
  \frametitle{Formal languages}

  A {\color{red}\bf formal language} is a set of strings over some
  finite alphabet $\Sigma$.

  Examples:
  \vspace{2.5in}
\end{frame}

\begin{frame}
  \frametitle{Careful...}

  These are different languages: $\emptyset,\{\epsilon\}$

  And $\epsilon$ is not a language.

\end{frame}

\begin{frame}
  \frametitle{How to describe languages?}
\end{frame}

\begin{frame}
  \frametitle{Composition}
\end{frame}

\begin{frame}
  \frametitle{Combining languages}

  If $A$ and $B$ are languages over alphabet $\Sigma$.
  
  \begin{itemize}
  \item Basic set operations: $A\cup B$, $A\cap B$, $\bar{A}=\Sigma^*\setminus A$.
  \item Concatenation: $A\sbullet B$.
    \vspace{1in}
  \item Kleene closure or Kleene star: $A^*$.

    \vspace{1in}
    
    Also $A^+=A\sbullet A^*$
  \end{itemize}
\end{frame}

\begin{frame}
  \frametitle{Examples}
\end{frame}

\begin{frame}
  \frametitle{Regular languages}
  \begin{block}{Definition: regular languages}
    A language $L$ is {\color{red}\bf regular} if and only if it
    satisfies one of the following conditions:
    \begin{itemize}
    \item $L$ is empty;
    \item $L$ contains one string (can be the empty string $\epsilon$);
    \item $L$ is a union of two regular languages;
    \item $L$ is the concatenation of two regular languages; or
    \item $L$ is the Kleene closure of a regular language.
    \end{itemize}
  \end{block}
\end{frame}

\begin{frame}
  \frametitle{Examples}
\end{frame}

\begin{frame}
  \frametitle{Regular expressions}
  Let $\Sigma=\{\czero,\cone\}$.  Consider
  \[
  (
  (\{\czero\cone\}\cup
  (\{\cone\}\sbullet(\{\czero\}\cup\{\cone\czero\})))
  \cup
  (\{\czero\czero\}\sbullet(\{\cone\})^*)
  )
  \sbullet((\{\czero\}\sbullet\{\czero\})\sbullet\{\cone\})
  \]
  \vspace{2in}
\end{frame}

\begin{frame}
  \frametitle{Regular expressions}
  Regular language
  \[
  (
  (\{\czero\cone\}\cup
  (\{\cone\}\sbullet(\{\czero\}\cup\{\cone\czero\})))
  \cup
  (\{\czero\czero\}\sbullet(\{\cone\})^*)
  )
  \sbullet((\{\czero\}\sbullet\{\czero\})\sbullet\{\cone\})
  \]
  is represented as
  \[
  (
  \czero\cone+
  \cone(\czero+\cone\czero)
  +\czero\czero(\cone)^*
  )
  \czero\czero\cone
  \]

  Regular expressions
  \begin{itemize}
  \item omit braces around one-string sets
  \item use $+$ instead of $\cup$
  \item omit $\sbullet$
  \item follow the precedence: Kleene star operator $*$, $\sbullet$ (implicitly), and $+$.
  \end{itemize}

  {\em
    Remark: $+$ and $\sbullet$ are associative, i.e., $(A+B)+C=A+(B+C)$ and $(A\sbullet B)\sbullet C=A\sbullet(B\sbullet C)$.
  }
\end{frame}

\begin{frame}
  \frametitle{Regular expressions: examples 1}
\end{frame}

\begin{frame}
  \frametitle{Regular expressions: examples 2}
  All strings over $\{\czero,\cone\}$ except $\czero\cone\czero$.
  \vspace{3in}
\end{frame}

\begin{frame}
  \frametitle{Subexpressions}
\end{frame}

\begin{frame}
  \frametitle{Regex is everywhere}
\end{frame}

\begin{frame}
  \frametitle{Proofs about regular expressions - structural induction} 
\end{frame}

\begin{frame}
  \begin{lemma}
    Every regular expression that does not use the symbol $\emptyset$
    represents a non-empty language.
  \end{lemma}

  {\bf Proof.}
  
  \pause

  Let $R$ be a regular expression that does not use the symbol
  $\emptyset$.  We prove by (structural) induction that $R$ represents
  a non-empty language.

  \pause

  {\bf Induction hypothesis:} Every subexpression of $R$ that does not
  use the symbol $\emptyset$ represents a non-empty language.
  
  \pause

  {\em Case 1:} $R=\emptyset$.
  \vspace{0.5in}

  \pause

  {\em Case 2:} $R$ is a single string.

  \pause
  \vspace{1.5in}
  
\end{frame}

\begin{frame}{}
  
  {\bf Proof.} (cont.2/4)
  
  {\em Case 3:} $R=S+T$ for some regular expressions $S$ and $T$.

  \vspace{2.5in}
\end{frame}

\begin{frame}{}
  
  {\bf Proof.} (cont.3/4)

  {\em Case 4:} $R=S\sbullet T$ for some regular expressions $S$ and $T$.

  \vspace{2.5in}
\end{frame}

\begin{frame}{}
  
  {\bf Proof.} (cont.4/4)

  {\em Case 5:} $R=S^*$ for some regular expression $S$.

  \vspace{2.5in}
  \pause

  In every case, the language $L(R)$ is non-empty.
\end{frame}


\begin{frame}
  \begin{lemma}
    Every non-empty regular language is represented by a regular
    expression that does not use the symbol $\emptyset$.
  \end{lemma}

  \pause

  Let $R$ be a regular expression. \pause We prove that if
  $L(R)\neq\emptyset$, then there exists a regular expression $R'$
  such that $L(R)=L(R')$ and $R'$ does not contain $\emptyset$.

  \pause We prove by induction.  What should the induction hypothesis
  be?

  \vspace{1.5in}
\end{frame}

\begin{frame}

  {\bf I.H.:} For every subexpression $S$ of $R$, if $L(S)\neq
  \emptyset$, there exists an $\emptyset$-free regular expression $S'$
  such that $L(S)=L(S')$.

  \vspace{0.1in}
  
  \pause

  \begin{block}{What are the cases that we have to consider?}

  \pause

  \begin{itemize}
  \item $R=\emptyset$
  \item $R$ is a single string.
  \item $R = S+T$ for some regular expressions $S$ and $T$.
  \item $R = S\sbullet T$ for some regular expressions $S$ and $T$.
  \item $R = S^*$ for some regular expression $S$.
  \end{itemize}
  \end{block}
  
\end{frame}

\begin{frame}

(E-ex1-6) For string $w$, the reversal $w^R$ is defined recursively as follows:
\[
w^R=\left\{
\begin{array}{ll}
  \varepsilon & \mbox{if $w=\varepsilon$} \\
  x^R\sbullet a & \mbox{if $w=ax$ for some symbol $a$ and some string $x$}
\end{array}
\right.
\]
For a language $L$, the reversal of $L$ is defined as
\[
L^R = \{ w^R \;|\; w\in L\}.
\]

You may assume the following facts.
\begin{itemize}
\item $L^*\sbullet L^*=L^*$ for every language $L$.
\item $(w^R)^R=w$ for every string $w$.
\item $(x\sbullet y)^R=y^R\sbullet x^R$ for all strings $x$ and $y$.
\end{itemize}

\end{frame}

\begin{frame}

Prove that $(L^R)^*\subseteq (L^*)^R$.

\vspace{3in}
\end{frame}
