\newcommand{\lecturetitle}[1]{
  \title{01204211 Discrete Mathematics \\ #1}
  \author{Jittat Fakcharoenphol}
  \frame{\titlepage}
}
\newcommand{\Mod}{\,\bmod\,}

\lecturetitle{Lecture 10b: Dimensions} 

\begin{frame}
  \frametitle{Review: Linear combinations}

  \begin{block}{Definition}
    For any scalars
    \[
    \alpha_1,\alpha_2,\ldots,\alpha_m
    \]
    and vectors
    \[
    \uv_1,\uv_2,\ldots,\uv_m,
    \]
    we say that
    \[
    \alpha_1\uv_1 + \alpha_2\uv_2 + \cdots + \alpha_m \uv_m
    \]
    is a \textcolor{red}{\bf linear combination} of $\uv_1,\ldots,\uv_m$.
  \end{block}
\end{frame}

\begin{frame}
  \frametitle{Review: Span}

  \begin{block}{Definition}
    A set of all linear combination of vectors $\uv_1,\uv_2,\ldots,\uv_m$ is called the \textcolor{red}{\bf span} of that set of vectors.

    It is denoted by $\mathrm{Span} \{\uv_1,\uv_2,\ldots,\uv_m\}$.
  \end{block}
\end{frame}

\begin{frame}
  \frametitle{Review: Vector spaces}
  \begin{block}{Definition}
    A set $\V$ of vectors over $\ff$ is a \textcolor{red}{\bf vector space} iff
    \begin{itemize}
    \item \textcolor{blue}{(V1)} $\vect{0}\in\V$,
    \item \textcolor{blue}{(V2)} for any $\uv\in\V$,
      \[
      \alpha\cdot\uv\in\V
      \]
      for any
      $\alpha\in\ff$, and
    \item \textcolor{blue}{(V3)} for any $\uv,\vv\in\V$,
      \[
      \uv+\vv\in\V.
      \]
    \end{itemize}
  \end{block}
\end{frame}

\begin{frame}
  \frametitle{Review: Linearly independence}

  \begin{block}{Definition}
    Vectors $\uv_1,\uv_2,\ldots,\uv_n$ are \textcolor{red}{\bf
      linearly independent} if no vector $\uv_i$ can be written as a
    linear combination of other vectors.
  \end{block}

  \begin{block}{(Another) Definition}
    Vectors $\uv_1,\uv_2,\ldots,\uv_n$ are \textcolor{red}{\bf
      linearly independent} if the only solution of equation
    \[
    \alpha_1\uv_1 + \alpha_2\uv_2 +\cdots+\alpha_n\uv_n = \vect{0}
    \]
    is
    \[
    \alpha_1=\alpha_2=\cdots=\alpha_n=0.
    \]
  \end{block}

\end{frame}
\begin{frame}
  \frametitle{Review: Bases}

  \begin{block}{Definition}
    A set of vectors $\{\uv_1,\uv_2,\ldots,\uv_k\}$ is a \textcolor{red}{\bf basis} for vector space $\V$ if
    \begin{itemize}
    \item $\mathrm{Span}~\{\uv_1,\uv_2,\ldots,\uv_k\} = \V$, and
    \item $\uv_1,\uv_2,\ldots,\uv_k$ are linearly independent.
    \end{itemize}
  \end{block}
\end{frame}

\begin{frame}
  \begin{lemma}[Superfluous Vector Lemma]
    Consider vectors $\uv_1,\uv_2,\ldots,\uv_n$.
    If $\vv\in \mathrm{Span}\ \{\uv_1,\uv_2,\ldots,\uv_n\}$, then
    \[
    \mathrm{Span}\ \{\uv_1,\uv_2,\ldots,\uv_n,\vv\}
    =
    \mathrm{Span}\ \{\uv_1,\uv_2,\ldots,\uv_n\}
    \]
    \label{lemma:sup-vect}
  \end{lemma}
  \pause
  \begin{lemma}
    Consider vectors $\uv_1,\uv_2,\ldots,\uv_n$.
    If $\uv_n\in \mathrm{Span}\ \{\uv_1,\uv_2,\ldots,\uv_{n-1}\}$, then
    \[
    \mathrm{Span}\ \{\uv_1,\uv_2,\ldots,\uv_{n-1}\}
    =
    \mathrm{Span}\ \{\uv_1,\uv_2,\ldots,\uv_n\}
    \]
  \end{lemma}
\end{frame}

\begin{frame}
  \begin{lemma}[Unique representation]
    Let $\uv_1,\uv_2,\ldots,\uv_k$ be a basis for vector space $\V$.
    For any $\vv\in\V$, there is a unique way to write $\vv$ as a
    linear combination of $\uv_1,\ldots,\uv_k$.
  \end{lemma}
\end{frame}

\begin{frame}
  \frametitle{Examples in $\rf^2$ and $\rf^3$}
\end{frame}

\begin{frame}
  \frametitle{Examples in $GF(2)$ - Vector spaces from graphs}
\end{frame}

\begin{frame}
  \frametitle{Examples in $GF(2)$ - Cycles}
\end{frame}

\begin{frame}
  \frametitle{Examples in $GF(2)$ - Basis}
\end{frame}

\begin{frame}
  \frametitle{Number of vectors in bases}
  \begin{itemize}
  \item We have an observation that for a vector space $\V$, every basis has the same size.
    \pause
  \item This is not a coincident.
  \item In this course, we will see two proofs.
    \pause
  \item Remark: there are vector spaces whose basis has infinite size,
    but we are not dealing with those vector spaces in this course.
  \end{itemize}
\end{frame}

\begin{frame}
  \begin{theorem}[Main Theorem]
    If $\uv_1,\uv_2,\ldots,\uv_n$ and $\vv_1,\vv_2,\ldots,\vv_m$ are bases for vector space $\W$, then $n=m$.
  \end{theorem}
\end{frame}

\begin{frame}
  \frametitle{Exchange Lemma}
  We will prove the main theorem using the ``exchange'' lemma.

  \pause

  \begin{lemma}[Simplified Exchange Lemma]
    Consider a set of vectors $S$ and let $\vect{z}$ be a non-zero
    vector in $\mathrm{Span}~S$.  There is a vector $\vect{w}\in S$
    such that
    $\mathrm{Span}~(S\cup\{\vect{z}\}-\{\vect{w}\})=\mathrm{Span}~S$.
  \end{lemma}

  \pause
  
  \begin{lemma}[Exchange Lemma]
    Consider a set of vectors $S$ and its subset $A$.  Let $\vect{z}$
    be a non-zero vector in $\mathrm{Span}~S$ such that
    $A\cup\{\vect{z}\}$ is linearly independent.  There is a vector
    $\vect{w}\in S-A$ such that
    $\mathrm{Span}~(S\cup\{\vect{z}\}-\{\vect{w}\})=\mathrm{Span}~S$.
  \end{lemma}

  \vspace{1in}
\end{frame}

\begin{frame}
  \begin{lemma}[Morphing Lemma]
    If a set of vectors $S$ spans a vector space $\W$ and $B$ is a
    linearly independent set of vectors in $\W$, then $|B|\leq |S|$.
  \end{lemma}
  \pause
  \begin{proof}
    {\footnotesize Let $B=\{\uv_1,\uv_2,\ldots,\uv_n\}$.  We show how to
      construct $S_1,\ldots,S_n$ such that for each $i$, $|S_i|=|S|$,
      $\mathrm{Span}~S_i=\mathrm{Span}~S$, and
      \[
      \{\uv_1,\ldots,\uv_i\}\subseteq S_i.
      \]
      \pause
      Let $S_0=S$.  We construct $S_i$ from $S_{i-1}$.  \pause Note
      that since $B$ is linearly independent,
      $\{\uv_1,\ldots,\uv_{i-1}\}\subseteq S_{i-1}$ is also linearly
      independent. \pause  We can use the Exchange Lemma to state that there
      exist $\vect{w}\in S_{i-1} - \{\uv_1,\ldots,\uv_{i-1}\}$ such
      that
      \[
      \mathrm{Span}~(S_{i-1}\cup\{\vect{\uv_i}\}-\{\vect{w}\})=\mathrm{Span}~S_{i-1}.
      \]
      We then let $S_i=S_{i-1}\cup\{\vect{\uv_i}\}-\{\vect{w}\}$.
      \textcolor{gray}{(You can check that $S_i$ has the properties
        as claimed above.)}

      \pause
      Since $|S_n|=|S|$ and $B\subseteq S_n$, we have that $|B|\leq |S|$.
    }
  \end{proof}
\end{frame}

\begin{frame}
  \frametitle{Morphing Lemma $\Rightarrow$ Main Theorem}
  \begin{theorem}[Main Theorem]
    If $\uv_1,\uv_2,\ldots,\uv_n$ and $\vv_1,\vv_2,\ldots,\vv_m$ are bases for vector space $\W$, then $n=m$.
  \end{theorem}

  \pause
  \begin{proof}
    Since $\{\uv_1,\uv_2,\ldots,\uv_n\}$ is a basis, it spans $\W$.
    Also, vectors $\vv_1,\vv_2,\ldots,\vv_m$ are linearly independent
    because they also form a basis.  Thus, from the Morphing Lemma, $m\leq n$.

    \pause
    We can reverse the roles of $\uv_i$'s and $\vv_i$'s to obtain that $n\leq m$.

    \pause
    Therefore, $n=m$.
  \end{proof}
\end{frame}

\begin{frame}
  \begin{proof}[Proof of the Simplified Exchange Lemma]
    {\footnotesize
      \pause
      Let $S=\{\uv_1,\uv_2,\ldots,\uv_n\}$. 
      Since $\vect{z}\in\mathrm{Span}~S$, \pause
      we note that $\mathrm{Span}~S=\mathrm{Span}~(S\cup\{\vect{z}\})$.
      \pause
      We can also write
      \[
      \vect{z}=\alpha_1\uv_1+\alpha_2\uv_2+\ldots+\alpha_n\uv_n.
      \]
      \pause
      Because $\vect{z}$ is non-zero, there exists some non-zero $\alpha_i$.
      \pause
      We can rewrite the above equation as
      \[
      \alpha_i\uv_i=\vect{z} -
      \alpha_1\uv_1-\ldots
      -\alpha_{i-1}\uv_{i-1}-\alpha_{i+1}\uv_{i+1}-\cdots
      -\alpha_n\uv_n,
      \]
      \pause
      or
      \[
      \uv_i = \left(
      \frac{1}{\alpha_i}\vect{z} -
      \frac{\alpha_1}{\alpha_i}\uv_1-\ldots
      -\frac{\alpha_{i-1}}{\alpha_i}\uv_{i-1}
      -\frac{\alpha_{i+1}}{\alpha_i}\uv_{i+1}-\cdots
      -\frac{\alpha_n}{\alpha_i}\uv_n
      \right),
      \]
      \pause
      i.e., $\uv_i\in\mathrm{Span}~(S\cup\{\vect{z}\})$.  \pause In this
      case, we can remove $\uv_i$, i.e.,
      \[
      \mathrm{Span}~(S\cup\{\vect{z}\}-\{\uv_i\})=
      \mathrm{Span}~(S\cup\{\vect{z}\})=
      \mathrm{Span}~S.
      \]
      \pause
      Therefore we
      can let $\vect{w}=\uv_i$ and the lemma follows.
    }
  \end{proof}
  \pause

  {\small
    How can we prove the full lemma?
  }
\end{frame}

\begin{frame}
  \frametitle{Dimensions}

  \begin{block}{Definition}
    The \textcolor{red}{\bf dimension} of a vector space $\V$ is the size of its basis.
    
    The dimension of $\V$ is written as $\mathrm{dim}~\V$.
  \end{block}
\end{frame}
