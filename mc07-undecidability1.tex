\newcommand{\lecturetitle}[1]{
  \title{01204211 Discrete Mathematics \\ #1}
  \author{Jittat Fakcharoenphol}
  \frame{\titlepage}
}
\newcommand{\Mod}{\,\bmod\,}


\newcommand\sbullet[1][.5]{\mathbin{\vcenter{\hbox{\scalebox{#1}{$\bullet$}}}}}

\lecturetitle{Lecture 12a: Undecidability (1)} 
\renewcommand{\epsilon}{\varepsilon}

\newcommand{\czero}{{\mathtt 0}}
\newcommand{\cone}{{\mathtt 1}}
\newcommand{\sigcupgam}{\Sigma\cup\Gamma}

\frame{
  \frametitle{Non-context-free languages}

  While
  \[
  \{\czero^n\cone^n \;|\; n\geq 0\}
  \]
  is context free, the language
  \[
  \{\czero^n\cone^n\czero^n \;|\; n\geq 0\}
  \]
  is not.

  \pause

  Can we write a python program to check if a string $w$ belongs to
  the language $\{\czero^n\cone^n\czero^n \;|\; n\geq 0\}$?

}

\frame{
  \frametitle{Big question}

  Is there a python program that ``solves'' any possible problem?

  \pause

  Can a computer solve any problem?

  \pause
  
  Is there an algorithm that solves every problem?

  \pause

  What is the limit of computation?
  
}

\frame{
  \frametitle{Answer by a counting argument}

  If there are ``more'' problems than any possible algorithms, then
  there should be some problem that algorithms cannot solve.

  (Think of an algorithm as ``a python program.)
  
  \pause

  However, there are infinitely many python programs and there are
  infinitely many problems.  It is not obvious how to much such an
  argument formally.
  
}

\frame{
  \frametitle{Bijections}

  \begin{block}{Definition}
    \begin{itemize}
    \item
      A function $f:A\longrightarrow B$ from domain $A$ to range $B$
      is {\color{red}\bf one-to-one} if for any $x\neq y\in A$,
      $f(x)\neq f(y)$.
    \item
      A function $f:A\longrightarrow B$ from domain $A$ to range $B$
      is {\color{red}\bf onto} if for any $x'\in B$, there exists
      $x\in A$ such that $f(x)= x'$.
    \item
      A function $f:A\longrightarrow B$ is a {\color{red}\bf
        bijection} (or bijective) if it is one-to-one and onto.
    \end{itemize}
  \end{block}

}

\frame{
  \frametitle{Bijection: examples}
}

\frame{
  \begin{lemma}
    For any set $A$, there is no bijective function $f:A\longrightarrow 2^A$.
  \end{lemma}
  \begin{proof}
    We prove by contradiction.  Assume that there exists a bijective
    function $f$ from $A$ to $2^A$.  We construct a set $B\subseteq A$
    such that there is no $x\in A$ such that $f(x)=B$. \pause

    We define $B$ as follows.
    \[
    B = \{ x\in A \;|\; x\not\in f(x)\}.
    \]

    Now suppose that there exists $x\in A$ such that $f(x)=B$.  There
    are two cases to consider:

    {\bf Case 1:} $x\in B.$ \pause
    
    {\bf Case 2:} $x\not\in B.$ \pause

    In both case, we have a contradiction; therefore, our assumption
    is false.  Thus, there is no bijection between $A$ and $2^A$.
  \end{proof}
}

\frame{
  \frametitle{Example: finite set}

  Let $A={1,2,3,4,5,6,7}$.  Consider function $f:A\longrightarrow 2^A$ defined as

  \vspace{0.1in}

  \begin{columns}
    \begin{column}{0.5\textwidth}
      {\small
        \begin{eqnarray*}
          f(1) &=& \{\} \\ 
          f(2) &=& \{1,2,3\} \\ 
          f(3) &=& \{1,2,3,4,5,6,7\} \\ 
          f(4) &=& \{1,3,5,7\} \\ 
          f(5) &=& \{2,4,6\} \\ 
          f(6) &=& \{7\} \\ 
          f(7) &=& \{1,2,3\}
        \end{eqnarray*}
      }
    \end{column}
    \begin{column}{0.5\textwidth}
      \only<2>{
      \begin{tabular}{|c|c|c|c|c|c|c|c|}
        \hline
        & 1 & 2 & 3 & 4 & 5 & 6 & 7 \\
        \hline
        1 & &&&&&& \\
        \hline
        2 & &&&&&& \\
        \hline
        3 & &&&&&& \\
        \hline
        4 & &&&&&& \\
        \hline
        5 & &&&&&& \\
        \hline
        6 & &&&&&& \\
        \hline
        7 & &&&&&& \\
        \hline
      \end{tabular}
      }
    \end{column}
  \end{columns}

  \vspace{0.1in}
  
  $B=$
}

\frame{
  \frametitle{Example: infinite set}

  Let $A={\mathbb N}=\{1,2,3,4,\ldots\}$.  Consider function
  $f:A\longrightarrow 2^A$ defined as

  \vspace{0.1in}

  \begin{columns}
    \begin{column}{0.5\textwidth}
      {\small
        \begin{eqnarray*}
          f(1) &=& \{\} \\ 
          f(2) &=& \{1,2,3\} \\ 
          f(3) &=& \{1,2,3,4,5,6,7,\ldots\} \\ 
          f(4) &=& \{1,3,5,7,\ldots\} \\ 
          f(5) &=& \{2,4,6,\ldots\} \\ 
          f(6) &=& \{7\} \\ 
          f(7) &=& \{1,2,3,11,12,13,21,22,23,\ldots\}\\
          \vdots && \vdots
        \end{eqnarray*}
      }
    \end{column}
    \begin{column}{0.5\textwidth}
      \begin{tabular}{|c|c|c|c|c|c|c|c|c}
        \hline
        & 1 & 2 & 3 & 4 & 5 & 6 & 7 & $\cdots$ \\
        \hline
        1 & &&&&&&& \\
        \hline
        2 & &&&&&&& \\
        \hline
        3 & &&&&&&& \\
        \hline
        4 & &&&&&&& \\
        \hline
        5 & &&&&&&& \\
        \hline
        6 & &&&&&&& \\
        \hline
        7 & &&&&&&& \\
        \hline 
        $\vdots$ & &&&&&&& \\ 
      \end{tabular}
    \end{column}
  \end{columns}

  \vspace{0.1in}
  
  $B=$
}

\frame{

  The previous lemma informally states that there are ``more'' subsets
  than the number of elements in the set.

  Let's think about:
  \begin{itemize}
  \item A set of all python programs, and
  \item A set of all languages.
  \end{itemize}

  \pause

  Since each python program ``solves'' at most one language, there are
  not ``enough'' python programs to solve all possible language.

  \pause

  But what exactly is a problem that cannot be ``solved''?
}

\frame{
  \frametitle{Decision problems}

  \begin{itemize}
  \item Given an integer $x$, is $x$ odd?
  \item Given a string $w$, is $w$ palindrome?
  \item Given a string $w$, is $w\in\{\czero^n\cone^n\;|\;n\geq 0\}$?
  \item Given a map, a starting position $s$, a destination $t$, and
    an integer $k$, does there exist a path from $s$ to $t$ with
    distance at most $k$?
  \item Given a program $P$ and input string $w$, when running $P$
    with $w$ as an input, does $P$ terminate?
  \end{itemize}
}

\frame{
  \frametitle{Decision problems and languages}

  For this problem:
  
  \begin{block}{}
    Given an integer $x$, is $x$ odd?
  \end{block}

  we can define a corresponding language
  \[
  L_E=\{,\ldots,-6,-4,-2,0,2,4,6,\ldots\}.
  \]

  To solve this problem, given $x$, we can ask if $x\in L_E$.
  
}

\frame{
  \frametitle{Languages and programs}

  We will talk about languages of particular programs.  For example,
  let ${\mathbb P}$ be the set of all python programs.  In this case,
  ${\mathbb P}$ is a language.

  \pause

  \[
  \{ P \in {\mathbb P} \;|\; \mbox{$P$ always terminates}\}
  \]
  \pause
  \[
  \{ P \in {\mathbb P} \;|\; \mbox{$P$ always loops}\}
  \]
  \pause
  \[
  \{ (P,x) \;|\; P\in{\mathbb P}, \mbox{when running $P$ with $x$ as
    an input, $P$ terminates}\}
  \]
  \pause
  \[
  \{ (P,x) \;|\; P\in{\mathbb P}, \mbox{$P(x)$ terminates}\}
  \]
  \pause
  \[
  \{ (P,Q,x) \;|\; P,Q\in{\mathbb P}, \mbox{$P(x)$ and $Q(x)$ terminate with the same output.}\}
  \]
  
}

\begin{frame}[fragile=true]
  \frametitle{Programs and inputs}
  \begin{columns}
    \begin{column}{0.5\textwidth}
\begin{verbatim}
x = int(input())
if x % 2 == 0:
    print('yes')
else:
    print('no')
\end{verbatim}
    \end{column}
    \pause
    \begin{column}{0.5\textwidth}
      {\small
\begin{verbatim}
$ python le.py 
10
yes
$ python le.py 
7
no
\end{verbatim}
\pause
\begin{verbatim}
$ python le.py 
fjdsklfjsdf
Traceback (most recent call last):
  File "le.py", line 1, in <module>
    x = int(input())
ValueError: invalid literal for int()
with base 10: 'fjdsklfjsdf'
\end{verbatim}
\pause
\begin{verbatim}
$ python le.py < le.py
\end{verbatim}
\pause
\begin{verbatim}
Traceback (most recent call last):
  File "le.py", line 1, in <module>
    x = int(input())
ValueError: invalid literal for int()
with base 10: 'x = int(input())'
\end{verbatim}
      }
    \end{column}
  \end{columns}
\end{frame}

\begin{frame}[fragile=true]
  \frametitle{Nice programs}

  {\small
    We can systematically modify any python program $P$ so that
  \begin{itemize}
  \item $P$ contains a main function that works with the input as a string.
  \item $P$ never crashes.  (If the original $P$ crashes, the modified
    $P$ outputs {\tt no}.)
  \end{itemize}
  }
  
  \begin{columns}
    \begin{column}{0.5\textwidth}
\begin{verbatim}
x = int(input())
if x % 2 == 0:
    print('yes')
else:
    print('no')
\end{verbatim}
    \end{column}
    \pause
    \begin{column}{0.5\textwidth}
      {\small
\begin{verbatim}
import sys
def main(w):
    try:
        x = int(w)
        if x % 2 == 0:
            print('yes')
        else:
            print('no')
    except:
        print('no')

if __name__ == '__main__':
    w = sys.stdin.read()
    main(w)
\end{verbatim}
}
    \end{column}
  \end{columns}

\end{frame}

\frame{
  \frametitle{When running a program}

  When you run a program $P$ with input $x$, there are three possible
  outcomes:

  \begin{itemize}
  \item $P$ terminates and outputs {\tt\bf yes},
  \item $P$ terminates and outputs {\tt\bf no}, and
  \item $P$ does not terminate. (It runs forever.)
  \end{itemize}

  \pause

  \vspace{0.5in}
  
  {\bf Remarks:} if $P$ crashes (even after modification), we treat it
  as if it terminates and outputs {\tt\bf no}.

}

\frame{
  \frametitle{Proving impossibility}
}

\frame{
  \frametitle{Reduction: rough idea}
}

\frame{
  \frametitle{Language $A$}

  Let ${\mathbb P}$ be the set of all python programs.  Let the
  language $A$ be
  \[
  \{ P\in {\mathbb P} \;|\; \mbox{when running $P$ with $P$ as an
    input, $P$ terminates} \}
  \]
  
  \pause
  
  We use a function call notation $P(x)$ when refering to the
  execution of program $P$ with input $x$.
  
  \pause
  
  We restate the definition of $A$ as 
  \[
  \{ P\in {\mathbb P} \;|\; \mbox{$P(P)$ terminates} \}.
  \]
}



\frame{
  \frametitle{Deciders}

  We say that a python program $P$ {\color{red}\bf decides} the
  language $L$ if for any input string $x$, $P$ when running with $x$
  as an input,
  \begin{itemize}
  \item $P$ always terminates, 
  \item $P$ outputs {\tt\bf yes}, if $x\in L$, and
  \item $P$ outputs {\tt\bf no}, if $x\not\in L$.
  \end{itemize}

}

\frame{
  \frametitle{Deciders: more examples}
}

\frame{
  \frametitle{Language $A$}

  Let ${\mathbb P}$ be the set of all python programs.  Let the
  language $A$ be
  \[
  \{ P\in {\mathbb P} \;|\; \mbox{when running $P$ with $P$ as an
    input, $P$ terminates} \}
  \]
  
  \pause
  
  We use a function call notation $P(x)$ when refering to the
  execution of program $P$ with input $x$.
  
  \pause
  
  We restate the definition of $A$ as 
  \[
  \{ P\in {\mathbb P} \;|\; \mbox{$P(P)$ terminates} \}.
  \]
}

\begin{frame}[fragile=true]
  \frametitle{Not a decider for $A$}
  
\begin{verbatim}
Input: python program P (as a string)

1. Load module P as Pmod
2. Call Pmod.main(P)
3. print('yes')               # we reach this line,
                              #  only if M.main(P) terminates
\end{verbatim}
\end{frame}

\begin{frame}[fragile=true]

  \begin{lemma}
    There is no python program that decides $A$.
  \end{lemma}
  \pause

  We will see the proof at the end of class.
\end{frame}

\frame{
  \frametitle{Undecidability}

  If we believe that anything that a computer can do can be written as
  a python program, \pause and there is no python program that decides
  $A$, when we say that

  \begin{block}{}
    $A$ is undecidable.
  \end{block}

  Language $A$ will be very important later on, we give it a proper
  name as $\textsc{SelfHalt}$.

}

\frame{
  \frametitle{The proof as a table}
  List all python programs in ${\mathbb P}$ as $P_1,P_2,P_3,\ldots$.

  \vspace{0.2in}
  
  \begin{tabular}{|c|c|c|c|c|c|c|c|c}
    \hline
    & $P_1$ & $P_2$ & $P_3$ & $P_4$ & $P_5$ & $P_6$ & $P_7$ & \ldots \\
    \hline
    $P_1$ &&&&&&& \\
    \hline
    $P_2$ &&&&&&& \\
    \hline
    $P_3$ &&&&&&& \\
    \hline
    $P_4$ &&&&&&& \\
    \hline
    $P_5$ &&&&&&& \\
    \hline
    $P_6$ &&&&&&& \\
    \hline
    $\vdots$ &&&&&&& \\
    \hline
    $(B)$ &&&&&&& \\
    \hline
  \end{tabular}

  \vspace{0.2in}

  What does $B$ do on each input program $P_i$?
}

\frame{
  \frametitle{Another language $\textsc{Halt}$}

  Let
  \[
  \textsc{Halt} = \{ (P,w) \;|\;
  \mbox{$P$ is a python program such that $P(w)$ terminates}
  \}
  \]

  We shall prove that $\textsc{Halt}$ is also undecidable (if we
  believe that python programs represent all possible computation).

}

\begin{frame}[fragile=true]
  \begin{lemma}
    {\sc Halt} is undecidable.
  \end{lemma}
  \begin{proof}
    We prove the lemma by contradiction.  Assume that there is a
    python program $P$ that decides {\sc Halt}. \pause

    We construct a program $C$ as follows
    {\small
\begin{verbatim}
Program C
Input Q
1.   Load P as module Pmod
2.   if Pmod.main(Q,Q) == 'yes':
3.      print('yes')
4.   else
5.      print('no')
\end{verbatim}
}
  \end{proof}
\end{frame}

\begin{frame}[fragile=true]
  \begin{proof}
    We prove the lemma by contradiction.  Assume that there is a
    python program $P$ that decides {\sc Halt}.

    We construct a program $C$ as follows
    {\small
\begin{verbatim}
Program C
Input Q
1.   if P(Q,Q) == 'yes':
2.      print('yes')
3.   else
4.      print('no')
\end{verbatim}
    }

    \pause

    Given program $P$, we can construct a program $C$ that decides
    {\sc SelfHalt}.  \pause However, we know that {\sc SelfHalt} is
    undecidable; thus, we reach a contradiction.

    We conclude that there does not exist a python program $P$ that
    decides {\sc Halt}.
  \end{proof}
\end{frame}

\begin{frame}
  \frametitle{Reduction}
  \pause

  \begin{itemize}
  \item We show that if {\sc Halt} is decidable, then {\sc SelfHalt} is
    also decidable.
    \pause
  \item However, {\sc SelfHalt is undecidable}.
    \pause
  \item We conclude that {\sc Halt} is also undecidable.
  \end{itemize}
\end{frame}

\begin{frame}
  \frametitle{Reduction in picture}
\end{frame}

\begin{frame}[fragile=true]
  
  {\small
    Let $\textsc{Accept} = \{(P,w) \;|\;
    \mbox{$P\in{\mathbb P}$ and $P(w)$ terminates with acceptance}\}.$
  }
  
  \begin{lemma}
    {\sc Accept} is undecidable.
  \end{lemma}
  \begin{block}{Proof.}

      We prove the lemma by contradiction.  Assume that there is a
      python program $Q$ that decides {\sc Accept}. \pause We
      construct a program $C$ that decides {\sc Halt} as follows \pause
      
      {\footnotesize
\begin{verbatim}
Program C       
Input P,w       
1.   Replace every "print('no')" statement in P with "print('yes')"
1.   if Q(P,w) == 'yes':
2.      print('yes')
3.   else
4.      print('no')
\end{verbatim}
      }

  \end{block}
\end{frame}

\begin{frame}[fragile=true]
  \begin{proof}[Proof (cont.)]
    {\footnotesize
\begin{verbatim}
Program C       
Input P,w       
1.   Replace every "print('no')" statement in P with "print('yes')"
1.   if Q(P,w) == 'yes':
2.      print('yes')
3.   else
4.      print('no')
\end{verbatim}
    }   
    {\small
      We have to make sure that our reduction is correct by considering
      two cases.
      
      Case 1: when $P(w)$ halts. \pause

      Case 2: when $P(w)$ does not halt. \pause

      Since in both cases, $C$ answers correctly, we know that given
      program $Q$ deciding {\sc Accept}, we can construct a program
      $C$ that decides {\sc Halt}.  However, we know that {\sc Halt}
      is undecidable; thus, we reach a contradiction.  We conclude
      that {\sc Accept} is also undecidable.  }
  \end{proof}
\end{frame}

\frame{
  \frametitle{Reduction from {\sc Halt} to {\sc Accept} in picture}
  
}

\frame{
  \frametitle{How about {\sc Reject}?}

  Let
  \[
  \textsc{Reject} = \{(P,w) \;|\;
  \mbox{$P\in{\mathbb P}$ and $P$ rejects $w$}\}.
  \]

  
}

\begin{frame}[fragile=true]

  \begin{lemma}
    There is no python program that decides $\textsc{SelfHalt}$.
  \end{lemma}
  \begin{proof}
    We prove by contradiction. Assume that there is a python program
    $P$ that decides $\textsc{SelfHalt}$.  \pause

    We describe a python program $B$ that reads a string $Q$ as an
    input as follows:

    {\small
\begin{verbatim}
Program B
Input Q
1.    Load P as module Pmod
2.    if Pmod.main(Q) == 'yes':    # when Pmod outputs yes
3.       while True: pass          #   loop forever 
4.    else:                        # when Pmod outputs no
5.       quit()                    #   halt
\end{verbatim}
    }

    Given program $Q$ as an input, $B$ loops forever when \pause

    It terminates when
  \end{proof}
\end{frame}

\frame{
  \begin{proof}
    We know that
    \begin{itemize}
    \item $B(Q)$ loops when $Q(Q)$ terminates, and
    \item $B(Q)$ terminates when $Q(Q)$ loops.
    \end{itemize}

    Does running $B$ using $B$ as an input terminate?

    \pause

    Let's try to plug in $Q=B$.  We have
    \begin{itemize}
    \item $B(B)$ loops when \pause $B(B)$ terminates, \pause and
    \item $B(B)$ terminates when \pause $B(B)$ loops. \pause
    \end{itemize}

    Since either $B(B)$ loops or terminates, and we cannot be in any
    of the cases, we obtain a contradiction.

    Therefore, we conclude that program $P$ does not exist.
  \end{proof}
}


\frame{
  \frametitle{Python as computation}

  Do you believe in this assumption:

  {\bf Anything that a computer can do can be written as a python
    program. }

}

\frame{
  \frametitle{Turing machines}

  {\bf Anything that a computer can do can be carried out using Turing
    machines. }

  \vspace{0.3in}
  \pause
  
  {\bf Any possible computation can be performed by Turing machines. }

}


