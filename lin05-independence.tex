\newcommand{\lecturetitle}[1]{
  \title{01204211 Discrete Mathematics \\ #1}
  \author{Jittat Fakcharoenphol}
  \frame{\titlepage}
}
\newcommand{\Mod}{\,\bmod\,}

\lecturetitle{Lecture 9c: Linear Independence and Bases} 

\begin{frame}
  \frametitle{Review: Linear combinations}

  \begin{block}{Definition}
    For any scalars
    \[
    \alpha_1,\alpha_2,\ldots,\alpha_m
    \]
    and vectors
    \[
    \uv_1,\uv_2,\ldots,\uv_m,
    \]
    we say that
    \[
    \alpha_1\uv_1 + \alpha_2\uv_2 + \cdots + \alpha_m \uv_m
    \]
    is a \textcolor{red}{\bf linear combination} of $\uv_1,\ldots,\uv_m$.
  \end{block}
\end{frame}

\begin{frame}
  \frametitle{Review: Span}

  \begin{block}{Definition}
    A set of all linear combination of vectors $\uv_1,\uv_2,\ldots,\uv_m$ is called the \textcolor{red}{\bf span} of that set of vectors.

    It is denoted by $\mathrm{Span} \{\uv_1,\uv_2,\ldots,\uv_m\}$.
  \end{block}
\end{frame}


\begin{frame}
  \frametitle{Previous Lemmas}
  \begin{lemma}
    Consider vectors $\uv_1,\uv_2,\ldots,\uv_n$.
    If $\vv_1,\vv_2,\ldots,\vv_k$ are generators for $\V$, and for each $i$,
    \[
    \vv_i\in \mathrm{Span}\ \{\uv_1,\uv_2,\ldots,\uv_n\},
    \]
    we have that $\V \subseteq \mathrm{Span}\ \{\uv_1,\uv_2,\ldots,\uv_n\}$.
  \end{lemma}
\end{frame}

\begin{frame}
  \begin{lemma}
    Consider vectors $\uv_1,\uv_2,\ldots,\uv_n$.
    If $\vv\in \mathrm{Span}\ \{\uv_1,\uv_2,\ldots,\uv_n\}$, then
    \[
    \mathrm{Span}\ \{\uv_1,\uv_2,\ldots,\uv_n,\vv\}
    =
    \mathrm{Span}\ \{\uv_1,\uv_2,\ldots,\uv_n\}
    \]
    \label{lemma:sup-vect}
  \end{lemma}
  \pause
  \begin{lemma}
    Consider vectors $\uv_1,\uv_2,\ldots,\uv_n$.
    If $\uv_n\in \mathrm{Span}\ \{\uv_1,\uv_2,\ldots,\uv_{n-1}\}$, then
    \[
    \mathrm{Span}\ \{\uv_1,\uv_2,\ldots,\uv_{n-1}\}
    =
    \mathrm{Span}\ \{\uv_1,\uv_2,\ldots,\uv_n\}
    \]
  \end{lemma}
\end{frame}

\begin{frame}
  \begin{proof}[Proof of Lemma~\ref{lemma:sup-vect}]

    {\small
      
      Since $\vv$ can be written as a linear combination of other
      vectors, there exist $\alpha_1,\alpha_2,\ldots,\alpha_n$ such
      that
      \[
      \vv = \alpha_1\uv_1+\alpha_2\uv_2+\cdots+\alpha_n\uv_n.
      \]
      
      Consider any vector $\vect{w}\in\mathrm{Span}~\{\uv_1,\uv_2,\ldots,\uv_n,\vv\}$; thus, we can write
      \[
      \vect{w} = \beta_0\vv + \beta_1\uv_1 + \beta_2\uv_2 +\cdots+\beta_n\uv_n.
      \]
      Plugging in $\vv$, we get that
      \begin{align*}
        \vect{w} & =  \beta_0\left(\alpha_1\uv_1+\cdots+\alpha_n\uv_n\right) + \beta_1\uv_1 + \beta_2\uv_2 +\cdots+\beta_k\uv_k \\
        &= (\beta_0\alpha_1+\beta_1)\uv_1 +
        (\beta_0\alpha_2+\beta_2)\uv_2 +
        \cdots+(\beta_0\alpha_n+\beta_n)\uv_n,
      \end{align*}
      implying that $\vect{w}\in\mathrm{Span}~\{\uv_1,\uv_2,\ldots,\uv_n\}$.

    }
  \end{proof}
\end{frame}

\begin{frame}
  \frametitle{Linearly independence}

  \begin{block}{Definition}
    Vectors $\uv_1,\uv_2,\ldots,\uv_n$ are \textcolor{red}{\bf
      linearly independent} if no vector $\uv_i$ can be written as a
    linear combination of other vectors.
  \end{block}

  \pause

  \begin{block}{(Another) Definition}
    Vectors $\uv_1,\uv_2,\ldots,\uv_n$ are \textcolor{red}{\bf
      linearly independent} if the only solution of equation
    \[
    \alpha_1\uv_1 + \alpha_2\uv_2 +\cdots+\alpha_n\uv_n = \vect{0}
    \]
    is
    \[
    \alpha_1=\alpha_2=\cdots=\alpha_n=0.
    \]
  \end{block}

\end{frame}

\begin{frame}
  \frametitle{Examples in $\rf^2$}
\end{frame}

\begin{frame}
  \frametitle{Examples in $\rf^3$}
\end{frame}

\begin{frame}
  \frametitle{Examples in $GF(2)$}
\end{frame}

\begin{frame}
  \frametitle{Examples in linear systems}
\end{frame}

\begin{frame}
  \frametitle{Subset of linearly independent vectors}
  \begin{lemma}
    If $A=\{\uv_1,\uv_2,\ldots,\uv_n\}$ be a set of linearly
    independent vectors, then any $B\subseteq A$ is also a set of
    linearly independent vectors.
  \end{lemma}
  \begin{proof}
    {\small
      We prove by contradiction.  Assume that $B$ is {\bf not}
      linearly independent.  Without loss of generality, assume that
      $B=\{\uv_1,\uv_2,\ldots,\uv_k\}$ where $k\leq n$.  \pause This
      means that there exists $\alpha_1,\alpha_2,\ldots,\alpha_k$ such
      that
      \[
      \alpha_1\uv_1+\alpha_2\uv_2+\cdots+\alpha_k\uv_k=\vect{0},
      \]
      and some $\alpha_i$'s is nonzero.  \pause If we let
      $\alpha_{k+1}=\alpha_{k+2}=\cdots=\alpha_n=0$, we have that
      \[
      \alpha_1\uv_1+\alpha_2\uv_2+\cdots+\alpha_n\uv_n=\vect{0},
      \]
      with some $\alpha_i$'s being nonzero as well.  \pause This
      implies that vectors in $A$ are not linearly indepedent; leading
      to a contradiction.
    }
  \end{proof}
\end{frame}

\begin{frame}
  \frametitle{Bases}

  \begin{block}{Definition}
    A set of vectors $\{\uv_1,\uv_2,\ldots,\uv_k\}$ is a \textcolor{red}{\bf basis} for vector space $\V$ if
    \begin{itemize}
    \item $\mathrm{Span}~\{\uv_1,\uv_2,\ldots,\uv_k\} = \V$, and
    \item $\uv_1,\uv_2,\ldots,\uv_k$ are linearly independent.
    \end{itemize}
  \end{block}
\end{frame}

\begin{frame}
  \frametitle{Examples 1: $\rf^2$ and $\rf^3$}
\end{frame}

\begin{frame}
  \frametitle{Examples 2}
\end{frame}

\begin{frame}
  \begin{lemma}[Unique representation]
    Let $\uv_1,\uv_2,\ldots,\uv_k$ be a basis for vector space $\V$.
    For any $\vv\in\V$, there is a unique way to write $\vv$ as a
    linear combination of $\uv_1,\ldots,\uv_k$.
  \end{lemma}
\end{frame}

\begin{frame}
  \begin{proof}[Proof of unique representation lemma]
    {\small
  We prove by contradiction.  \pause Assume that there exists a vector
  $\vv\in\V$ with more than one ways to be written as linear
  combinations of the basis.  Thus, there exist
  \[ \alpha_1,\alpha_2,\ldots,\alpha_k, \] and
  \[ \beta_1,\beta_2,\ldots,\beta_k, \]
  that are not equal (i.e., there exists $i$ where
  $\alpha_i\neq\beta_i$)
  such that
  $\vv=\alpha_1\uv_1+\alpha_2\uv_2+\cdots+\alpha_k\uv_k$ and
  $\vv=\beta_1\uv_1+\beta_2\uv_2+\cdots+\beta_k\uv_k$.  \pause
  This implies that
  \[
  \alpha_1\uv_1+\alpha_2\uv_2+\cdots+\alpha_k\uv_k=\vv=\beta_1\uv_1+\beta\uv_2+\cdots+\beta_k\uv_k,
  \]
  and
  \[
  (\alpha_1-\beta_1)\uv_1 +
  (\alpha_2-\beta_2)\uv_2 + \cdots + (\alpha_k-\beta_k)\uv_k=0.
  \]
  \pause
  Since $\alpha_i\neq\beta_i$, we have that at least one of the
  coefficients is non-zero, implying that $\uv_1,\ldots,\uv_k$ are not
  linearly independent.  This contradicts the assumption that
  $\uv_1,\ldots,\uv_k$ form a basis.
  }
  \end{proof}
\end{frame}
