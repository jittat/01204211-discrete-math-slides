\newcommand{\lecturetitle}[1]{
  \title{01204211 Discrete Mathematics \\ #1}
  \author{Jittat Fakcharoenphol}
  \frame{\titlepage}
}
\newcommand{\Mod}{\,\bmod\,}

\lecturetitle{Lecture 16: Binomial Coefficients (3)} 

\begin{frame}\frametitle{The binomial coefficients\footnote{This lecture mostly follows Chapter 3 of [LPV].}}
  In this lecture, we discuss advanced counting with binomial
  coefficients.  Then we shall study the function $\binom{n}{k}$
  itself.  First, let's see the actual value of the binomial
  coefficients $\binom{n}{k}$ for various values of $n$.  \vspace{2in}
\end{frame}

\begin{frame}\frametitle{More on counting}
  We shall see more techniques for counting when we consider the
  following problems.
  \begin{itemize}
  \item How many anagrams does the word ``KASETSARTUNIVERSITY'' have?
    (They do not have to be real English words.)
  \item How can you give out $n$ different presents to $k$ students
    when student $i$ has to get $n_i$ pieces of presents?
  \item How many ways can you distribute $n$ baht coins to $k$
    children?
  \end{itemize}
\end{frame}

\begin{frame}\frametitle{Easy anagrams}
  \begin{itemize}
  \item An anagram of a particular word is a word that uses the same
    set of alphabets.  For example, the anagrams of $ADD$ are $ADD$,
    $DAD$, and $DDA$. \pause
  \item How many anagrams does ``$ABCD$'' have? \pause
    \begin{itemize}
    \item $4!$, because every permutation of A B C or D is a different
      anagram. \pause
    \end{itemize}
  \end{itemize}
\end{frame}

\begin{frame}\frametitle{Harder anagrams}
  \begin{itemize}
  \item How many anagrams does ``$ABCC$'' have? Is it $4!$ ? \pause
    \begin{itemize}
    \item This time we have to be careful because the answer of $4!$
      is too large as it over counts many anagrams, i.e., it
      ``distinguishes'' the two $C$'s. \pause
    \item Let's try to be concrete. How many times does ``$CABC$'' get
      counted in $4!$? \pause
    \item If we treat two $C$'s differently as $C_1$ and $C_2$, we can
      see that $CABC$ is counted twice as $C_1ABC_2$ and $C_2ABC_1$.
      This is true for any anagram of $ABCC$.  \pause
    \item Since each anagram is counted in $4!$ twice, the number of
      anagrams is $4! / 2 = 4\cdot 3 = 12$.
    \end{itemize}
  \end{itemize}
\end{frame}

\begin{frame}\frametitle{General anagrams}
  \begin{tcolorbox}
    Let's try to use the same approach to count the anagram of
    $HELLOWORLD$. (It has 3 $L$'s, 2 $O$'s, $H$, $E$, $W$, $R$, and
    $D$.)
  \end{tcolorbox}
  
  \pause
  \vspace{0.2in}
  
  The number of permutation of alphabets in $HELLOWORLD$, treating
  each character differently is $10!$.  However, each anagram is
  counted for $3!2!$ times because of the 3 copies of $L$ and the 2
  copies of $O$.  Therefore, the number of anagrams is
  \[
  \frac{10!}{3!2!}.
  \]
\end{frame}

\begin{frame}\frametitle{Distributing presents}
  \begin{tcolorbox}
    I have $9$ different presents.  I want to give them to $3$
    students: A, B, and C.  I want to give each student $3$ presents.
    In how many ways can I do it?
  \end{tcolorbox}
  
  \pause

  {\small
    \begin{itemize}
    \item Let's think about the process of distributing the
      presents. \pause We can first let A choose $3$ presents, then B
      chooses the next $3$ presents, and C chooses the last $3$
      presents. \pause If we distinguish the order which each child
      chooses the presents, then there are $9!$ ways. \pause However, in
      this case, we consider the distribution of presents, i.e., we
      consider the set of presents each child gets. \pause
    \item To see how many times each distribution is counted in the $9!$
      ways, we can let children form a line and let each child permute
      his or her presents.  Each child has $3!$ choices.  Thus, one
      distribution appears $3!3!3!$ times. \pause
    \item Thus, the number of ways we can distribute presents is
      \[ 
      \frac{9!}{3!3!3!}
      \]
    \end{itemize}
  }
\end{frame}

\begin{frame}\frametitle{Another way to look at the present distribution}
  \begin{itemize}
  \item Let's look closely at a particular present distribution in the
    previous question.  Let $\{1,2,\ldots,9\}$ be the set of presents.
  \item Consider the case where A gets $\{1,3,8\}$, B gets
    $\{2,4,6\}$, and C gets $\{5,7,9\}$. \pause
  \item Another way to look at this distribution is to fix the order
    of the presents and see who gets each of the presents.  Thus, the
    previous distribution is represented in the following table:
    \begin{tabular}{|c|c|c|c|c|c|c|c|c|c|}
      Presents & 1 & 2 & 3 & 4 & 5 & 6 & 7 & 8 & 9\\ \hline
      Children & A & B & A & B & C & B & C & A & C
    \end{tabular}
  \item \pause This is essentially an anagram problem.  You can think
    of one particular way of present distribution as anagram of
    AAABBBCCC.  Thus, we reach the same solution of
    \[\frac{9!}{3!3!3!}.\]
  \end{itemize}
\end{frame}

\begin{frame}\frametitle{Distributing identical presents}
  \begin{tcolorbox}
    Now suppose that I have $9$ identical presents.  I want to give
    them to $3$ students: A, B, and C.  I want to give each student
    $3$ presents.  In how many ways can I do it?
  \end{tcolorbox}
  \begin{itemize}
  \item Note that when we state that the presents are identical, we
    mean that we do not distinguish them, i.e., the first present and
    the second present are indistinguishable.
  \end{itemize}
  \vspace{1in}
\end{frame}

\begin{frame}\frametitle{Distributing coins (1)}
  \begin{tcolorbox}
    I have $9$ indentical coins.  I want to give them to $3$ students:
    A, B, and C.  In how many ways can I do it so that each student
    gets at least one coin?
  \end{tcolorbox}

  \begin{itemize}
  \item Let's first try to organize the distribution of coins.  \pause
    We place all 9 coins in a line.  We let the first student picks
    some coin, then the second student, then the last one. \pause
  \item Since each coin is identical, we can let the first student
    picks the coin from the beginning of the line.  Then the second
    one pick the next set of coins, and so on. \pause
  \item One possible distribution is
    \[
    \underbrace{o o}_{1} \underbrace{o o o o}_{2} \underbrace{o o o}_{3}
    \]
    \pause
  \item In how many ways can we do that?
  \end{itemize}
\end{frame}

\begin{frame}\frametitle{Distributing coins (2)}
  The example below provides us with a hint on how to count.
  \[
  \underbrace{o o}_{1} \underbrace{o o o o}_{2} \underbrace{o o o}_{3}
  \]
  \pause

  Since all coins are identical, what matters are where the first
  student and the second student stop picking the coins. \pause
  I.e, the previous example can be depicted as
  \[
  o o | o o o o | o o o
  \]

  Thus, in how many ways can we do that? \pause
  
  Since there are 8 places we can mark starting points, and
  there are 2 starting points we have to place, then there are
  $\binom{8}{2}$ ways to do so. \pause

  This is a fairly surprising use of binomial coefficients.
\end{frame}

\begin{frame}\frametitle{Distributing coins (3)}
  Let's consider a general problem where we have $n$ identical coins
  to give out to $k$ students so that each student gets at least one
  coin.  In how many ways can we do that?

  \pause Since there are $n-1$ places between $n$ coins and we need to
  place $k-1$ starting points, there are $\binom{n-1}{k-1}$ ways to do
  so.

  \pause
  \begin{tcolorbox}
    There are $\binom{n-1}{k-1}$ ways to distribute $n$ identical
    coins to $k$ children so that each child get at least one coin.
  \end{tcolorbox}
\end{frame}

\begin{frame}\frametitle{Distributing coins (4)}
  \begin{tcolorbox}
    I have $9$ indentical coins.  I want to give them to $3$ students:
    A, B, and C.  In how many ways can I do it, given that some
    student may not get any coins?
  \end{tcolorbox}
  
  \vspace{1.5in}
\end{frame}

\begin{frame}\frametitle{Identities in the Triangle}
  \begin{tcolorbox}
    {\footnotesize
      \begin{tabular}{ccccccccccccccc}
        & & & & & & & 1 & & & & & & & \\
        & & & & & & 1 & & 1 & & & & & & \\
        & & & & & 1 & & 2 & & 1 & & & & & \\
        & & & & 1 & & 3 & & 3 & & 1 & & & & \\
        & & & 1 & & 4 & & 6 & & 4 & & 1 & & & \\
        & & 1 & & 5 & & 10 & & 10 & & 5 & & 1 & & \\
        & 1 & & 6 & & 15 & & 20 & & 15 & & 6 & & 1 & \\
        1 & & 7 & & 21 & & 35 & & 35 & & 21 & & 7 & & 1 \\
      \end{tabular}
    }
    \vspace{0.2in}
  \end{tcolorbox}
\end{frame}

\begin{frame}\frametitle{The binomial coefficients\footnote{This lecture mostly follows Chapter 3 of [LPV].}}
  We now focus on the function $\binom{n}{k}$.

  First, let's see the actual value of the binomial coefficients
  $\binom{n}{k}$ for various values of $n$.  \vspace{2in}
\end{frame}

\begin{frame}\frametitle{What do you see?}
  \begin{itemize}
  \item The function $\binom{n}{\cdot}$ is symmetric around $n/2$.
  \item Why? \pause This is true because we know that $\binom{n}{k}=\binom{n}{n-k}$. \pause
  \item The maximum is at the middle, i.e., when $n$ is even the
    maximum is at $\binom{n}{n/2}$ and when $n$ is odd, the maximum is
    at $\binom{n}{\lfloor n/2 \rfloor}$ and $\binom{n}{\lceil
      n/2\rceil}$.
  \item Why? \pause Can we prove that?
  \end{itemize}
\end{frame}

\begin{frame}\frametitle{Largest in the middle}
  To understand the behavior of $\binom{n}{k}$ as $k$ changes, let's
  look at two consecutive values:
  \[ \binom{n}{k} \ \ \heartsuit \ \ \binom{n}{k+1}\]
  \pause

  Let's write them out:
  \[ \frac{n(n-1)(n-2)\cdots(n-k+1)}{k!} \ \heartsuit \ \frac{n(n-1)(n-2)\cdots(n-k)}{(k+1)k!}.\]
  \pause
  Removing common terms, we can see that we are comparing these two terms:
  \[ 1 \ \heartsuit \ \frac{n-k}{k+1} \Leftrightarrow k \ \heartsuit \ \frac{n-1}{2},\]
  that is, \pause
  \begin{itemize}
  \item if $k<(n-1)/2$, $\binom{n}{k} < \binom{n}{k+1}$; and
  \item if $k>(n-1)/2$, $\binom{n}{k} > \binom{n}{k+1}$.
  \end{itemize}
\end{frame}

\begin{frame}\frametitle{How large is the middle $\binom{n}{n/2}$}
  Here, to simplify the calculation, we shall only consider the case
  when $n$ is even. Let's try to estimate the value of
  $\binom{n}{n/2}$ by finding its upper and lower bounds.
  \pause

  A simple upper bound can be obtain using the fact that
  $\binom{n}{n/2}$ counts subsets of certain size:
  \[\binom{n}{n/2} < 2^n.\]
  \pause

  We can also get a lower bound by noting that the maximum must be at
  least the average, i.e.,
  \[\binom{n}{n/2} \geq \frac{2^n}{n+1}\]
\end{frame}

\begin{frame}
  Combining both bounds, we get that
  \[\frac{2^n}{n+1}\leq \binom{n}{n/2} < 2^n.\]

  \pause Let's plug in $n=200$, and calculate the number of digits to
  see how close these bounds.
  \[27.80 \approx 200\cdot\log 2 - \log 201 \leq \log\binom{n}{n/2} < 200\cdot \log 2\approx 30.10\]
  \pause

  Can we get a better approximation? \pause

  Yes, with Stirling's formula. (homework)
\end{frame}

\begin{frame}\frametitle{Concentration}
  \begin{itemize}
  \item We know that the maximum of $\binom{n}{k}$ is obtained when
    $k=n/2$.  From the graph, you can see that, as you move further
    from the middle, the value of the function drops rapidly.
  \item Since we consider even $n$, we let $2m=n$.  One way to
    quantify how fast the values drop is to think about the ratio
    \[ \binom{2m}{m-t}\Big/\binom{2m}{m}. \] \pause
  \item In fact, it is known that
    \[ \binom{2m}{m-t}\Big/\binom{2m}{m} \approx e^{-t^2/m} \]
    \pause
  \item We will use our basic tools to obtain weaker bounds.
  \end{itemize}
\end{frame}

\begin{frame}\frametitle{How close is the approximation?}
  The estimation $e^{-t^2/m}$ is extremely close as shown in the
  figure below, where the gray bars are the actual value of
  $\binom{2m}{m-t}/\binom{2m}{m}$ and the red line is $e^{-t^2/m}$.

  \includegraphics[height=2.2in]{images/binom-approx.png}
\end{frame}

\begin{frame}\frametitle{The actual values}
  Because dealing with numbers less than 1 with logarithms is
  error-prone, we will work on the reciprocal.  Let's try to calculate
  the ratio
  \begin{eqnarray*}
    \binom{2m}{m}\Big/\binom{2m}{m-t}
    &=&
    \frac{(2m)!}{m!m!}\times\frac{(2m-m+t)!(m-t)!}{(2m)!}\\
    &=&
    \frac{(m+t)(m+t-1)\cdots(m+1)}{m(m-1)(m-2)\cdots(m-t+1)}.
  \end{eqnarray*}
  \pause
  We can use the same logarithm trick.  We have that the log of the
  ratio is
  \[
  \ln\left(\frac{m+t}{m}\right) +
  \ln\left(\frac{m+t-1}{m-1}\right)+\cdots+
  \ln\left(\frac{m+1}{m-t+1}.\right).
  \]
  \pause
  Then we can apply the bounds we have for $\ln x$:
  \[
  \frac{x-1}{x}\leq \ln x \leq x - 1
  \]
\end{frame}

\begin{frame}\frametitle{The upper bound on the reciprocal}
  Each term in the sum is in this form $\ln((m-i)/(m+t-i))$.  Applying
  the upper bound, we get
  \[ \ln\left(\frac{m+t-i}{m-i}\right) \leq \frac{m+t-i}{m-i}-1 = \frac{m+t-i-m+i}{m-i}=\frac{t}{m-i}.\]
  \pause

  Let's sum them up to get
  \[
    \ln\left(\frac{m+t}{m}\right) +
    \ln\left(\frac{m+t-1}{m-1}\right)+\cdots+
    \ln\left(\frac{m+1}{m-t+1}.\right) \quad\quad\quad
  \]
  \begin{eqnarray*}
    \quad\quad &\leq&
    \frac{t}{m} + \frac{t}{m-1} + \cdots + \frac{t}{m-t+1} \\
    &\leq&
    \frac{t}{m-t+1} + \frac{t}{m-t+1} + \cdots + \frac{t}{m-t+1} \\
    & = & \frac{t^2}{m-t+1}. \\
  \end{eqnarray*}
\end{frame}

\begin{frame}
  This implies that
  \[
  \ln\left(\frac{(m+t)(m+t-1)\cdots(m+1)}{m(m-1)(m-2)\cdots(m-t+1)}\right)
  \leq \frac{t^2}{m-t+1},
  \]
  i.e.,
  \begin{eqnarray*}
    \binom{2m}{m}\Big/\binom{2m}{m-t} &=& 
    \left(\frac{(m+t)(m+t-1)\cdots(m+1)}{m(m-1)(m-2)\cdots(m-t+1)}\right)\\
    &\leq& e^{t^2/(m-t+1)}.
  \end{eqnarray*}
  \pause

  Taking the reciprocal, we get
  \[
  e^{-t^2/(m-t+1)}\leq
  \binom{2m}{m-t}\Big/\binom{2m}{m}.
  \]
\end{frame}

\begin{frame}\frametitle{Upper bounds}
  Using the same approach, we can show that
  \[
  \binom{2m}{m-t}\Big/\binom{2m}{m}\leq e^{-t^2/(m+t)}.
  \]
  \pause

  Thus, we derived the estimates:
  \begin{tcolorbox}
    \[
    e^{-t^2/(m-t+1)}\leq
    \binom{2m}{m-t}\Big/\binom{2m}{m}
    \leq e^{-t^2/(m+t)},
    \]
  \end{tcolorbox}
  \pause
  which is fairly close the the estimate of $e^{-t^2/m}$.
\end{frame}

\begin{frame}\frametitle{How fast?}
  \begin{itemize}
  \item
    Let's return to the question on how fast do the values of the
    binomial coefficients decrease as you move further from the
    middle.  Let's use the better estimate
    $\binom{2m}{m-t}\big/\binom{2m}{m}\approx e^{-t^2/m}$.  \pause

  \item
    Given a constant $C$, we want to estimate the value of $t$ such
    that $\binom{2m}{m-t}$ is less than $\binom{2m}{m}\big/C$.  (E.g.,
    we can set $C=2$ to see when the value drops by 50\%.) Therefore,
    we want to find $t$ such that
    \[
    1/C \geq 
    \binom{2m}{m-t}\big/\binom{2m}{m}\approx e^{-t^2/m}
    \]
    Taking the logs, we get
    \[
    \ln 1/C = -\ln C \geq 
    \ln \binom{2m}{m-t}\big/\binom{2m}{m}\approx -t^2/m.
    \]
    This is true when
    \[
    t \geq \sqrt{m\ln C}.
    \]
  \end{itemize}
\end{frame}

\begin{frame}\frametitle{What does this means?}
  As an example, let $m=20$ and $C=2$.  We know that when $t$ is
  approximately $\sqrt{20\cdot \ln 2} = 3.723$ the value of
  $\binom{2m}{m-t}$ drops by 50\%.

  \includegraphics[height=2.2in]{images/binom-approx.png}

\end{frame}
