\newcommand{\lecturetitle}[1]{
  \title{01204211 Discrete Mathematics \\ #1}
  \author{Jittat Fakcharoenphol}
  \frame{\titlepage}
}
\newcommand{\Mod}{\,\bmod\,}

\lecturetitle{Lecture 10b: Polynomials (2)\footnote{This section is from Berkeley CS70 lecture notes.}} 

\begin{frame}
  \frametitle{Fun fact: Check digit for Thai National ID}
\end{frame}

\begin{frame}
  \frametitle{Review: Polynomials}

  A \textcolor{red}{\bf single-variable polynomial} is a function
  $p(x)$ of the form
  \[
  p(x) = a_dx^d+a_{d-1}x^{d-1}+\cdots+a_1x+a_0.
  \]

  We call $a_i$'s {\em coefficients}.  Usually, variable $x$ and
  coefficients $a_i$'s are real numbers.  The \textcolor{red}{\bf
    degree} of a polynomial is the largest exponent of the terms with
  non-zero coefficients.
  
\end{frame}

\begin{frame}
  \frametitle{Review: Basic facts}

  \begin{block}{Definition}
    $a$ is a \textcolor{red}{\bf root} of polynomial $f(x)$ if
    $f(a)=0$.
  \end{block}
  
  \begin{block}{Properties}
    {\bf Property 1:} A non-zero polynomial of degree $d$ has at most
    $d$ roots.

    {\bf Property 2:} Given $d+1$ pairs
    $(x_1,y_1),\ldots,(x_{d+1},y_{d+1})$ with distinct $x_i$'s, there
    is a {\em unique} polynomial $p(x)$ of degree at most $d$ such
    that $p(x_i)=y_i$ for $1\leq i\leq d+1$.
  \end{block}
\end{frame}

\begin{frame}
  \frametitle{Polynomial division}
  \pause

  \begin{tcolorbox}
    If you have a polynomial $p(x)$ of degree $d$, you can divide it
    with a polynomial $q(x)$ of degree $\leq d$.  You have that there
    exists a pair of polynomial $q'(x)$ and $r(x)$ such that
    \[
    p(x) = q'(x)q(x) + r(x),
    \]
    and $r(x)$ is of degree {\bf less} than $q(x)$'s degree.
  \end{tcolorbox}
\end{frame}

\begin{frame}
  \begin{lemma}
    If $a$ is a root of polynomial $p(x)$ with degree $d\geq 1$, then
    $p(x)=(x-a)q(x)$ for some polynomial $q(x)$ with degree at most
    $d-1$
  \end{lemma}
  \begin{proof}
    \pause
    Dividing $p(x)$ with $(x-a)$, we get that
    \[
    p(x) = q'(x)(x-a) + r(x),
    \]
    where $r(x)$ is of degree at most $1-1=0$, i.e., $r(x)$ must be a
    constant; thus, we assume that $r(x)=c$.  Let's evaluate $p(a)$; note
    that $p(a)=c$, since
    \[
    p(a) = q'(a)(a-a) + c = 0 + c = c.
    \]
    However we know that $a$ is a root of $p(x)$, i.e., $p(a)=0$.
    Therefore $c=0$, or $r(x)=0$.  Thus, the lemma follows.
  \end{proof}
\end{frame}

\begin{frame}
  \begin{lemma}
    If $p(x)$ is a polynomial of degree $d$ with $d$ distinct roots
    $a_1,a_2,\ldots,a_d$, $p(x)$ can be written as
    $c(x-a_1)(x-a_2)\cdots(x-a_d)$.
  \end{lemma}
  \begin{proof}
    {\small
      \pause
      We prove by induction on $d$. \pause
      
      {\bf Base case:}
      \pause

      {\bf Inductive step:} \pause Assume that $p(x)$ is a polynomial
      of degree $d+1$ with distinct roots $a_1,\ldots,a_d,a_{d+1}$.
      \pause Since $a_{d+1}$ is $p(x)$'s root, we can divide $p(x)$
      with $(x-a_{d+1})$ and get that
      \[
      p(x) = (x-a_{d+1})q(x),
      \]
      where \pause $q(x)$ is a polynomial of degree $d$ with $d$
      distinct roots $a_1,\ldots,a_d$. 
      \vspace{0.5in}
    }
  \end{proof}
\end{frame}

\begin{frame}
  \frametitle{Property 1}
\end{frame}

\begin{frame}
  \frametitle{Polynomials over a finite field $GF(p)$}
\end{frame}

\begin{frame}
  \frametitle{Examples - evaluation}

  Suppose that we work over $GF(m)$ where $m=11$.  Let $p(x) = 4\cdot
  x^2 + 5\cdot x + 3$.  We have

  {\small
  \begin{tabular}{c|c|c}
    $x$ & $p(x)$ & $p(x)\bmod m$ \\
    \hline
    $0$ & $3$ & $3$ \\
    $1$ & $12$ & $1$ \\
    $2$ & $29$ & $7$ \\
    $3$ & $54$ & $10$ \\
    $4$ & $87$ & $10$ \\
    $5$ & $128$ & $7$ \\
    $6$ & $177$ & $1$ \\
    $7$ & $234$ & $3$ \\
    $8$ & $299$ & $2$ \\
    $9$ & $372$ & $9$ \\
    $10$ & $453$ & $2$ \\
    $11$ & $542$ & $3$ \\
  \end{tabular}
  }
\end{frame}

\begin{frame}
  \frametitle{Examples - interpolation}

  Let $m=11$. Suppose that $p(x)$ is a polynomial over $GF(m)$ of
  degree $2$ passing through $(2,7),(4,10),$ and $(7,3)$.  Find $p(x)$.

  Let
  \begin{itemize}
  \item
    $\Delta_1(x)=\frac{(x-4)(x-7)}{(2-4)(2-7)}
    =\frac{x^2-11x+28}{(-2)\cdot(-5)}
    =\frac{x^2+6}{10}
    =10x^2+5$
  \item
    $\Delta_2(x)=\frac{(x-2)(x-7)}{(4-2)(4-7)}
    =\frac{x^2-9x+14}{2\cdot(-3)}
    =\frac{x^2+2x+3}{5}
    =9x^2+7x+5$
  \item
    $\Delta_2(x)=\frac{(x-2)(x-4)}{(7-2)(7-4)}
    =\frac{x^2-6x+8}{5\cdot 3}
    =\frac{x^2+5x+8}{4}
    =3x^2+4x+2$
  \end{itemize}

  Thus,
  \begin{eqnarray*}
    p(x) &=&  7\Delta_1(x) + 10\Delta_2(x) + 3\Delta_3(x) \\
    &=& (70x^2 + 35) + (90x^2 + 70x+50) + (9x^2+12x+6) \\
    &=& 4x^2+5x+3
  \end{eqnarray*}
\end{frame}

\begin{frame}
  \frametitle{How many?}
\end{frame}

\begin{frame}
  Two ways of specifying a polynomial $p(x)$ of degree $d$:
  \begin{itemize}
  \item Specify its coefficients $a_0,a_1,\ldots,a_d$, i.e., the
    polynomial is
    \[
    p(x) = a_d x^d + \ldots a_1 x + a_0.
    \]
    \pause
  \item Specify $d+1$ points, i.e.,
    $(x_1,y_1),(x_2,y_2),\ldots,(x_{d+1},y_{d+1})$, where all $x_i$
    are distinct.  There is a {\em unique} polynomial $p(x)$ of degree
    at most $d$ that passes through these points (from Property 2).
  \end{itemize}
\end{frame}

\begin{frame}
  For polynomials of degree at most $d$ over $GF(m)$, if you specify
  $q$ points, there are:
  
  \begin{tabular}{c|c}
    $q$ & numbers of polynomials \\
    \hline
    $d+1$ & $1$ \\
    $d$ & $m$ \\
    $d-1$ & $m^2$ \\
    $d-2$ & $m^3$ \\
    $\vdots$ & $\vdots$ \\
    $1$ & $m^d$ \\
    $0$ & $m^{d+1}$
  \end{tabular}
\end{frame}

\begin{frame}
  \frametitle{Secret sharing scheme - settings}
  \pause
  \begin{itemize}
  \item There are $n$ people, a secret $s$, and an integer $k$.
  \item We want to ``distribute'' the secret in such a way that any
    set of $k-1$ people cannot know anything about $s$, but any set of
    $k$ people can reconstruct $s$.
  \end{itemize}
\end{frame}

\begin{frame}
  \frametitle{Secret sharing scheme}
  \pause
  \begin{itemize}
  \item Pick $m$ to be larger than $n$ and $s$. (Much larger than $s$,
    i.e., $m >>> s$.)
  \item Pick a random polynomial of degree $k-1$ such that $P(0)=s$.
  \item Give $P(i)$ to person $i$, for $1\leq i\leq n$.
  \item Correctness: for any set of $k$ people,
    \pause

  \item Correctness: for any set of $k-1$ people, how many possible
    candidate secrets compatible with the information these people
    have?
  \end{itemize}
\end{frame}

\begin{frame}
  \frametitle{A more complex secret sharing scheme}

  Suppose that a company has 3 VPs and 5 senior members.  You want to
  distribute a secret such that (1) any 2 VPs can obtain the secret or
  (2) a single VP with 3 senior members can also obtain the secret.
  How can you do that?
  
  \vspace{2.5in}
\end{frame}

\begin{frame}
  \frametitle{Sending a message}

  Suppose that you want to send a message {\tt 1,2,1,1,3,4,4,10} over
  the internet.

  \pause
  
  Since the internet does not maintain the ordering (if you send with
  UDP), you have to maintain the ``ordering'' youself, e.g., you can
  add the message indices, i.e.,
  \pause

  {\bf Lossy internet:}
  
  \vspace{1.5in}
\end{frame}

\begin{frame}
  \frametitle{Erasure codes}

  Suppose that we want to send a message $m_1,m_2,\ldots,m_n$ where
  $m_i\leq p-1$ for some prime $p$. 

  However, we know that our communication channel is lossy, i.e., some
  messages can be {\em dropped}.  How can we send this message?

  \vspace{2.5in}
\end{frame}

\begin{frame}
  \frametitle{Two ways of encoding}

  Suppose that we want to send a message $m_1,m_2,\ldots,m_n$ where
  $m_i\leq p-1$ for some prime $p$.  We want to tolerate up to $k$
  missing messages.

  We use a polynomial of degree \pause $n-1$ and generate $n+k$ points.

  How can we obtain the polynomial $P(x)$?

  \begin{itemize}
  \item We can let the message be the coefficients, i.e., let
    \[
    P(x) = m_n\cdot x^{n-1} + m_{n-1}\cdot x^{n-2} + \cdots + m_2\cdot x + m_1.
    \]
    \pause
  \item We can try to obtain a degree-$(n-1)$ polynomial $P(x)$ such that
    \[
    P(0)=m_1, \
    P(1)=m_2, \ \ldots 
    P(n-2)=m_{n-1}, \ 
    P(n-1)=m_{n}.
    \]
  \end{itemize}
\end{frame}
