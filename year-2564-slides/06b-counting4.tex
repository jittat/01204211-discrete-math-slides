\newcommand{\lecturetitle}[1]{
  \title{01204211 Discrete Mathematics \\ #1}
  \author{Jittat Fakcharoenphol}
  \frame{\titlepage}
}
\newcommand{\Mod}{\,\bmod\,}

\lecturetitle{Lecture 6b: Counting 4}

\begin{frame}\frametitle{Quick practice}
  \textcolor{blue}{Theorem:} For a non-empty set, the number of subsets whose sizes are odd equals the number of subsets whose sizes are even. \pause
  I.e., for $n>0$,
  \[
  \binom{n}{0} + \binom{n}{2} + \binom{n}{4} + \cdots =
  \binom{n}{1} + \binom{n}{3} + \binom{n}{5} + \cdots.
  \]

  \pause

  \vspace{2in}

\end{frame}

\begin{frame}\frametitle{Quick questions (1)}
  \begin{tcolorbox}
    There are 40 students in the classroom.  There are 35 students who
    like Naruto, 10 students who like Bleach, and 7 students who like
    both of them.  How many students in this classroom who do not like
    either Bleach or Naruto?
  \end{tcolorbox}

  \vspace{2in}
\end{frame}

\begin{frame}\frametitle{Quick questions (2)}
  \begin{tcolorbox}
    There are 35 students in the classroom.  There are 25 students who
    like Naruto, 15 students who like Bleach, 12 students who like One
    Piece.  There are 10 students who like both Naruto and Bleach, 7
    students who like both Bleach and One Piece, and 9 students who
    like both Naruto and One Piece.  There are 5 students who like all
    of them.

    How many students in this classroom who do not like any of Bleach,
    Naruto, or One Piece?
  \end{tcolorbox}

  \vspace{1.2in}
\end{frame}

\begin{frame}\frametitle{Is this correct?}
  The answer from the previous quick question is
  \[ 35 - (25 + 15 + 12 - 10 - 7 - 9 + 5) = 4.\]

  Is this correct?  Why?

  \pause

  \vspace{0.2in}

  Let's try to argue that this answer is, in fact, correct and try to
  find general answers to this kind of counting questions.
\end{frame}

\begin{frame}\frametitle{Let's look at an individual student (1)}
  {\small
  \begin{tabular}{c|c||c|c|c|c|c|c|c|c||c}
    & &  & N & B & O & NB & BO & NO & NBO & \\
    \hline
    & & $35$ & $-25$ & $-15$ & $-12$ & $+10$ & $+7$ & $+9$ & $-5$ & $4$ \\
    \hline
    Alfred & N,O & \pause * & * & & * & & & * & & \\
    Bobby & B & \pause * & & * & & & & & & \\
    Cathy & B,O & \pause * & & * & * & & * & & & \\
    Dave & N,B,O & \pause * & * & * & * & * & * & * & * & \\
    Eddy & - & \pause * & &  & & & & & & \\
    $\vdots$ & $\vdots$ & & & & & & & & & \\
  \end{tabular}
  }
\end{frame}

\begin{frame}\frametitle{Let's look at an individual student (2)}
  {\small
  \begin{tabular}{c|c||c|c|c|c|c|c|c|c||c}
    & &  & N & B & O & NB & BO & NO & NBO & \\
    \hline
    & & $35$ & $-25$ & $-15$ & $-12$ & $+10$ & $+7$ & $+9$ & $-5$ & $4$ \\
    \hline
    Alfred & N,O & 1 & -1 & & -1 & & & +1 & & 0 \\
    Bobby & B & 1 & & -1 & & & & & & 0 \\
    Cathy & B,O & 1 & & -1 & -1 & & +1 & & & 0 \\
    Dave & N,B,O & 1 & -1 & -1 & -1 & +1 & +1 & +1 & -1 & 0 \\
    Eddy & - & 1 & &  & & & & & & 1 \\
    $\vdots$ & $\vdots$ & & & & & & & & & \\
  \end{tabular}
  }
\end{frame}

\begin{frame}\frametitle{Let's see how each one is counted}
  Alfred (N,O): \pause $$1 - {2\choose 1} + {2\choose 2} = \pause 1 - 2 + 1 = 0$$

  Bobby (B): \pause $$1 - {1\choose 1} = \pause 1 - 1 = 0$$

  Dave (N,B,O): \pause $$1 - {3\choose 1} + {3\choose 2} - {3\choose 3} = \pause 1 - 3 + 3 - 1 = 0$$

  \pause

  Do you see any patterns here?
  \pause
  How about $$1 - {5\choose 1} + {5\choose 2} - {5\choose 3} + {5\choose 4} - {5\choose 5}\ \ ?$$
\end{frame}

\begin{frame}\frametitle{Underlying structures}
  Let's write $1$ as ${5\choose 0}$.  Also, let's separate plus terms
  and minus terms:

  $${5\choose 0} + {5\choose 2} + {5\choose 4}
  \ \ \ \ \ \heartsuit \ \ \ \ \
  {5\choose 1} + {5\choose 3} + {5\choose 5}$$

  \pause
  \vspace{0.1in}
  Note that the left terms are the number of even subsets and the
  right terms are the number of odd subsets.  Do you recall what we have done at the beginning of this lecture?
  \pause We have proved this:

  \begin{tcolorbox}
    {\bf Theorem:} The number of even subsets is equal to the number
    of odd subsets.
  \end{tcolorbox}

  This theorem also shows that our calculation technique is correct.
  This technique is usually called the {\bf Inclusion-Exclusion
    principle}.
\end{frame}
