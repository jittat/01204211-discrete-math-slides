\newcommand{\lecturetitle}[1]{
  \title{01204211 Discrete Mathematics \\ #1}
  \author{Jittat Fakcharoenphol}
  \frame{\titlepage}
}
\newcommand{\Mod}{\,\bmod\,}

\lecturetitle{Lecture 6c: The pigeonhole principle and the birthday problem}

\begin{frame}\frametitle{The sock problem}
  \begin{tcolorbox}
    I have $n$ pairs of socks.  Each pair is different from the other
    pair.  How many socks do I have to pick out to be sure that I have
    at least one matching pair.
  \end{tcolorbox}
  \vspace{2in}
\end{frame}

\begin{frame}\frametitle{The Pigeonhole Principle}
  The answer of the previous question seems obvious.  But it appears
  to be very useful in numerous cases.  It is called {\bf the
    pigeonhole principle}.

  \begin{tcolorbox}[title=The pigeonhole principle]
    If we put $n+1$ objects into $n$ boxes, at least one box gets more
    than one objects.
  \end{tcolorbox}
\end{frame}

\begin{frame}\frametitle{Example}
  \begin{tcolorbox}
    Assume that nobody is taller than 250 cm.  In a group of 251
    people, there are at least two people whose heights differ by at
    most 1cm.
  \end{tcolorbox}
  \vspace{2in}
\end{frame}

\begin{frame}\frametitle{Students with the same birthday\footnote{This section follows section 2.5 in [LPV].}}
  \begin{itemize}
  \item It is quite often that you find people with the same birthday.
  \item Since there is at most 366 days in a year, the pigeonhole
    principle states that if you have 367 people is a room, there is
    at least one pair with the same birthday.
    \pause
  \item But that's the worst case scenario, as it is more common to
    find people with the same birthday.  (In the next class, we will
    try to see if there is a pair of students in the class with the
    same birthday.)
    \pause
  \item So, let's think about the probability that there are two
    students with the same birthday in a room with 40 students.
  \end{itemize}
\end{frame}

\begin{frame}\frametitle{A simple case}
  \begin{itemize}
  \item Let's start with $2$ people in the room.
    \pause
  \item {\small {\bf Notes:} While we have not defined properly what
    probabilities mean, we can count the number of all possibilities
    and the number of cases that we are interested in, and then
    calculate probability as the ratio between the two.  \pause E.g.,
    if there are $50$ possible outcomes and $30$ of them are the ones
    we are interested in, the probability is $0.6$. \pause (Note that
    we assume that every outcome is equally likely.)}  \pause
  \item How many possible birthdays can two people have? \pause
    \begin{itemize}
    \item Since each person has $366$ choices, and the first person
      and the second person can choose independently, the number is
      $366\cdot 366$.
    \end{itemize}
    \pause
  \item How many possible ways can they share the same birthday?
    \pause
    \begin{itemize}
    \item Since the first person has $366$ choices, and the second
      person has to choose the same day, there are only $366$ ways.
    \end{itemize}
    \pause
  \item Thus, the probability is $\frac{366}{366^2} = 0.0027$, very
    unlikely.
  \end{itemize}
\end{frame}

\begin{frame}\frametitle{3 people}
  \begin{itemize}
  \item Let's consider $3$ people.
    \pause
  \item How many possible birthdays can $3$ people have? \pause
    \begin{itemize}
    \item Since each person has $366$ choices, and each person can
      choose independently, the number is $366\cdot 366\cdot 366=366^3$.
    \end{itemize}
    \pause
  \item How many possible ways can at least two of them share the same
    birthday?  \pause
    \begin{itemize}
    \item There are many cases. \pause
    \item So let's think about the case when everyone do not share any
      birthdays.
    \item The first person has $366$ choices.  \pause The second one
      has $366 - 1 = 365$ choices.  \pause The third one has $366 - 2
      = 364$ choices. \pause Thus, the number of ways they do not
      share any birthdays is $366\cdot 365\cdot 364$. \pause
    \item Notice that this is the number of ordered subsets. \pause
    \end{itemize}
    \pause
  \item Thus, the probability that they do not share birthdays is
    $\frac{366\cdot 365\cdot 364}{366^3} = 0.9918$.  Thus the
    probability that two of them share a birthday is $1-0.9918 =
    0.0082$.
  \end{itemize}
\end{frame}

\begin{frame}\frametitle{40 people}
  \begin{itemize}
  \item Let's extend our previous argument to the case with $40$ people.
  \item How many possible birthdays can $40$ people have? \pause
    \begin{itemize}
    \item $366^{40}$.
    \end{itemize}
    \pause
  \item How many possible ways that they do not sure any birthdays? \pause
    \begin{itemize}
    \item This is the number of ordered subsets with $40$ elements of
      a $366$-set. \pause
    \item Thus, there are $366\cdot 365\cdot 364\cdots 327$ ways.
    \end{itemize}
    \pause
  \item Thus, the probability that they do not share birthdays is
    \[\frac{366\cdot 365\cdot 364\cdots 327}{366^{40}}.\]
  \item \pause Umm... how small is it? \pause
  \item Again you can use a computer to compute the exact value of
    this quantity.  For example, you may want to use Wolfram Alpha.
    \pause

  \item Anyway, we will try to estimate it using basic mathematical
    tools.
  \end{itemize}
\end{frame}

\begin{frame}\frametitle{General case: $n$ days $k$ people}
  \begin{itemize}
  \item Let's continue on the general case.  When we have $k$ people
    and a year contains $n$ days, the probability that no two people
    share the same birthday is
    \[\frac{n\cdot (n-1)\cdot (n-2)\cdots (n-k+1)}{n^k}.\]
  \item \pause If this number is very close to 0, then it is very
    unlikely that no two people share the same birthday, i.e., it is
    very likely that there exists two people with the same birthday.
  \end{itemize}
\end{frame}

\begin{frame}\frametitle{A few tweaks}
  \begin{itemize}
  \item Dealing with small numbers is sometimes troublesome. (The
    reason will be more apparent later when we start introducing the
    tools.)  So let's consider the reciprocal instead:
    \[\frac{n^k}{n\cdot (n-1)\cdot (n-2)\cdots (n-k+1)}.\]
    \pause
  \item The top term looks easy to deal with; the bottom one does not.
    Let's break up the product:
    \[\left(\frac{n}{n}\right)\cdot
    \left(\frac{n}{n-1}\right)\cdot\left(\frac{n}{n-2}\right)
    \cdots\left(\frac{n}{n-k+1}\right).\]
    \pause
  \item If you look closely at this product, you can see that each
    term is at least one.  In the beginning, the terms are very close
    to one and they get larger at the end.
  \end{itemize}
\end{frame}

\begin{frame}\frametitle{The logarithms}
  \begin{itemize}
  \item There is a nice tool that you can turn multiplications to
    additions: {\bf logarithms.}  So let's try to take the logarithms; \pause
    we get
    \[\ln\left(\left(\frac{n}{n}\right)\cdot
    \left(\frac{n}{n-1}\right)\cdot\left(\frac{n}{n-2}\right)
    \cdots\left(\frac{n}{n-k+1}\right)\right) \qquad \qquad \qquad \qquad \]
    \[ = \ln\left(\frac{n}{n}\right)+
    \ln\left(\frac{n}{n-1}\right)+\ln\left(\frac{n}{n-2}\right) +
    \cdots+\ln\left(\frac{n}{n-k+1}\right).\]
    \pause
  \item The terms do not look that much better.  But there's a nice
    fact about the natural logarithms.
  \end{itemize}
\end{frame}

\begin{frame}\frametitle{$\ln x$: the upper bound}
  \begin{tcolorbox}
    {\bf Fact:}
    \[\ln x \leq x - 1\]
  \end{tcolorbox}
  This fact can be proved with elementary calculus.  But it is fairly clear if you plot the functions $\ln x$ and $x-1$.
  \vspace{1.2in}
\end{frame}

\begin{frame}\frametitle{$\ln x$: the lower bound}
  We know that
  \[ \ln x \leq x - 1 \]
  
  If we use the fact that $\ln\frac{1}{x}=-\ln x$, we can obtain the
  lower bound.
  \[ \ln x = -\ln\frac{1}{x}\geq -\left(\frac{1}{x}-1\right)=\frac{x-1}{x}. \]

  \pause
  Let's conclude by stating the lemma:
  \begin{tcolorbox}
    \begin{lemma}
      \[\frac{x-1}{x}\leq \ln x\leq x-1.\]
    \end{lemma}
  \end{tcolorbox}
\end{frame}

\begin{frame}\frametitle{The lower bound }
  Let's look at each term in the sum: $\ln\left(\frac{n}{n-j}\right)$.
  Using the lower bound in Lemma 1, we get that
  \[
  \ln\left(\frac{n}{n-j}\right)
  \geq \frac{\frac{n}{n-j}-1}{\frac{n}{n-j}}
  = \frac{\frac{n-n+j}{n-j}}{\frac{n}{n-j}}
  = \frac{j}{n}.
  \]
  \pause
  Thus,
  \[
  \ln\left(\left(\frac{n}{n}\right)\cdot
  \left(\frac{n}{n-1}\right)\cdot\left(\frac{n}{n-2}\right)
  \cdots\left(\frac{n}{n-k+1}\right)\right)
  \qquad\qquad\qquad\qquad\qquad\qquad
  \]
  \begin{eqnarray*}
  &=& \ln\left(\frac{n}{n}\right)+
  \ln\left(\frac{n}{n-1}\right)+\ln\left(\frac{n}{n-2}\right) +
  \cdots+\ln\left(\frac{n}{n-k+1}\right)\\
  &\geq& \frac{0}{n}+\frac{1}{n}+\frac{2}{n}+\cdots+\frac{k-1}{n}\\
  &=& \frac{1}{n}\left(1+2+\cdots+(k-1)\right) = \frac{k(k-1)}{2n}.
  \end{eqnarray*}
\end{frame}

\begin{frame}\frametitle{The upper bound }
  Again, let's look at each term in the sum:
  $\ln\left(\frac{n}{n-j}\right)$.  Using the upper bound in Lemma 1,
  we get that
  \[
  \ln\left(\frac{n}{n-j}\right)
  \leq \frac{n}{n-j}-1=\frac{j}{n-j}.
  \]
  \pause
  Thus,
  \[
  \ln\left(\left(\frac{n}{n}\right)\cdot
  \left(\frac{n}{n-1}\right)\cdot\left(\frac{n}{n-2}\right)
  \cdots\left(\frac{n}{n-k+1}\right)\right)
  \qquad\qquad\qquad\qquad\qquad\qquad
  \]
  \begin{eqnarray*}
  &\leq& \frac{0}{n-0}+\frac{1}{n-1}+\frac{2}{n-2}+\cdots+\frac{k-1}{n-k+1}\\
  &\leq& \frac{0}{n-k+1}+\frac{1}{n-k+1}+\frac{2}{n-k+1}+\cdots+\frac{k-1}{n-k+1}\\
  &=& \frac{1}{n-k+1}\left(1+2+\cdots+(k-1)\right) = \frac{k(k-1)}{2(n-k+1)}.
  \end{eqnarray*}
\end{frame}

\begin{frame}\frametitle{Both}
  Using the derived upper and lower bounds, we get
  \[
  e^{\frac{k(k-1)}{2n}}\leq
  \frac{n^k}{n(n-1)(n-2)\cdots(n-k+1)}
  \leq e^{\frac{k(k-1)}{2(n-k+1)}}
  \]
  \pause
  Let's plug in $n=366$ and $k=40$:
  \[
  8.42\leq \frac{366^{40}}{366\cdot 365\cdots 327}\leq 10.86.
  \]
  \pause
  So the probability that we get no two people with the same birthday
  is between $1/8.42 \approx 0.118$ and $1/10.86 \approx 0.092$.  So
  we have high chance of finding two students with the same
  birthday. \pause This is pretty close as the actual value is
  $0.1094$.
\end{frame}
