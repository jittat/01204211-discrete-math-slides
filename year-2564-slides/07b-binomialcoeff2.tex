\newcommand{\lecturetitle}[1]{
  \title{01204211 Discrete Mathematics \\ #1}
  \author{Jittat Fakcharoenphol}
  \frame{\titlepage}
}
\newcommand{\Mod}{\,\bmod\,}

\lecturetitle{Lecture 7b: Binomial Coefficients (2)} 

\begin{frame}\frametitle{The binomial coefficients\footnote{This lecture mostly follows Chapter 3 of [LPV].}}
  There is a reason why the term $\binom{n}{k}$ is called the binomial
  coefficients.  In this lecture, we will discuss
  \begin{itemize}
  \item identities on binomial coefficients.
  \end{itemize}
\end{frame}

\begin{frame}\frametitle{Identities in the Triangle}
  \begin{tcolorbox}
    {\footnotesize
      \begin{tabular}{ccccccccccccccc}
        & & & & & & & 1 & & & & & & & \\
        & & & & & & 1 & & 1 & & & & & & \\
        & & & & & 1 & & 2 & & 1 & & & & & \\
        & & & & 1 & & 3 & & 3 & & 1 & & & & \\
        & & & 1 & & 4 & & 6 & & 4 & & 1 & & & \\
        & & 1 & & 5 & & 10 & & 10 & & 5 & & 1 & & \\
        & 1 & & 6 & & 15 & & 20 & & 15 & & 6 & & 1 & \\
        1 & & 7 & & 21 & & 35 & & 35 & & 21 & & 7 & & 1 \\
      \end{tabular}
    }
    \vspace{0.2in}
  \end{tcolorbox}
\end{frame}

\begin{frame}\frametitle{Odd and even subsets}
  \begin{tcolorbox}
    {\footnotesize
      \begin{tabular}{ccccccccccccccc}
        & & & & & & & 1 & & & & & & & \\
        & & & & & & 1 & & 1 & & & & & & \\
        & & & & & 1 & & 2 & & 1 & & & & & \\
        & & & & 1 & & 3 & & 3 & & 1 & & & & \\
        & & & 1 & & 4 & & 6 & & 4 & & 1 & & & \\
        & & 1 & & 5 & & 10 & & 10 & & 5 & & 1 & & \\
        & 1 & & 6 & & 15 & & 20 & & 15 & & 6 & & 1 & \\
        1 & & 7 & & 21 & & 35 & & 35 & & 21 & & 7 & & 1 \\
      \end{tabular}
    }
  \end{tcolorbox}

  Let's try to prove this identity with the Pascal's triangle
  \[
  {n\choose 0} - {n\choose 1} + {n\choose 2} +\cdots +(-1)^{n}{n\choose n} = 0.
  \]
\end{frame}

\begin{frame}\frametitle{A more formal proof}
  \begin{tcolorbox}
    \[
    {n\choose 0} - {n\choose 1} + {n\choose 2} +\cdots +(-1)^{n}{n\choose n} = 0.
    \]
  \end{tcolorbox}
  \vspace{2in}
\end{frame}

\begin{frame}\frametitle{The next experiment}
  \begin{tcolorbox}
    {\footnotesize
      \begin{tabular}{ccccccccccccccc}
        & & & & & & & 1 & & & & & & & \\
        & & & & & & 1 & & 1 & & & & & & \\
        & & & & & 1 & & 2 & & 1 & & & & & \\
        & & & & 1 & & 3 & & 3 & & 1 & & & & \\
        & & & 1 & & 4 & & 6 & & 4 & & 1 & & & \\
        & & 1 & & 5 & & 10 & & 10 & & 5 & & 1 & & \\
        & 1 & & 6 & & 15 & & 20 & & 15 & & 6 & & 1 & \\
        1 & & 7 & & 21 & & 35 & & 35 & & 21 & & 7 & & 1 \\
      \end{tabular}
    }
  \end{tcolorbox}

  Let's try to compute the sum of squares of numbers in each row.
  \begin{eqnarray*}
    1^2 &=& 1\\ \pause
    1^2 + 1^2 &=& 2 \\ \pause
    1^2 + 2^2 + 1^2 &=& 6 \\ \pause
    1^2 + 3^2 + 3^2 + 1^2 &=& 20 \\ \pause
    1^2 + 4^2 + 6^2 + 4^2 + 1^2 &=& 70 \\
  \end{eqnarray*}
\end{frame}

\begin{frame}
  \textcolor{blue}{Theorem:}
  \[
  \binom{n}{0}^2 + \binom{n}{1}^2 + \binom{n}{2}^2 + \cdots+ \binom{n}{n}^2
  = \binom{2n}{n}.
  \]
  \vspace{2.5in}
\end{frame}

\begin{frame}\frametitle{Another identity}
  \begin{tcolorbox}
    {\footnotesize
      \begin{tabular}{ccccccccccccccc}
        & & & & & & & 1 & & & & & & & \\
        & & & & & & 1 & & 1 & & & & & & \\
        & & & & & 1 & & 2 & & 1 & & & & & \\
        & & & & 1 & & 3 & & 3 & & 1 & & & & \\
        & & & 1 & & 4 & & 6 & & 4 & & 1 & & & \\
        & & 1 & & 5 & & 10 & & 10 & & 5 & & 1 & & \\
        & 1 & & 6 & & 15 & & 20 & & 15 & & 6 & & 1 & \\
        1 & & 7 & & 21 & & 35 & & 35 & & 21 & & 7 & & 1 \\
      \end{tabular}
    }
  \end{tcolorbox}
  \pause

  This suggests
  \[
  \binom{n}{0} + \binom{n+1}{1} + \binom{n+2}{2} + \cdots + \binom{n+k}{k} = \binom{n+k+1}{k}.
  \]
\end{frame}

\begin{frame}
  \textcolor{blue}{Theorem:}
  \[
  \binom{n}{0} + \binom{n+1}{1} + \binom{n+2}{2} + \cdots + \binom{n+k}{k} = \binom{n+k+1}{k}.
  \]
  \vspace{2.5in}
\end{frame}
