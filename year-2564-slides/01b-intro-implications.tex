\newcommand{\lecturetitle}[1]{
  \title{01204211 Discrete Mathematics \\ #1}
  \author{Jittat Fakcharoenphol}
  \frame{\titlepage}
}
\newcommand{\Mod}{\,\bmod\,}

\lecturetitle{Lecture 1b: Implications and equivalences}

\begin{frame}\frametitle{This lecture covers:}
  \begin{itemize}
  \item More connectives: implications and equivalences
  \end{itemize}
\end{frame}

\begin{frame}\frametitle{Review (1)}
  \begin{itemize}
  \item A {\em proposition} is a statement which is either {\bf true}
    or {\bf false}.
  \item We can use variables to stand for propositions, e.g., $P$ =
    ``today is Tuesday''.
  \item We can use connectives to combine variables to get
    propositional forms.
    \begin{itemize}
    \item {\bf Conjunction:} $P\wedge Q$ (``$P$ and $Q$''), 
    \item {\bf Disjunction:} $P\vee Q$ (``$P$ or $Q$''), and
    \item {\bf Negation:} $\neg P$ (``not $P$'') 
    \end{itemize}
  \end{itemize}
\end{frame}

\begin{frame}\frametitle{Review (2)}
  To represents values of propositional forms, we usually use truth tables.
  \begin{tcolorbox}[title=And/Or/Not]
    \begin{tabular}{|c|c||c|c|c|}
      \hline
      $P$ & $Q$ & $P\wedge Q$ & $P\vee Q$ & $\neg P$ \\
      \hline
      $T$ & $T$ & $T$ & $T$ & $F$ \\
      $T$ & $F$ & $F$ & $T$ & \\
      $F$ & $T$ & $F$ & $T$ & $T$ \\
      $F$ & $F$ & $F$ & $F$ & \\
      \hline
    \end{tabular}
  \end{tcolorbox}
\end{frame}

\begin{frame}\frametitle{Quick check 1}
  As we said before, the truth value of propositional forms may not
  depend on the values of its variables.  As you can see in this
  exercise.
  
  Use a truth table to find the values of (1) $P\wedge \neg P$ and (2)
  $P\vee\neg P$.
  \pause

  \begin{tcolorbox}[title=And/Or/Not]
    \begin{tabular}{|c|c||c|c|}
      \hline
      $P$ & $\neg P$ & $P\wedge\neg P$ & $P\vee\neg P$\\
      \hline
      $T$ & $F$ & $F$ & $T$ \\
      $F$ & $T$ & $F$ & $T$ \\
      \hline
    \end{tabular}
  \end{tcolorbox}
  \pause

  Note that $P\wedge\neg P$ is always false and $P\vee\neg P$ is always true.

  A propositional form which is always true regardless of the truth
  values of its variables is called a {\em tautology.}  On the other
  hand, a propositional form which is always false regardless of the
  truth values of its variables is called a {\em contradiction.}
\end{frame}

\begin{frame}\frametitle{Implications\footnote{Materials in this lecture are mostly from Berkeley CS70's lecture notes.}}
  Given $P$ and $Q$, an implication
  \[P\Rightarrow Q\]
  stands for ``if $P$, then $Q$''.  This is a very important
  propositional form.

  It states that ``when $P$ is true, $Q$ must be true''.  Let's try to
  fill in its truth table:

  \begin{tcolorbox}[title=Implications]
    \begin{tabular}{|c|c||c|}
      \hline
      $P$ & $Q$ & $P\Rightarrow Q$ \\
      \hline
      $T$ & $T$ & \pause $T$ \\
      $T$ & $F$ & \pause $F$ \\
      $F$ & $T$ & \pause $T$ \\
      $F$ & $F$ & \pause $T$ \\
      \hline
    \end{tabular}
  \end{tcolorbox}
\end{frame}

\begin{frame}\frametitle{What?}
  \begin{itemize}
  \item
    Yes, when $P$ is false, $P\Rightarrow Q$ is {\bf always true} no
    matter what truth value of $Q$ is. 

  \item We say that in this case, the statement $P\Rightarrow Q$ is
    {\em vacuously true.}
    \pause

  \item
    You might feel a bit uncomfortable about this, because in most
    natural languages, when we say that if $P$, then $Q$ we sometimes
    mean something more than that in the logical expression
    ``$P\Rightarrow Q$.''
  \end{itemize}
\end{frame}

  
\begin{frame}\frametitle{One explanation}
  \begin{itemize}
  \item
    But let's look closely at what it means when we say that:

    \begin{tcolorbox}
      if $P$ is true, $Q$ must be true.
    \end{tcolorbox}

  \item
    Note that this statement does not say anything about the case when
    $P$ is false, i.e., it only considers the case when $P$ is true.
    \pause
    
  \item
    Therefore, having that $P\Rightarrow Q$ is true is OK with the
    case that (1) $Q$ is false when $P$ is false, and (2) $Q$ is true
    when $P$ is false.
    \pause
    
  \item
    This is an example when mathematical language is ``stricter'' than
    natural language.
  \end{itemize}
\end{frame}

\begin{frame}\frametitle{Noticing if-then}
  We can write ``if $P$, then $Q$'' for $P\Rightarrow Q$, but there
  are other ways to say this. E.g., we can write (1) $Q$ if $P$, (2) $P$
  only if $Q$, or (3) when $P$, then $Q$.

  \pause

  \begin{tcolorbox}[title=Quick check 2]
    For each of these statements, define
    propositional variables representing each proposition inside the
    statement and write the proposition form of the statement.
    \begin{itemize}
    \item If you do not have enough sleep, you will feel dizzy during class.
    \item If you eat a lot and you do not have enough exercise, you will
      get fat.
    \item You can get A from this course, only if you work fairly hard.
    \end{itemize}
  \end{tcolorbox}
  
\end{frame}

\begin{frame}\frametitle{Only-if}
  Let $P$ be ``you get A from this course.''

  Let $Q$ be ``you work fairly hard.''
  
  Let $R$ be ``You can get A from this course, only if you work fairly hard.''

  Let's think about the truth values of $R$.
  
  \begin{tcolorbox}[title=Only if you work fairly hard.]
    \begin{tabular}{|c|c||c|}
      \hline
      $P$ & $Q$ & $R$ \\
      \hline
      $T$ & $T$ & \\
      $T$ & $F$ & \\
      $F$ & $T$ & \\
      $F$ & $F$ & \\
      \hline
    \end{tabular}
  \end{tcolorbox}
  \pause

  Thus, $R$ should be logically equivalent to $P\Rightarrow Q$.  (We
  write $R\equiv P\Rightarrow Q$ in this case.)
\end{frame}

\begin{frame}\frametitle{If and only if: ($\Leftrightarrow$)}
  Given $P$ and $Q$, we denote by
  \[ P\Leftrightarrow Q \]
  the statement ``$P$ if and only if $Q$.''
  \pause
  It is logically equivalent to
  \[ (P\Leftarrow Q)\wedge (P\Rightarrow Q), \]
  i.e., $P\Leftrightarrow Q \equiv (P\Leftarrow Q)\wedge (P\Rightarrow Q)$.

  Let's fill in its truth table.
  \begin{tcolorbox}
    \begin{tabular}{|c|c||c|c|c|}
      \hline
      $P$ & $Q$ & $P\Rightarrow Q$ & $P\Leftarrow Q$ & $P\Leftrightarrow Q$ \\
      \hline
      $T$ & $T$ & & & \\
      $T$ & $F$ & & & \\
      $F$ & $T$ & & & \\
      $F$ & $F$ & & & \\
      \hline
    \end{tabular}
  \end{tcolorbox}
\end{frame}

\begin{frame}\frametitle{An implication and its friends}
  When you have two propositions
  \begin{itemize}
  \item $P$ = ``I own a cell phone'', and
  \item $Q$ = ``I bring a cell phone to class''.
  \end{itemize}
  We have
  \begin{itemize}
  \item an implication $P\Rightarrow Q$ $\equiv$ \\ ``If I own a cell phone,
    I'll bring it to class'',
  \item its {\bf converse} $Q\Rightarrow P$ $\equiv$ \\ ``If I bring a cell phone
    to class, I own it'', and
  \item its {\bf contrapositive} $\neg Q\Rightarrow\neg P$ $\equiv$ \\ ``If I do not
    bring a cell phone to class, I do not own one''.
  \end{itemize}
\end{frame}

\begin{frame}\frametitle{Quick check 3}
  Let's consider the following truth table:
  \begin{tcolorbox}
    \begin{tabular}{|c|c||c|c|c|}
      \hline
      $P$ & $Q$ & $P\Rightarrow Q$ & $Q\Rightarrow P$ & $\neg Q \Rightarrow \neg P$ \\
      \hline
      $T$ & $T$ & & & \\
      $T$ & $F$ & & & \\
      $F$ & $T$ & & & \\
      $F$ & $F$ & & & \\
      \hline
    \end{tabular}
  \end{tcolorbox}
  \pause
  Do you notice any equivalence?
  \pause

  Right, $P\Rightarrow Q\equiv \neg Q\Rightarrow\neg P$.
\end{frame}

