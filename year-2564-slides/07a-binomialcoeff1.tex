\newcommand{\lecturetitle}[1]{
  \title{01204211 Discrete Mathematics \\ #1}
  \author{Jittat Fakcharoenphol}
  \frame{\titlepage}
}
\newcommand{\Mod}{\,\bmod\,}

\lecturetitle{Lecture 7a: Binomial Coefficients (1)} 

\begin{frame}\frametitle{The binomial coefficients\footnote{This lecture mostly follows Chapter 3 of [LPV].}}
  There is a reason why the term $\binom{n}{k}$ is called the binomial
  coefficients.  In this lecture, we will discuss
  \begin{itemize}
  \item the Pascal's triangle, 
  \item the binomial theorem
  \end{itemize}
\end{frame}

\begin{frame}\frametitle{The equation}
  Last time we proved that, for $n,k>0$,
  \[\binom{n}{k} = \binom{n-1}{k-1} + \binom{n-1}{k}.\]
  \pause

  While we can prove this equation algebraically using definitions of
  binomial coefficients, proving the fact by describing the process of
  choosing $k$-subsets reveals interesting insights.  This equation
  also hints us how to compute the value of $\binom{n}{k}$ using
  values of $\binom{n-1}{\cdot}$'s.

  \pause
  So, let's try to do it.
\end{frame}

\begin{frame}\frametitle{The table}
  We shall use the fact that $\binom{n}{0}=1$ and $\binom{n}{k} =
  \binom{n-1}{k-1} + \binom{n-1}{k}$ to fill in the following table.

  \begin{tabular}{|r|c|c|c|c|c|c|c|}
    \hline
    $n$ & 0 & 1 & 2 & 3 & 4 & 5 & 6 \\ 
    \hline
    $0$ & 1 &&&&&&\\
    \hline
    $1$ & 1 & 1 &&&&&\\
    \hline
    $2$ & 1 & \pause 2 & 1 &&&&\\
    \hline
    \pause
    $3$ & 1 & \pause 3 & 3 & 1 &&&\\
    \hline
    \pause
    $4$ & 1 & \pause 4 & 6 & 4 & 1 &&\\
    \hline
    \pause
    $5$ & 1 & \pause 5 & 10 & 10 & 5 & 1 &\\
    \hline
    \pause
    $6$ & 1 & \pause 6 & 15 & 20 & 15 & 6 & 1 \\
    \hline
  \end{tabular}
  
  \vspace{0.1in}

  \pause You can note that the table is left-right symmetric.  This is
  true because of the fact that $\binom{n}{k} = \binom{n}{n-k}$.
\end{frame}

\begin{frame}\frametitle{The Triangle}
  If we move the numbers in the table slightly to the right, the table
  becomes the Pascal's triangle.
  \pause

  \vspace{0.1in}

  \begin{tcolorbox}
  \begin{tabular}{ccccccccccccc}
    & & & & & & 1 & & & & & & \\
    & & & & & 1 & & 1 & & & & & \\
    & & & & 1 & & 2 & & 1 & & & & \\
    & & & 1 & & 3 & & 3 & & 1 & & & \\
    & & 1 & & 4 & & 6 & & 4 & & 1 & & \\
    & 1 & & 5 & & 10 & & 10 & & 5 & & 1 & \\
    1 & & 6 & & 15 & & 20 & & 15 & & 6 & & 1 \\
    & $\vdots$ & & $\vdots$ & & & & & & & & $\vdots$ & \\
  \end{tabular}
  \end{tcolorbox}

  \vspace{0.1in}
  
  The table and the binomial coefficients have many other interesting
  properties.
\end{frame}

\begin{frame}\frametitle{Polynomial expansions}
  Let's start by looking at polynomial of the form $(x+y)^n$.  Let's
  start with small values of $n$:
  \begin{itemize}
  \item $(x+y)^1=x+y$
  \item $(x+y)^2 = \pause x^2 + 2\cdot xy + y^2$\\
  \item \pause $(x+y)^3 = \pause x^3 + 3\cdot x^2y + 3\cdot xy^2 + y^3$\\
  \item \pause $(x+y)^4 = \pause x^4 + 4\cdot x^3y + 6\cdot x^2y^2 + 4\cdot xy^3 + y^4$.
  \end{itemize}
  
  \vspace{0.1in}
  Let's focus on the coefficient of each term.  You may notice that
  terms $x^n$ and $y^n$ always have 1 as their coefficients.  {\em Why
    is that?} \pause

  Let's look further at the coefficients of terms $x^{n-1}y$.  Do you
  see any pattern in their coefficients?  {\em Can you explain why?}
\end{frame}

\begin{frame}\frametitle{Another way to look at it}
  Let's take a look at $(x+y)^4$ again.  It is

  \[ (x+y)(x+y)(x+y)(x+y). \]

  \begin{itemize}
  \item How do we get $x^4$ in the expansion?  \pause For every
    factory, you have to pick $x$.
  \item How do we get $x^3y$ in the expansion? \pause Out of the 4
    factors, you have to pick $y$ in one of the factor (or you have to
    pick $x$ in 3 of the factors).  \pause Thus there are
    $\binom{4}{3}=\binom{4}{1}$ ways to do so.
  \end{itemize}
\end{frame}

\begin{frame}\frametitle{The binomial theorem}
  \begin{tcolorbox}
    \textcolor{blue}{Theorem:} If you expand $(x+y)^n$, the
    coefficient of the term $x^ky^{n-k}$ is $\binom{n}{k}$.
  \end{tcolorbox}
  That is,
  \[ (x+y)^n = \sum_{k=0}^n \binom{n}{k} x^ky^{n-k} = \]
  \[ \binom{n}{n} x^n + \binom{n}{n-1} x^{n-1}y^1 + \binom{n}{n-2} x^{n-2}y^2 + \cdots + \binom{n}{1}xy^{n-1} + \binom{n}{0} y^n.\]
\end{frame}

\begin{frame}\frametitle{Additional applications of the binomial theorem}
  The binomial theorem can be used to prove various identities
  regarding the binomial coefficients.  For example, if we let $x=1$
  and $y=1$, we get that
  \[(1+1)^n=2^n=\binom{n}{0}+\binom{n}{1}+\cdots+\binom{n}{n-1}+\binom{n}{n}.\]

  \pause

  \vspace{0.2in}

  \begin{tcolorbox}
    {\bf Quick check.}  Can you prove that
    \[\binom{n}{0} - \binom{n}{1} + \binom{n}{2} - \binom{n}{3} + \cdots = 0.\]

    {\em Note that this statements says that the number of odd subsets
      equals the number of even subsets.}
  \end{tcolorbox}
\end{frame}
