\newcommand{\lecturetitle}[1]{
  \title{01204211 Discrete Mathematics \\ #1}
  \author{Jittat Fakcharoenphol}
  \frame{\titlepage}
}
\newcommand{\Mod}{\,\bmod\,}

\lecturetitle{Lecture 4a: Mathematical Induction 1}

\begin{frame}\frametitle{Mathematical Induction}
  \begin{itemize}
  \item In this lecture, we will focus on how to prove properties on natural numbers. \pause
  \item For example, we may want to prove that for any integer $n\geq 1$,
    \[ \sum_{i=1}^n i = n(n+1)/2, \]
    \pause
    or for any integer $n\geq 1$,
    \[ \sum_{i=1}^n i^2 = \frac{n}{6}(n+1)(2n+1),\]
    \pause
    or ``We can pay any integer amount $x\geq 4$ baht with 2-baht
    coins and 5-baht coins.''
  \end{itemize}
\end{frame}

\begin{frame}\frametitle{A review of the summation notation (by examples)}
  \begin{itemize}
  \item $\displaystyle\sum_{i=1}^{10} i =$ \pause $1+2+\cdots+10.$ \\ \pause (reads ``sum from
    $i=1$ to $10$ of $i$'' or ``sum of $i$ from $i=1$ to $10$'') \pause
  \item $\displaystyle\sum_{i=7}^{9} (i^2+i) =$ \pause $(7^2+7)+(8^2+8)+(9^2+9).$ \\ \pause (reads ``sum from $i=7$ to $9$ of $i^2 + i$'' or ``sum of $i^2+i$ from $i=7$ to $9$'') \pause
  \item The range of the index may be sets.  For example, let
    $A=\{1,2,4,15\}$, we have that $\displaystyle\sum_{i\in A} i^2 =$ \pause $1^2+2^2+4^2+15^2$.
    \pause
  \item What is $\sum_{i=5}^{2} i$? \pause Note that in this case, the
    range is empty.  This sum is called an {\bf empty sum}.  By
    convention, we define it to be zero.
  \end{itemize}
\end{frame}

\begin{frame}\frametitle{Informal arguments (1)}
  \begin{itemize}
  \item Let's try to check that $\sum_{i=1}^n i = n(n+1)/2$, for any
    integer $n\geq 1$, by experimentation.
  \item Try $n=1$: \pause LHS\footnote{LHS = left hand side}: $1$, \pause RHS\footnote{RHS = right hand side}: $1(1+1)/2 = 1$, \pause OK
  \item Try $n=2$: \pause LHS: $1+2=3$, \pause RHS: $2(2+1)/2 = 3$, \pause OK
  \item Try $n=3$: \pause LHS: $1+2+3=6$, \pause RHS: $3(3+1)/2 = 6$, \pause OK
  \item Try ... \pause
  \item With this trying-all approach, we can't actually prove this statement.
  \end{itemize}
\end{frame}

\begin{frame}\frametitle{Informal arguments (2)}
  \begin{itemize}
  \item Our goal is to show that $\sum_{i=1}^n i = n(n+1)/2$, for any
    integer $n\geq 1$.
  \item Try $n=2$: LHS: $1+2=3$, RHS: $2(2+1)/2 = 3$.
  \item Try $n=3$: LHS: $1+2+3$, RHS: $3(3+1)/2$ \pause
  \item If we compare these two lines, we can see that
    \begin{eqnarray*}
      1+2+3 &=& (1+2)+3 \\ \pause
      &=& 2(2+1)/2 + 3 \mbox{\ \ \ \ \ \ \ \ \ (*)} \\ \pause
      &=& 2(2+1)/2 + (2+1) \\ \pause
      &=& 2(2+1)/2 + 2\cdot(2+1)/2 \\ \pause
      &=& (2+2)(2+1)/2 = (3+1)(3)/2,
    \end{eqnarray*}
    which is equal to $3(3+1)/2$. \pause
  \item Line (*) is important here.  That is because we use the fact
    that the statement is true when $n=2$ there.
  \end{itemize}
\end{frame}

\begin{frame}\frametitle{Informal arguments (3)}
  \begin{itemize}
  \item Goal: show that $\sum_{i=1}^n i = n(n+1)/2$, for any integer
    $n\geq 1$.
  \item \textcolor{blue}{What we have just done?} \pause We show that
    the statement is true when $n=3$ if it is true when $n=2$. \pause
  \item Let's try to make a more general argument. \pause
  \item Assume that the statement is true for $n=k$.  I.e.,
    \[ \sum_{i=1}^k i = k(k+1)/2. \] \pause
  \item Can we show that, with this assumption, the statement is true
    for $n=k+1$?  I.e., can we show that
    \[ \sum_{i=1}^{k+1} i = (k+1)((k+1)+1)/2 ? \]
  \end{itemize}
\end{frame}

\begin{frame}\frametitle{Informal arguments (4)}
  Let's try... \\
  {\bf Assumption:} $\sum_{i=1}^k i = k(k+1)/2$. \\
  {\bf Goal:} $\sum_{i=1}^{k+1} i = (k+1)((k+1)+1)/2$. \\ \pause

  \begin{eqnarray*}
    \sum_{i=1}^{k+1} i &=& \left(\sum_{i=1}^k i\right) + (k+1) \\ \pause
    &=& k(k+1)/2 + (k+1) \\ \pause
    &=& k(k+1)/2 + 2\cdot(k+1)/2 \\ \pause
    &=& (k+2)(k+1)/2\\
    &=& (k+1)((k+1)+1)/2,
  \end{eqnarray*}
  as required.

\end{frame}

\begin{frame}\frametitle{Informal arguments (5)}
  We have all the ingredients required to prove this statement:
  \begin{tcolorbox}
    For integer $n\geq 1$, $\sum_{i=1}^n i = n\cdot(n+1)/2.$
  \end{tcolorbox}
  \pause

  Let $P(n)\equiv$ ``$\sum_{i=1}^n i = n\cdot(n+1)/2$''. \pause

  The statement we want to prove becomes:

  \begin{tcolorbox}
    For any natural number $n$, $P(n)$.
  \end{tcolorbox}
  \pause

  We have shown:
  \begin{enumerate}
  \item $P(1)$ (by experimentation)
  \item $P(k)\Rightarrow P(k+1)$ for any integer $k\geq 1$.
  \end{enumerate}
  \pause

  What do these two statements imply?
  
\end{frame}

\begin{frame}\frametitle{Informal arguments (6)}
  We have:
  \begin{enumerate}
  \item $P(1)$ (by experimentation)
  \item $P(k)\Rightarrow P(k+1)$ for any integer $k\geq 1$.
  \end{enumerate}
  What do these two statements imply?
  \pause

  \vspace{0.2in}
  
  $P(1)$ (1st statement itself) \\
  \pause $\Rightarrow P(2)$ (from 2nd statement, let $k=1$) \\
  \pause $\Rightarrow P(3)$ (from 2nd statement, let $k=2$) \\
  \pause $\Rightarrow P(4)$ (from 2nd statement, let $k=3$) \\
  \pause $\Rightarrow P(5)$
  \pause $\Rightarrow P(6)$
  \pause $\Rightarrow P(7)$
  \pause $\ldots$
  \pause

  \vspace{0.2in}

  Informally, these chain of reasoning will eventually reach any
  natural number $n$.  Therefore, we can conclude that $P(n)$ for any
  natural number $n$.

  \pause

  We have just shown the statement with mathematical induction.
  
\end{frame}

\begin{frame}\frametitle{Mathematical Induction}
  \begin{tcolorbox}
    Suppose that you want to prove that property $P(n)$ is true for
    every natural number $n$.\\
    
    Suppose that we can prove the following two facts:
    
    {\bf Base case:} $P(1)$ \\
    {\bf Inductive step:} For any $k\geq 1$, $P(k)\Rightarrow P(k+1)$ \\
    
    The {\bf Principle of Mathematical Induction} states that $P(n)$
    is true for every natural number $n$.
  \end{tcolorbox}

  The assumption $P(k)$ in the inductive step is usually referred to
  as {\bf the Induction Hypothesis}.
  
\end{frame}

\begin{frame}\frametitle{Let's re-write the proof again}
  \begin{theorem}
    For every natural number $n$, $\sum_{i=1}^n i = n(n+1)/2$
  \end{theorem}
  {\bf \textcolor{blue}{Proof:}}{\small
    We prove by induction.  The property that we want to prove $P(n)$
    is ``$\sum_{i=1}^n i = n(n+1)/2$.''

    {\bf Base case:} We can plug in $n=1$ to check that $P(1)$ is
    true: $1 = 1(1+1)/2$.

    {\bf Inductive step:} We assume that $P(k)$ is true for $k\geq 1$
    and show that $P(k+1)$ is true.

    Let's state the Induction Hypothesis $P(k)$:
    $ \sum_{i=1}^k i = k(k+1)/2.$

    Let's show $P(k+1)$.  We write
    $ \sum_{i=1}^{k+1} i = \left(\sum_{i=1}^k i\right) + (k+1) .$
    Using the Induction Hypothesis, we know that this is equal to
    \begin{eqnarray*}
      k(k+1)/2 + (k+1) &=& k(k+1)/2 + 2\cdot(k+1)\\
      &=& (k+2)(k+1)/2,
    \end{eqnarray*}
    which implies $P(k+1)$ as required.

    From the Principle of Mathematical Induction, this implies that
    $P(n)$ is true for every natural number $n$.
  }
\end{frame}
