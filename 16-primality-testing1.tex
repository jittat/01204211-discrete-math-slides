\newcommand{\lecturetitle}[1]{
  \title{01204211 Discrete Mathematics \\ #1}
  \author{Jittat Fakcharoenphol}
  \frame{\titlepage}
}
\newcommand{\Mod}{\,\bmod\,}

\lecturetitle{Lecture 16: Primality testing (1)} 

\begin{frame}\frametitle{Integers}
  This is the second major area that we shall study in this course:
  integers and their properties.  This area is called {\bf number
    theory.}  We will see many properties of integers and their
  applications that include data encoding techniques and data
  encryption.

  Most techniques depend on having large primes.  So, we start this
  area by focus on the primality testing algorithm.  This is also the
  first algorithm that we considered in this class as well.
\end{frame}

\begin{frame}\frametitle{Brute-force division}
  \begin{tcolorbox}
    {\bf FUNCTION} CheckPrime2($n$)\\
    1. $k\leftarrow 2$\\
    2. WHILE $k\leq\sqrt{n}$ DO\\
    3. \ \ IF $k$ divides $n$ THEN\\
    4. \ \ \ \ RETURN {\bf false}\\
    5. \ \ ENDIF\\
    6. \ \ $k\leftarrow k+1$\\
    7. ENDWHILE\\
    8. RETURN {\bf true}
  \end{tcolorbox}

  The WHILE loop in CheckPrime2 runs for at most $\sqrt{n}$ times.  This
  improves over the original algorithm that uses roughly $n$ rounds.
  While this is a big improvement, it is usually not good enough when
  we consider a typical usage of the algorithm that we need, where we
  need to check if a 1000-digit number is a prime.
\end{frame}

\begin{frame}\frametitle{Running time analysis: the $O$-notation}
  \begin{itemize}                  
  \item We shall study the running time of various algorithms in this
    section.  Since we will not be very precise, keeping tracks of all
    details, we will use the $O$-notation when we talk about the
    running time.
  \item
    We will informally use the notation.  When we say that function
    $f(n)$ is $O(g(n))$, we means that the growth of $f(n)$ is at most
    that of $g(n)$.  This means that the largest terms in $f(n)$ is
    not larger than that of $g(n)$.
  \item
    {\bf Examples:}
    \begin{itemize}
    \item $n^2 + 100n = O(n^2)$
    \item $\sqrt{n} + n = O(n)$
    \item $2^n + 10000 + 100000n^{10} = O(2^n)$
    \item $5\cdot n^3 + n^2 = O(n^3)$
    \item $10n^2 \times 7n + 12n^3\times 75n^2  = O(n^3) + O(n^5) = O(n^5)$
    \end{itemize}
  \item Note that it is also true that $n^2 = O(n^3)$.
  \end{itemize}
\end{frame}

\begin{frame}\frametitle{Polynomial running times}
  Let's discuss CheckPrime2's running time.

  We usually think about the running time as a function on the size of
  the input.  When we want to sort $n$ numbers, the size of the input
  is $n$.  However, when dealing with big numbers (i.e., those much
  larger than directly manipulatable in the CPU), you cannot
  manipulate them in constant time.  In this case, we usually count
  the number of bits as the size of the input.

  Since $n$ is the value of the input, to keep this integer in
  computer memory, you need at least $\log_2 n$ bits\footnote{We shall
    use logarithm base 2 in this part of the course.}.  This will be
  the size of the input that we shall consider.

  \vspace{0.1in}
  Let $m=O\log n$ be the number of bits of $n$.  The algorithm that
  runs in time $O(\sqrt{n})$, actually runs in time
  $O(\sqrt{2^m})=O(2^{m/2})$, an exponential running time.  This means
  that the algorithm does not scale very well, as the size of the
  input increases.
  
  In this case, we want a more efficient algorithm, i.e., it runs in
  time in the polynomial of $m$.
\end{frame}

\begin{frame}\frametitle{Running time: integer operations}
  When we have two $m$-bit integers $a$ and $b$ The running times for
  \begin{itemize}
  \item Addition and subtraction: $O(m)$, and
  \item Multiplication and division: $O(m^2)$.
  \end{itemize}

  \vspace{0.2in}

  {\bf Examples:}

  \vspace{0.1in}
  \begin{columns}
    \begin{column}{0.4\textwidth}
      1000 1110 1011 0110\\
      +\\
      0010 0111 0110 1111\\
      ------ \\
      1011 0110 0010 0111
    \end{column}
    \begin{column}{0.4\textwidth}
      \textcolor{white}{0000 0000} 1000 1110\\
      x\\
      \textcolor{white}{0000 0000} 0010 0111\\
      ------ \\
      \textcolor{white}{0000 0000} 1000 1110\\
      \textcolor{white}{0000 000}1 0001 110\\
      \textcolor{white}{0000 00}10 0011 10\\
      \textcolor{white}{000}1 0001 110\\
      = \\
      \textcolor{white}{000}1 0101 1010 0010
    \end{column}
  \end{columns}
\end{frame}

\begin{frame}
  This lecture covers
  \begin{itemize}
  \item basic definitions related to division and modulo operation
  \item prime factorization
  \item a fundamental theorem stating a fact related to prime numbers
    called the Fermat's Little Theorem
  \end{itemize}
\end{frame}

\begin{frame}\frametitle{Definitions: divisibilty}
  Let's start with basic definitions.
  \begin{itemize}
  \item We say that ``$a$ divides $b$'', ``$b$ is divisible by $a$'',
    or ``$b$ is a multiple of $a$'' if there exists an integer $k$
    such that $b = ka$.  In this case, we write \[a|b.\]
  \item If it's not the case, we write $a\not| b$.
  \item When $a$ does not divides $b$, there is a remainder.  We say
    that $r$ is a remainder of dividing $b$ by $a$ if $0\leq r<a$ and
    there exists integer $k$ such that $b = ka + r$.  We also write
    \[ r = a\mod b.\]
    \begin{itemize}
    \item $10\mod 3 = 1$, $10\mod 2 = 0$, $10\mod 15=10$
    \item $-10\mod 3 = 2$, $-10\mod 15=5$
    \end{itemize}
  \end{itemize}
\end{frame}

\begin{frame}\frametitle{The modulo operation}
  \begin{tcolorbox}
    For integers $a$ and $b$ and positive integer $q$, we have
    \begin{itemize}
    \item $(a+b)\;\bmod\;q = ((a\;\bmod\;q) + (b\;\bmod\;q))\;\bmod\; q$
    \item $(a-b)\;\bmod\;q = ((a\;\bmod\;q) - (b\;\bmod\;q))\;\bmod\; q$
    \item $(ab)\;\bmod\;q = ((a\;\bmod\;q) \times (b\;\bmod\;q))\;\bmod\; q$
    \end{itemize}
  \end{tcolorbox}

  Examples:
  \begin{itemize}
    \item $(14 + 7)\;\bmod\; 5 = ((14\;\bmod\;5) + (7\;\bmod\;5))\;\bmod\;5 = (4+2)\;\bmod\;5 = 1$
    \item $(14\cdot 7\cdot 13\cdot 19)\;\bmod\; 5 = ((14\;\bmod\;5)\cdot(7\;\bmod\;5)\cdot(13\;\bmod\;5)\cdot(19\;\bmod\;5))\;\bmod\;5 = (4\cdot 2\cdot 3\cdot 4)\;\bmod\;5 = 96\;\bmod\;5=1$
  \end{itemize}

  {\small These facts are really helpful when you try to compute
    $x\;\bmod\; y$ when $x$ is very large compared to $y$ and $x$ is a
    result of many operations of small numbers.  In this case, we can
    keep moduloing intermediate results to keep them under $y$.
  
    You will prove these properties in your homework.
  }
\end{frame}

\begin{frame}\frametitle{Definitions: primes}
  \begin{tcolorbox}
    An integer $p$ is a {\bf prime} if $p>1$ and $p$ has only 4
    factors: $1,-1,p,$ and $-p$.  If a number larger than $1$ is not a
    prime, we say that it is a {\bf composite}.
  \end{tcolorbox}

  Prime numbers are very fascinating.  There are many facts that have
  proved about them.  E.g., we looked at Euclid's proof that there are
  infinitely many primes.  Here's another one by Euclid:

  {\small
    \begin{tcolorbox}
      \textcolor{blue}{Theorem:} For any positive integer $n$, there
      are $n$ consecutive composites.
      
      \textcolor{blue}{Proof:} Let $m=n+1$.  Consider
      \[
      (m!+2),(m!+3),\ldots,(m!+m).
      \]
      Note that these $m-1=n$ numbers are composite because for any
      $1\leq i\leq m$, $i|m!$, $i|i$, and thus,
      $i|(m!+i)$. $\blacksquare$
    \end{tcolorbox}
  }
\end{frame}

\begin{frame}\frametitle{Prime factorization}
  It is known since the Greeks that if you have a composite $n$, you
  can factor it as a product of prime numbers.  For example, you can
  write
  \[ 140 = 2\times 2\times 5\times 7. \]

  \vspace{0.1in} Not only you can do that, but the Greeks also know
  that you can do that in only one way (except the permutation of the
  prime factors).  Many proofs we shall introduce later on require
  this fact, so we shall prove it next.
\end{frame}

\begin{frame}
  \textcolor{blue}{Theorem:} A prime factorization of a positive
  number larger than 1 is unique.

  \begin{tcolorbox}[title=Proof]
  We shall prove by contradiction.  We also
  use an argument usually referred to as the ``minimal criminal''
  argument, for which we use the fact that we can pick the minimal
  element of a subset of positive integers.

  \vspace{0.1in}
  Assume that the statement is false, i.e., there exists a positive
  integer with two prime factorizations.  Let $n$ be the smallest
  integer with that property.  I.e., $n$ has at least two prime
  factorizations:
  \[ n = p_1 \cdot p_2 \cdots p_r, \]
  and
  \[ n = q_1 \cdot q_2 \cdots q_s.\]
  \end{tcolorbox}
\end{frame}

\begin{frame}
  \begin{tcolorbox}[title=Proof (cont.)]
  To simplify our proof, we shall make a few assumptions.  Note that
  these assumptions do not change the actual assumption of the
  theorem.  When we make these types of assumption, we usually say
  that we make assumptions ``without loss of generality''.

  \vspace{0.1in} {\em 1st assumption.} We assume that the two prime
  factorizations do not share any primes, i.e., the sets
  $\{p_1,p_2,\ldots,p_r\}$ and $\{q_1,q_2,\ldots,q_s\}$ are disjoint.
  If this is not the case, we can divide $n$ by a common prime factor
  $p_i$ to obtain a smaller integer $n'$ with two prime
  factorizations.

  \vspace{0.1in} {\em 2nd assumption.} We also assume that $p_1$ is
  the smallest prime factor in both prime factorizations, i.e.,
  $p_1\leq p_i$ for all $1\leq i\leq r$ and $p_1 < q_j$ for all $1\leq
  j\leq s$.  Otherwise, we can switch the roles of prime
  factorizations $p_1p_2\cdots p_r$ and $q_1q_2\cdots q_s$.
  \end{tcolorbox}
\end{frame}

\begin{frame}
  \begin{tcolorbox}[title=Proof (cont.)]
  We shall proceed to show that the assumption leads to a
  contradiction by proving that there exists an integer $n'$ smaller
  than $n$ with two prime factorizations.

  Let's take $p_1$ and divide every prime $q_j$ with $p_1$.  Let $r_j$
  be each remainder, i.e., for $1\leq j\leq s$, let
  \[ r_j=q_j \Mod p_1.\]

  Since $p_1$ does not appear in the second prime factorization, it
  does not divide any $q_j$; thus, $r_j\neq 0$.
  
  Let
  \[ n' = r_1\cdot r_2\cdots r_s.\]
  From this definition, we can obtain one prime factorization of $n'$
  by combining all prime factorizations of all $r_j$.
  \end{tcolorbox}
\end{frame}

\begin{frame}
  \begin{tcolorbox}[title=Proof (cont.)]
  It is left to show that $n'=r_1\cdot r_2\cdots r_s$ has another
  prime factorization.  To do so, we shall prove that $p_1|n'$.

  \vspace{0.1in}
  Since $r_j=q_j\Mod p_1$, we can write
  \[ q_j=k_j\cdot p_1 + r_j, \]
  for some integer $k_j$.  Thus, we have that
  \[ n = (k_1p_1+r_1)(k_2p_1+r_2)\cdots(k_sp_1+r_s), \]
  which can be written as
  \[ n = K\cdot p_1 + r_1\cdot r_2\cdots r_s, \]
  for some integer $K$.  Since $p_1|n$, we have that $p_1|r_1\cdot
  r_2\cdots r_s$ (or, $p_1|n'$).
  \end{tcolorbox}
\end{frame}

\begin{frame}
  \begin{tcolorbox}[title=Proof (cont.)]
  Since $p_1|n'$, there exists a prime factorization of $n'$ that has
  $p_1$ as a factor.  To see that this is a different prime
  factorization, recall that all $r_j < p_1$ (because they are
  remainders of divisions by $p_1$).  This means that every prime in
  the first prime factorization is less than $p_1$, but the second
  prime factorization has $p_1$ as a factor.  Hence, they are
  different.

  \vspace{0.1in} This leads to the contradiction, because we now have
  a smaller integer $n'$ with more than one prime
  factorizations. $\blacksquare$
  \end{tcolorbox}
\end{frame}

\begin{frame}\frametitle{Fermat's Little Theorem}
  Fermat has another famous theorem which is very useful in number
  theory.  It can be stated as follows.

  \begin{tcolorbox}
    \textcolor{blue}{Theorem:} If $p$ is a prime and $a$ is an integer
    not divisible by $p$, we have that
    \[ a^{p-1}\Mod p = 1. \]
  \end{tcolorbox}

  To prove this theorem, we shall use the unique prime factorization
  theorem to prove a short lemma related to divisibility of binomial
  coefficients first.
\end{frame}

\begin{frame}
  \textcolor{blue}{Lemma:} If $p$ is a prime for any integer $1\leq
  k<p$, we have that
  \[
  p\Big|\binom{p}{k}.
  \]

  \textcolor{blue}{Proof:}  Note that
  \[\binom{p}{k}=\frac{p(p-1)(p-2)\cdots(p-k+1)}{k!}\]

  Since $p$ is in the product in the numerator,
  \[
  p|p(p-1)(p-2)\cdots(p-k+1).
  \]
  Now consider the denominator $k!$.
  Since it is a product of numbers less than $p$, its unique prime
  factorization does not contain $p$.  Hence, the fraction has $p$ as
  a factor; thus, $p|\binom{p}{k}$ as required. $\blacksquare$
\end{frame}

\begin{frame}\frametitle{Fermat's Little Theorem}
  We shall prove a different form of the theorem. (Its equivalence to
  the standard form shall be proved in the next lecture.)

  \begin{tcolorbox}
    \textcolor{blue}{Theorem:} If $p$ is a prime and for any integer
    $a$, we have that $p|a^p-a$.
  \end{tcolorbox}
  \begin{tcolorbox}[title=Proof]
    We shall prove by induction on $a$.  Let $P(a)$ be the statement
    that $p|a^p-a$.

    {\bf Base case:} Since $p|0^p-0$, $P(0)$ is true.
  \end{tcolorbox}
\end{frame}

\begin{frame}
  \begin{tcolorbox}[title=Proof (cont.)]    
    {\bf Inductive step:} Assume $P(k)$ is true, we shall prove that
    $P(k+1)$ is true.  From the binomial theorem, we have
    \[ (k+1)^p - (k+1) \qquad\qquad\qquad\qquad\qquad\qquad\qquad\qquad\qquad \]
    \begin{eqnarray*}
      &=&
      \left(k^p + \tbinom{p}{p-1}k^{p-1} + \tbinom{p}{p-2}k^{p-2}+\cdots + \tbinom{p}{1}k + 1\right)\\
      && - (k+1)\\
      &=&
      \left(k^p - k\right) + \left(\tbinom{p}{p-1}k^{p-1} + \tbinom{p}{p-2}k^{p-2}+\cdots + \tbinom{p}{1}k\right).
    \end{eqnarray*}

    The first term is divisible by $p$ from the induction hypothesis.
    The second term is a sum of terms divisible by $p$ from the
    previous lemma.  Thus, $p|(k+1)^p-(k-1)$, implying $P(k+1)$ as
    required. $\blacksquare$
  \end{tcolorbox}
\end{frame}

