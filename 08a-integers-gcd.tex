\newcommand{\lecturetitle}[1]{
  \title{01204211 Discrete Mathematics \\ #1}
  \author{Jittat Fakcharoenphol}
  \frame{\titlepage}
}
\newcommand{\Mod}{\,\bmod\,}

\lecturetitle{Lecture 8a: Integers and GCD} 

\begin{frame}
  \frametitle{Number theory: integers and divisibility}

  In the third part of the course, we study number theory, a
  once-thought-to-be ``useless'' branch of mathematics.

  \vspace{0.2in}
  \pause
  {\bf Why?}
  \pause
  
  \begin{itemize}
  \item The topic itself is very very beautiful.
  \item It has many applications in cryptography and error correcting codes.
  \end{itemize}

  \vspace{0.1in}
  \pause
  {\bf We will cover:}
  \begin{itemize}
  \item Basic concepts of divisibility, prime numbers, and congruence.
  \item How to quickly check if a number is prime.
  \item How to essentially perform ``division'' with integers,
    allowing us to work with important and useful objects like
    polynomials using only integers.  \pause
  \item Applications like cryptography (RSA), secret sharing, erasure
    codes and error correcting codes.
  \end{itemize}
    
\end{frame}

\begin{frame}
  \frametitle{Definitions}

  \begin{block}{Definition (divisibility)}
    We say that an integer $a$ \textcolor{red}{\bf divides} $b$ or $b$
    \textcolor{red}{\bf is divisible by} $a$ if there exist an integer
    $k$ such that
    \[
    b = ak.
    \]
    If it is the case, we also write $a|b$.  We also say that $a$ is a
    {\bf divisor} (or a {\bf factor}) of $b$.

    On the other hand if $a$ does not divide $b$, we write $a\not|b$.
  \end{block}
\end{frame}

\begin{frame}
  \frametitle{Examples}

  If $a|b$ and $a|c$, prove that $a|(b+c)$.
  \vspace{1.5in}
  \pause
  
  If $a|b$ and $b|c$, prove that $a|c$.
  \vspace{1.5in}
\end{frame}

\begin{frame}
  \frametitle{Remainder}

  \begin{block}{Defintion (remainder)}
    The \textcolor{red}{\bf remainder} of the division of $b$ with $a$
    is an integer $r$ when there exists an integer $q$ such that
    \[
    b = qa + r,
    \]
    where $0\leq r < a$.

    \pause We refer to $q$ as the \textcolor{red}{\bf quotient} of the
      division.
  \end{block}

  \pause

  {\bf Examples:}
  \vspace{.5in}

  \pause We use operator {\tt mod} to denote an operation for finding
  the remainder of a division.  I.e., $a \bmod b$ is the remainder of
  dividing $a$ with $b$.
\end{frame}

\begin{frame}
  \frametitle{Examples}

  Let $r$ be the remainder of the division of $b$ by $a$.  Assume that
  $c|a$ and $c|b$.  Prove that $c|r$.

  \vspace{2.5in}
\end{frame}

\begin{frame}
  \frametitle{More examples}

  For every integer $a$, $a-1 | a^2-1$.

  \vspace{2.5in}
\end{frame}


\begin{frame}
  \frametitle{Primes}

  \begin{block}{Definition (primes)}
    \begin{itemize}
    \item An integer $p>1$ is a \textcolor{red}{\bf prime} if its
      divisors are only $p$, $-p$, $1$, and $-1$.
    \item If an integer $n>1$ is not a prime, it is called a
      \textcolor{red}{\bf composite}.
    \item Note: $1$ is not a prime and also not a composite.
    \end{itemize}
  \end{block}

\end{frame}

\begin{frame}[fragile]\frametitle{Algorithm for testing primes}
  Recall our {\tt CheckPrime2} algorithm
  
  \begin{tcolorbox}
  {\small
\begin{verbatim}
  Algorithm CheckPrime2(n):  // Input: an integer n
      if n <= 1:
          return False
      let s = square root of n
      i = 2
      while i <= s:
          if n is divisible by i:
              return False
          i = i + 1
      return True
\end{verbatim}
  }
  \end{tcolorbox}
  
  How fast can it run? \pause Note that $s = \sqrt{n}$; therefore, it
  takes time $O(\sqrt{n})$ to run.
\end{frame}

\begin{frame}
  \frametitle{Efficient algorithms}

  Is $O(\sqrt{n})$ for checking a prime number efficient?

  \pause

  What is the ``size'' of the input to the problem?  \pause The input
  contains one integer $n$; is the size of the input just 1?

  \pause When working with input consisting only a few numbers, we
  typically use the number of bits.  For integer $n$, the number of
  bits of $n$ is $\lceil \log_2 n\rceil$.

  \pause
  {\tiny
  \begin{tabular}{r|r|r}
    $n$ & number of bits of $n$ & $\sqrt{n}$ \\
    \hline
    2 & 1 & 1.414 \\
    4 & 2 & 2 \\
    16 & 4 & 4 \\
    1,024 & 10 & 32 \\
    1,048,576 & 20 & 1,024 \\
    1,125,899,906,842,624 & 50 & 33,554,432 \\
    1,267,650,600,228,229,401,496,703,205,376 & 100 & 1,125,899,906,842,624 \\
  \end{tabular}
  }

  \vspace{0.2in} \pause {\small Side note: Recall that the first step
    in RSA is to find a pair of large primes.  Typically we want them
    to be of size in the {\em thousand} bits.}
\end{frame}

\begin{frame}[fragile]
  \frametitle{Greatest Common Divisors (GCD)}

  \begin{block}{Definition (GCD)}
    For integers $x$ and $y$, the \textcolor{red}{\bf greatest common
      divisor} (or GCD) of $x$ and $y$ is the largest integer $g$ such
    that $g|x$ and $g|y$.  We refer to it as $gcd(x,y)$.
  \end{block}

  \vspace{0.1in}
  \pause
  A simple way to find $gcd(x,y)$:

  \begin{tcolorbox}
  {\small
\begin{verbatim}
  g = min(x,y)
  while (x mod g != 0) or (y mod g != 0):
    g -= 1
  return g
\end{verbatim}
  }
  \end{tcolorbox}
  
\pause What is the running time of this algorithm? \pause
Does it run in polynomial time on the size of the input?
\end{frame}

\begin{frame}[fragile]
  \frametitle{Euclid's algorithm}
  
  \begin{tcolorbox}
  {\small
\begin{verbatim}
  Algorithm Euclid(x,y):
    if x mod y == 0:
      return y
    else:
      return Euclid(y, x mod y)
\end{verbatim}
  }
  \end{tcolorbox}

  \pause
  Let's see how it works with $Euclid(12311,24324)$:
  
  {\footnotesize
    Euclid( 12311, 24324) \\
    Euclid( 24324, 12311) \\
    Euclid( 12311, 12013) \\
    Euclid( 12013,   298) \\
    Euclid(   298,    93) \\
    Euclid(    93,    19) \\
    Euclid(    19,    17) \\
    Euclid(    17,     2) \\
    Euclid(     2,     1) \\
  }
\end{frame}

\begin{frame}
  \frametitle{Proofs}

  We have to prove two properties:
  \begin{itemize}
  \item For any integers $x$ and $y$, $\mathrm{Euclid}(x,y) = gcd(x,y)$.
  \item The running time of $\mathrm{Euclid}$.
  \end{itemize}

  \vspace{0.2in}

  \pause Note that when $x<y$, $\mathrm{Euclid}(x,y)$ just calls
  itself with both arguments swapped, i.e., $\mathrm{Euclid}(y,x)$.
  After that, in each call, $x$ is always larger than $y$.  For
  simplicity of the analysis, we shall work only with the case that $x
  > y$.
\end{frame}

\begin{frame}
  \begin{theorem}
    For any integers $x$ and $y$ such that $x>y$,
    $\mathrm{Euclid}(x,y) = gcd(x,y)$.
  \end{theorem}
  \begin{proof}
    {\small

      We prove using strong induction.  For the base case, note that
      when $y|x$, $gcd(x,y)=y$; therefore, the base case of the
      algorithm is correct.

      Our induction hypothesis is: for any $x'<x$ and $y'<y$,
      $\mathrm{Euclid}(x',y')=gcd(x',y')$.
    
      Now assume that $y\not| x$.  The Euclid algorithm returns
      $\mathrm{Euclid}(y, x\bmod y)$ as the gcd.  Note that $y < x$
      and $x\bmod y < y$.  Therefore, we can use the I.H. to claim that
      \[
      \mathrm{Euclid}(y, x\bmod y) = gcd(y, x\bmod y).
      \]

      Thus, we are left to show that
      \[
      gcd(x,y) = gcd(y, x\bmod y).
      \]
    }
  \end{proof}
\end{frame}

\begin{frame}
  \frametitle{What is $x \bmod y$?}

  \pause Let $\lfloor a \rfloor$ be the largest integer $a'$ such that
  $a'\leq \lfloor a\rfloor$.
  
  \pause
  \[
  x\bmod y = x - \left\lfloor\frac{x}{y}\right\rfloor\cdot y
  \]
\end{frame}

\begin{frame}
  \begin{lemma}
    If $a|x$ and $a|y$, then $a|x \bmod y$.
  \end{lemma}
  \vspace{0.2in}
  \pause
  \begin{lemma}
    $gcd(x,y)=gcd(y, x\bmod y)$
  \end{lemma}
  \vspace{1.5in}
\end{frame}

\begin{frame}
  \frametitle{How many recursive calls does Euclid's algorithm makes?}
  \pause

  Consider Euclid$(x,y)$:
  \begin{itemize}
  \item If we start with $x<y$, the next calls will always have that
    $x>y$; so we have at most one call with $x<y$.
    \pause
  \item When can we decrease the value of $x$ or $y$ in the calls?
    \pause
  \item When $y\leq x/2$, when Euclid$(x,y)$ calls Euclid$(y,x\bmod y)$ the
    first argument decreases by half.
    \pause
  \item How about when $y> x/2$? \pause \\
    {\small
    Euclid$(x,y) \Rightarrow$ \pause
    Euclid$(y, x\bmod y) \Rightarrow$ \pause
    Euclid$(x\bmod y, y\bmod (x\bmod y))$} \pause
    Note that in this case, $x\bmod y\leq y/2\leq x/2$. \pause
    Thus, after two recursive calls, the first argument decreases by half.
  \item How many times can that happen?
  \item The first argument can decrease by a factor of two for at most
    $\log x$ times.  Therefore, the Euclid algorithm runs in time
    $O(\log\max\{x,y\})=O(\log x + \log y)$. 
  \end{itemize}
\end{frame}

\begin{frame}[fragile]
  \frametitle{Computing power}

  How fast can we compute $x^y$?

  \pause
  \begin{tcolorbox}
  {\small
\begin{verbatim}
  Algorithm power(x,y):
    a = 1
    for i = 1,2,...,y:
      a *= x
    return a
\end{verbatim}
  }
  \end{tcolorbox}
  
  \pause What is the running time? \pause Is it efficient?
\end{frame}

\begin{frame}[fragile]
  \frametitle{Repeated squaring}

  If $y$ is a power of two, we can find $x^y$ using small number of
  multiplications using repeated squaring.  E.g.,
  \[
  x^{16} = (x^8)^2 = ((x^4)^2)^2 = (((x^2)^2)^2)^2.
  \]
  \pause

  \begin{tcolorbox}
  {\small
\begin{verbatim}
  Algorithm power(x,y):  // for y=2^k
    if y == 0:
      return 1
    else:
      a = power(x, y / 2)
      return a*a
\end{verbatim}
  }
  \end{tcolorbox}
\end{frame}

\begin{frame}[fragile]
  \frametitle{Repeated squaring (general $y$)}

  \begin{tcolorbox}
  {\small
\begin{verbatim}
  Algorithm power(x,y): 
    if y == 0:
      return 1
    else:
      a = power(x, floor(y / 2))
      if y mod 2 == 0:
        return a*a
      else
        return a*a*x
\end{verbatim}
  }
  \end{tcolorbox}

  \pause
  What is the number of recursive calls?

  \pause
  What is the running time?

  \pause While the number of multiplication is small, the numbers
  involved is huge as $x^y$ has $y\log x$ bits.  Computing $x^y$
  exactly definitely takes a long time.
\end{frame}

\begin{frame}[fragile]
  \frametitle{Repeated squaring (general $y$, mod $n$)}

  Computing $x^y \bmod n$:
  
  \begin{tcolorbox}
  {\small
\begin{verbatim}
  Algorithm power(x,y,n): 
    if y == 0:
      return 1
    else:
      a = power(x, floor(y / 2)) mod n
      if y mod 2 == 0:
        return a*a mod n
      else
        return a*a*x mod n
\end{verbatim}
  }
  \end{tcolorbox}

\end{frame}

